% Stará kronika TJ Sokol Poruba
% Typeset from transcript_stara_kronika.txt

%%%%%%%%%%%%%%%%%%%%%%%%%%%%%%%%%%%%%%%%%%%%%%%%%%%%%%%%%%%%%%%%%%%%%%%%%%%%%%%
% ÚVODNÍ STRÁNKY
%%%%%%%%%%%%%%%%%%%%%%%%%%%%%%%%%%%%%%%%%%%%%%%%%%%%%%%%%%%%%%%%%%%%%%%%%%%%%%%

\section*{Historické pohlednice Sokolovny}
\phantomsection
\addcontentsline{toc}{section}{Historické pohlednice Sokolovny}
\markright{Historické pohlednice Sokolovny}

\begin{figure}[H]
\centering
\adjustimage{width=\textwidth, height=0.3\textheight, keepaspectratio}{images/stara_kronika_transcript_fotky/1a.png}
\adjustimage{width=\textwidth, height=0.3\textheight, keepaspectratio}{images/stara_kronika_transcript_fotky/1b.png}
\adjustimage{width=\textwidth, height=0.3\textheight, keepaspectratio}{images/stara_kronika_transcript_fotky/1c.png}
\caption{Stránka zobrazuje tři historické černobílé pohlednice budovy Sokolovny v~Porubě. Každá pohlednice ukazuje budovu z~různých úhlů a~s~různými popisky.}
\end{figure}

\noindent\textbf{Text na pohlednicích:}

\textbf{Horní pohlednice:}\\
PORUBA. Sokolovna.\\
r. 1901.\\
16950

\textbf{Prostřední pohlednice:}\\
Poruba. Sokolovna.

\textbf{Spodní pohlednice:}\\
RESTAURACE\\
SOKOLOVNA\\
ZAHRADA\\
A KUŽELNA\\
Poruba.\\
Spisar\\
Ostrava

\clearpage


\section*{Bratrům a sestrám porubského Sokola}
\phantomsection
\addcontentsline{toc}{section}{Bratrům a sestrám porubského Sokola}
\markright{Bratrům a sestrám porubského Sokola}

Česká obec sokolská uložila každé jednotě po převratu sepsati spolkovou kroniku. Tento nesnadný úkol byl svěřen podepsanému jako nejstaršímu členu a~spoluzakladateli Těl.\ jednoty Sokol v~Porubě. Obtíže byly v~tom, že od založení jednoty uplynula dlouhá řada let, plných rušné a~mnohostranné činnosti, o~níž nebylo buď žádných zápisů nebo byly neúplné; také lidská paměť není vždycky přesná a~spolehlivá. O~působení a~vlivu mnohých osob na rozvoj jednoty buď ve smyslu příznivém nebo nepřátelském nebylo možno podrobně pojednati, jednak že by látka příliš vzrostla a~jednak z~toho důvodu, že by vylíčení mnohé neblahé události nepříjemně se dotklo mnohých osob, které dosud žijí a~zatím svůj poměr k~jednotě změnily. --- Snahou podepsaného bylo vylíčiti vše pokud možno přesně a~pravdivě. Jestli se mu to nepodařilo, stůje na omluvu, že každé lidské dílo je nedokonalé. Jeho nástupci je možno dílo doplniti, případně opraviti. Kronika je napsána do r.~1930, kdy pisatel odešel z~Poruby a~k~dalším záznamům neměl spolehlivých zpráv.

Podepsaný přeje milované jednotě porubské, se kterou zažil tolik slunných i~bouřlivých dnů, s~níž se těšil i~truchlil, aby ve své činnosti neochabovala, nýbrž dále kráčela mužně neochvějně vpřed za svým cílem: povznésti tělesné i~duševní síly českého člověka.

\begin{center}
\textbf{Nazdar!}
\end{center}

\vspace{1cm}
\noindent Svinov v~srpnu 1933.\\[0.5cm]
\textbf{Josef Bárta},\\
\textit{řídící učitel v.v.}

\clearpage

%%%%%%%%%%%%%%%%%%%%%%%%%%%%%%%%%%%%%%%%%%%%%%%%%%%%%%%%%%%%%%%%%%%%%%%%%%%%%%%
% POMĚRY PŘED ZALOŽENÍM SOKOLA
%%%%%%%%%%%%%%%%%%%%%%%%%%%%%%%%%%%%%%%%%%%%%%%%%%%%%%%%%%%%%%%%%%%%%%%%%%%%%%%

\section*{Poměry v~Porubě před založením Sokola}
\phantomsection
\addcontentsline{toc}{section}{Poměry v~Porubě před založením Sokola}
\markright{Poměry v~Porubě před založením Sokola}

Obec Poruba čítala v~letech osmdesátých minulého století okolo 800 obyvatel české národnosti, avšak bez uvědomění národního. Kněží i~učitelé, odchovaní německými školami, lnuli k~němčině, četli německé knihy i~časopisy a~bavívali se mezi sebou i~s~úředníky panství hraběte Wilczka většinou německy. Starostou byl buď rolník nebo panský úředník, tajemníkem učitel nebo člověk němčiny znalý, který německým úředním dopisům rozuměl a~mohl s~úřady, jak si toho přály, v~německé řeči jednati. Protokoly čili zápisy o~schůzích byly německé, jakož i~úřední razítka. Úřední knihy školské, jako matriky, katalogy se vedly jen německy. Zkrátka němčina vládla všude a~byla ve vážnosti u~lidu, poněvadž mu poskytovala mnohých výhod jak ve styku s~úřady, tak ve vojenské službě a~poněvadž bez znalosti němčiny nemohl se domoci řádného místa ve státních a~panských službách.

Avšak již v~té době vyskytli se jednotlivci, kteří tento stav národního ponížení a~neuvědomělosti těžce nesli. Stykem s~některými vlastenci z~okolních obcí, četbou českých knih a~novin, hlavně Opavského Besedníka, později Op.\ Týdenníka, vydávaného horlivým redaktorem Janem Zacpalem, probudilo se jich národní vědomí a~vzplanula láska k~vlasti. Mezi prvními byli: Ambrož Zdražila, Josef Komár, Antonín Besta, Josef Tichý, kteří se scházívali o~národních otázkách debatovali a~četbou novin se vzdělávali. Josef Komár, jenž byl dovedným stolařem, vystěhoval se později do Ameriky, tři ostatní stali se uvědomělými a~vlivnými rolníky porubskými. Zdražila způsobil, že bratr jeho synov.\ Josef Martiník, nešel studovat na německé školy, jak tomu dosud bývalo, nýbrž na slovanské gymnasium do Olomouce. Jako uvědomělý student způsobil, že o~prázdninách roku 1884 bylo v~Porubě v~hostinci Fr.\ Švidrnocha sehráno první české divadlo; byl hrán kus \enquote{Mlynář a~jeho dítě} od Raupacha. Účinkovali v~něm místní studenti (Martiník, Pavlík, Josef Bárta), děvčata, z~okolí Ludvík Šimoš, správce školy v~Malé Lhotě se svou sestrou Marií; oba hráli titulní úlohy s~velikým zdarem. Účast veliká z~místa i~okolí. Příštího roku hrálo se pose. ---

Uvědomění národnímu ve Slezsku velmi nápomohlo české gymnasium v~Opavě, zřízené r.~1883, kam chodívali studovat i~hoši z~Poruby.

Zdejší jednotřídní škola byla roku 1889 rozšířena na dvojtřídku. Mladým učitelem se stal František Matlášek, který převzal tajemnictví obce a~vedl je na přání uvědomělého a~řádného starosty obce Ant.\ Besty, úplně česky. Ant.\ Besta založil 22./2.\ 1893 český hasičský sbor, místní kooperator P.~Jan Matonoha založil 1./10.\ 1893 Čtenářskou Besedu v~hostinci u~Františka Klosa, kterou vedl v~duchu klerikálním. V~roce 1894 1.~března odešel na odpočinek po dlouholetém působení místní nadučitel Petr Komárek, německého smýšlení, a~s~ním zároveň se dal přesadit z~Poruby učitel Matlášek.

Místo nadučitelské obdržel Josef Bárta, narozený 21./7.\ 1869 ve Vřesině a~vychovaný od r.~1873 v~Porubě, kamž jeho otec, rolník Ignác Bárta, se přistěhoval. Mladším učitelem se stal Jan Dedek, narozený roku 1873 v~Kotouči u~Zlína na Moravě. Jsa byl velmi milý a~čilý člověk a~zvláštní přítel tělocviku. V~nedostatku tělocvičného nářadí vyrobil si sám hrazdu a~ve školním dvoře na ní cvičíval. Brzy se vybavil s~návrhem, aby v~Porubě byla založena tělocvičná jednota Sokol. Byl to návrh na tu dobu dost odvážný a~sice z~různých příčin.

Nehledě na to, že německé úřady nepříznivě se dívaly na sokolské jednoty, i~obyvatelé neměli pro něho málo porozumění. V~obci naší, jak řečeno, byly v~r.~1893 založeny již 2~spolky, které dosti členů soustředily, proto se nemohlo na mnoho členů v~naší poměrně malé obci počítati. K~tomu spolek Sokol vyžaduje mnohé oběti a~vydání na nářadí, kroje a~p.~d. Dosud v~celém Slezsku byla pouze v~Opavě sokolská jednota, založená r.~1884.

Mnozí také namítali, že sokolské cvičení nemá na vesnici významu, poněvadž mladíci dosti se zdržují na vzduchu a~cvičí tělo těžkou prací na poli nebo v~dílnách. Než přes tyto a~jiné námitky a~pochybnosti zvítězil cit národní a~myšlenka pokroku, a~tak vlivem Jana Dedka a~za souhlasu nadučitele Josefa Bárty, starosty obce Ant.\ Besty, Ambrože Zdražily, Josefa Tichého a~jiných došlo v~červenci r.~1894 ku založení tělocvičné jednoty Sokol v~Porubě.

\clearpage

%%%%%%%%%%%%%%%%%%%%%%%%%%%%%%%%%%%%%%%%%%%%%%%%%%%%%%%%%%%%%%%%%%%%%%%%%%%%%%%
% ROK 1894 - ZALOŽENÍ
%%%%%%%%%%%%%%%%%%%%%%%%%%%%%%%%%%%%%%%%%%%%%%%%%%%%%%%%%%%%%%%%%%%%%%%%%%%%%%%

\phantomsection
\section{Rok 1894 --- Založení jednoty}

Ustavující valná hromada Sokola konala se v~hostinci Františka Švidrnocha dne 15.~července 1894. Volba výboru vykonána byla aklamací. Zvoleni byli:

\begin{description}
\item[Starostou:] Josef Tichý, rolník v~Porubě.
\item[Jeho náměstkem:] Josef Bárta, nadučitel v~Porubě. \checkmark
\item[Pokladníkem:] Rudolf Still, šafář u~velkostatku v~Porubě.
\item[Hospodářem:] Antonín Besta, rolník a~starosta obce.
\item[Členové výboru:] MUDr Jan Moravec, lékař v~Klimkovicích,\\
    Frydolín Strnad, mlynář ve Svinově,\\
    František Švidrnoch, maj.\ hostince v~Porubě,\\
    František Martiník, hostinský ve Vřesině,\\
    Ambrož Zdražila, rolník v~Porubě.
\item[Dozorci cvičení:] Josef Vilts, rolník v~Porubě,\\
    Adolf Besta, domkář.
\item[Členové oblekové komise:] Ambrož Zdražila, Josef Bárta, Jan Dedek.
\item[Náčelníkem a~jednatelem:] Jan Dedek, ml.\ učitel v~Porubě. \checkmark
\end{description}

\noindent Členů bylo místních 19, z~okolních obcí 21, úhrnem \textbf{40}.

Za cvičební místnost určen hostinský sál za souhlasu majitele hostince Fr.\ Švidrnocha i~nájemce hostince Josefa Rokyty, kterýž se také uvolil propůjčiti jednotě ke schůzím vhodnou místnost, již slíbil vytápěti a~osvětlovati. Scházeti se měli členové v~sobotu a~v~neděli, cvičení se mělo konati každou neděli o~4~hod.\ odpoledne.

\phantomsection
\subsection{Přijaty tyto volné návrhy:}

\begin{enumerate}
\item Jednota se zúčastní veřejného cvičení, pořádaného jednotou mor.\ ostravskou v~Děhylově 22.~července t.~r.
\item Žerď a~trubka náčelnická se objednají od Gubče.
\item Jednota se přihlásí k~župě středomoravské.
\item Jednota souhlasí se cvičením omladiny (dorostu).
\item Dá se vytisknouti prosba k~jednotám o~podporu.
\item Podá se žádost župě i~Č.O.S.\ o~darování starého cvičebního nářadí.
\item Za vykání se zaplatí pokuta 5~krejcarů.
\end{enumerate}

\phantomsection
\subsection{Odchod br.\ Jana Dedka}

Tak byla založena jednota, které bylo souzeno těžce se probíjeti a~zápasiti s~překážkami různého druhu. První nehoda ji stihla tím, že br.\ Dedek byl 1.~září t.r.\ přeložen za učitele do Polanky u~Klimkovic. Na žádost výboru podržel sice funkci náčelníka i~jednatele, ale docházka z~Polanky do Poruby byla mu hodně obtížná, zvláště v~zimě po zasněžených a~k~nám skoro nepřístupných cestách. Než přicházel ochotně dvakráte v~týdnu, aby řídil cvičení a~život v~jednotě.

\phantomsection
\subsection{Cvičení ve stolarské dílně}

Poněvadž hostinský Jos.\ Rokyta se k~jednotě nepříznivě choval, usnesla se jednota na valné hromadě 27.12.1894 hostinec opustiti a~cvičiti zatím ve stolarské dílně br.\ Ignáta Blažeje, kterou si za humny blízko otcova domku zřídil a~jednotě večer dal k~disposici. A~tak se scházeli členové jednoty večer do dílny, kterou po práci učni uklidili, ku cvičení a~zábavě. Cvičilo se pilně, zvláště na nově zakoupeném koni. Po cvičení se rozproudila milá zábava, kořeněná zpěvem sokolských písní a~hojnými vtipy.

\clearpage

%%%%%%%%%%%%%%%%%%%%%%%%%%%%%%%%%%%%%%%%%%%%%%%%%%%%%%%%%%%%%%%%%%%%%%%%%%%%%%%
% ROK 1895
%%%%%%%%%%%%%%%%%%%%%%%%%%%%%%%%%%%%%%%%%%%%%%%%%%%%%%%%%%%%%%%%%%%%%%%%%%%%%%%

\phantomsection
\section{Rok 1895}

Ve valné hromadě 5.~ledna byl zvolen jednatelem br.\ Vilém Kavčík, který se stal po br.\ Dedkovi učitelem v~Porubě. Na téže valné hromadě usneseno, aby se 10.~února t.r.\ konal v~Klimkovicích v~panském hostinci sokolský ples s~ukázkou cvičení a~proslovem br.\ MUDra Jana Moravce, lékaře v~Klimkovicích. Vše se podařilo.

Členové jednoty jeli dílem pěšky, dílem jeli na saních s~jinými příznivci neb ženami a~v~přestávce o~plese cvičili koně. Po cvičení promluvil pěkně k~četným hostům v~Klimkovicích br.\ Moravec o~významu Sokola pro český národ. --- Až do prázdnin vyvíjel se život v~naší jednotě velmi slibně, když tu nová nehoda ji stihla.

Br.\ Dedek, jenž byl jenom výpomocným učitelem, byl nucen učitelství se vzdáti; odešel z~Polanky a~přijal místo v~advokátní kanceláři dra Táborského v~Mor.\ Ostravě, čímž byl pro naši jednotu ztracen. Jeho jako dobrého borce a~u~všeho členstva velmi oblíbeného náčelníka nedovedl nikdo nahraditi. V~nedostatku jiné schopné síly převzal náčelnictví Josef Bárta.

\textbf{Divadlo.} V~tom roku sehrála jednota Sokol u~Švidrnochů divadlo, lidovou operettu \enquote{Lucifer, čili Zázračný elixír}. Návštěva veliká.

Koncem roku měla jednota \textbf{70 členů}, z~nich místních 38.

\clearpage

%%%%%%%%%%%%%%%%%%%%%%%%%%%%%%%%%%%%%%%%%%%%%%%%%%%%%%%%%%%%%%%%%%%%%%%%%%%%%%%
% ROK 1896
%%%%%%%%%%%%%%%%%%%%%%%%%%%%%%%%%%%%%%%%%%%%%%%%%%%%%%%%%%%%%%%%%%%%%%%%%%%%%%%

\phantomsection
\section{Rok 1896}

\phantomsection
\subsection{Jednota se vrací do hostince Fr.\ Švidrnocha}

Dne 1.~ledna 1896 převzal hostinec po Rokytovi br.\ Rudolf Hill, následkem čehož se vrátila jednota do hostince Švidrnochova.

Ve valné hromadě 5.~ledna 1896 byli zvoleni:

\begin{description}
\item[Starostou:] JUDr Rudolf Hess, advokát v~Klimkovicích.
\item[Náměstkem:] Antonín Besta, rolník v~Porubě.
\item[Náčelníkem:] Josef Bárta, nadučitel.
\item[Jednatelem:] Fr.\ Vl.\ Sigmund, obchodník v~Porubě.
\item[Pokladníkem:] Josef Vilts, rolník v~Porubě.
\item[Hospodářem:] Jan Malík.
\item[Členy výboru:] Dr Jan Moravec, lékař v~Klimkovicích,\\
    Josef Tichý, rolník v~Porubě,\\
    Matěj, četník ze Hrabové,\\
    Ig.\ Blažej, stolař v~Porubě (byl též cvičitelem),\\
    Josef Uherek, učitel ve Vřesině, který byl též předsedou zábavního odboru.
\end{description}

\noindent Usneseno předplatiti časopisy: \emph{Sokol}, \emph{Radhošť} (redaktor Frant.\ Tůma), \emph{Hospodář}.\\
Br.\ Hill předplatil \emph{Opavský Týdenník} a~\emph{Nazdar!}\\
Br.\ Sigmund předplatil \emph{Humoristické Listy}.\\
Br.\ Dr.\ Hess předplatil \emph{Těšínské Noviny}.

V~únoru 1896 koupila jednota od Čtenářské Besedy staré jeviště za 21~zlatých, na němž provedla několik her. Některým bratrům byly od firmy Zvolský v~Praze, objednány na účet jednoty výletní kroje, které měli bratři jednotě po částkách spláceti. Někteří tak činili, ale většinou spláceli nepravidelně, nebo nedali ničeho, takže proto měla jednota mnohé mrzutosti a~citelnou škodu.

Hrazda dřevěná, ocelí protažená koupena od Jana Sýkory v~Lomnici za 9~zl 50~kr.

Ku slavnosti Sokola v~Kroměříži vyslána čtyřčlenná deputace, slavnosti Sokola ve Vítkovicích zúčastnila se jednota v~kroji. Fotografie účastníků nachází se ve spolkové místnosti.

Dne 13.~září převzal jednatelství br.\ Josef Uherek, který v~jednotě vyvíjel horlivou činnost, zvláště pokud se týče pořádání divadel. Jeho přičiněním sehrána byla poprvé \emph{Maryša}, drama od bří Mrštíků. Maryšu hrála paní Gutvaldová ze Svinova, Francka její choť, Lízala br.\ Sigmund, Vávru br.\ Uherek.

Místo učitele Viléma Vavříka, který obdržel místo ve Starém Jičíně, přišel na porubskou školu učitel Adolf Sokol, narozený v~Hluboči, který se stal ihned členem jednoty a~po dobu svého 23tiletého pobytu ve zdejší obci zastával důležité funkce v~jednotě a~velmi mnoho k~jejímu rozvoji přispěl.

V~tomto roku zúčastnila se jednota veřejného cvičení Opavského Sokola.

16.~srpna pořádáno veřejné cvičení, které finančně skončilo schodkem, neboť vydání činilo 96~zl 48~kr, příjem 75~zl 07~kr.

Hrána byla 2~divadla: \emph{Maryša} a~\emph{Trhák a~dělník}.

Na paměť 40tiletého výročí úmrtí K.~Havlíčka Borovského pořádána 13.~prosince členská schůze s~přednáškou o~tomto velikém buditeli národním.

Na žádosti rozeslané r.~1895 jednotám o~příspěvky na stavbu sokolovny došlo od 26~jednot 120~zl 60~kr, od župy středočeské 5~zl, úhrnem 125~zl 60~kr.

Členů koncem roku \textbf{86}, z~nich místních 42.

\clearpage

%%%%%%%%%%%%%%%%%%%%%%%%%%%%%%%%%%%%%%%%%%%%%%%%%%%%%%%%%%%%%%%%%%%%%%%%%%%%%%%
% ROK 1897
%%%%%%%%%%%%%%%%%%%%%%%%%%%%%%%%%%%%%%%%%%%%%%%%%%%%%%%%%%%%%%%%%%%%%%%%%%%%%%%

\phantomsection
\section{Rok 1897}

Ve valné hromadě konané 2.~února byly 2~přednášky:
\begin{itemize}
\item Redaktor a~spisovatel br.\ Fr.\ Sokol-Tůma přednášel: \enquote{O~činnosti a~důležitosti sokolských jednot vnitř a~na venek.}
\item Br.\ Josef Gutvald, účetní parního mlýna ve Svinově, přednášel na thema: \enquote{Ni zisk, ni slávu!}
\end{itemize}

\noindent Zvoleni byli:

\begin{description}
\item[Starostou:] Dr Rudolf Hess.
\item[Náčelníkem:] Josef Uherek.
\item[Cvičitelem:] Gustav Husník ze Svinova.
\item[Jednatelem:] Adolf Sokol.
\item[Vyslancem do Č.O.S.:] Sokol-Tůma.
\item[Vyslancem do župy:] Dr Hess.
\item[Předsedou zábavního odboru:] Josef Bárta.
\end{description}

Ve výborové schůzi 6.~února bylo usneseno:
\begin{enumerate}
\item Knihu ze spolkové knihovny může si vypůjčiti na dobu 14~dní, pak platí za každý týden 2~kr.
\item Poněvadž bylo přijato za členy Sokola mnoho takových, kteří pak své povinnosti nekonali a~byli jednotě jenom na obtíž, usneseno, aby přihlášení byli teprve po tříměsíčním pobytu v~jednotě na výslovnou žádost přijati za členy.
\end{enumerate}

Toho roku vznikla neshoda mezi členstvem a~starostou Dr~Hessem, který následkem toho se starostenství vzdal a~z~jednoty vystoupil. Až do příští valné hromady ho zastupoval jeho náměstek br.\ Antonín Besta.

Poněvadž staré divadelní jeviště, koupené od Čtenářské besedy nijak nevyhovovalo, usneseno pořídit nové jeviště, k~čemuž výbor povolil obnos 50~zlatých. Zplnomocněni byli bři Adolf Sokol a~Josef Tichý, aby o~zřízení jeviště se starali.

Činnost toho roku pro vnitřní neshody byla nepatrná.

\clearpage

%%%%%%%%%%%%%%%%%%%%%%%%%%%%%%%%%%%%%%%%%%%%%%%%%%%%%%%%%%%%%%%%%%%%%%%%%%%%%%%
% ROK 1898
%%%%%%%%%%%%%%%%%%%%%%%%%%%%%%%%%%%%%%%%%%%%%%%%%%%%%%%%%%%%%%%%%%%%%%%%%%%%%%%

\phantomsection
\section{Rok 1898}

Valná hromada konána byla 2.~února. Zvoleni byli:

\begin{description}
\item[Starostou:] Josef Bárta.
\item[Náměstkem:] Josef Vilts.
\item[Náčelníkem:] Adolf Sokol.
\item[Jednatelem:] Fr.\ Vl.\ Sigmund.
\item[Pokladníkem:] Rudolf Hill.
\item[Ostatní členové:] Josef Tichý, Ant.\ Besta, Fr.\ Kozub, Frant.\ Kudela, Hajduk, Lazar Eduard.
\end{description}

Ve výborové schůzi 16.~února stanoveno, by členský lístek obdržel člen teprve po zaplacení příspěvků. Kroj se objedná členu teprve až aspoň polovici předem zaplatí. Po cvičení v~neděli budou se cvičiti sokolské písně.

Ve schůzi 16.~března usneseno staré jeviště prodati Čtenářské besedě za 10~zl. Cvičící členové budou platit ročního příspěvku 1~zl. Bude se odebírat \emph{Sokolský Věstník}.

Z~činnosti jednoty v~tomto roku dlužno uvésti: Jednota zúčastnila se počtem 14~členů v~kroji slavnosti Palackého ve Val.\ Meziříčí a~Hodslavicích, počtem 12~členů založení Sokola v~Přívoze, 10~členů zúčastnilo se veřejného cvičení Sokola ve Vítkovicích. Cvičilo se v~příhodných měsících dvakráte v~týdnu v~sále hostince br.\ Švidrnocha.

Po vypuštění nevhodných a~neplatících členů zbylo koncem roku:
\begin{itemize}
\item zakládajících členů 18
\item činných (cvičících) 12
\item přispívajících 18
\end{itemize}
Úhrnem \textbf{48}.

V~tom roku pořízeno také nové jeviště se 4~proměnami, jež namaloval Antonín Svoboda z~Příbora. Nápovědní budku daroval klempíř Jan Lužinský z~Klimkovic. Br.\ Fr.\ Sigmund daroval sádrové poprsí Tyrše a~Fügnera. Br.\ Josef Uherek odešel na učitelské místo do Přívozu.

\clearpage

%%%%%%%%%%%%%%%%%%%%%%%%%%%%%%%%%%%%%%%%%%%%%%%%%%%%%%%%%%%%%%%%%%%%%%%%%%%%%%%
% ROK 1899
%%%%%%%%%%%%%%%%%%%%%%%%%%%%%%%%%%%%%%%%%%%%%%%%%%%%%%%%%%%%%%%%%%%%%%%%%%%%%%%

\phantomsection
\section{Rok 1899}

\begin{description}
\item[Starostou zvolen:] Antonín Besta.
\item[Jeho náměstkem:] Josef Vilts.
\item[Náčelníkem:] Adolf Sokol.
\item[Jeho náměstkem:] Ig.\ Blažej.
\item[Jednatelem:] Fr.\ Vl.\ Sigmund.
\item[Pokladníkem a~zapisovatelem:] Josef Bárta.
\item[Ostatní členové:] Rudolf Hill (účetní), Eduard Lazar, Adolf Bárta (rolnický syn), Alois Besta (rolnický syn), Josef Tichý, Jan Muřijovský z~Klimkovic.
\end{description}

\phantomsection
\subsection{Z~činnosti sokolské}

Cvičeno 2~krát týdně v~příhodných měsících. Z~výboru vystoupil br.\ Sigmund, poněvadž převzal obchod v~Mor.\ Ostravě. Jednatelství vedl dále br.\ Adolf Sokol.

V~září přišel do Poruby na rozšířenou trojtřídní školu nový učitel Vladimír Košťřica, rodák z~Mokrých Lazců, který se ihned práce v~jednotě ujal a~po celou dobu svého pobytu v~Porubě buď jako jednatel nebo náčelník jí důležité služby prokazoval.

Výborových schůzí bylo 6.

Na veřejnost vystoupila jednota 4~krát a~sice:
\begin{enumerate}
\item při pozdvižení praporu Sokola v~Mor.\ Ostravě,
\item při cvičení Sokola ve Vítkovicích,
\item při výletu místního sboru hasičského,
\item při výletu Čtenářské besedy v~Porubě v~obou krojích.
\end{enumerate}

Pořádána byla 3~divadelní představení: Veselohry: \emph{Staří blázni}, \emph{Poklad}, \emph{Místecký Bedrník a~jeho chasa}.

Při divadlech vypomáhali: Leopoldina Bártová, choť nadučitele, Marie Coufalíková, učitelka ruč.\ prací, Jan Černý, poštovní úředník ze Svinova (výborný komik).

Předplaceny časopisy: \emph{Ostravský Obzor}, \emph{Lidové Noviny}, \emph{Věstník Sokolský}, \emph{Sokol}.

Někteří členové vystoupili, někteří byli vypuštěni, takže koncem roku bylo:
\begin{itemize}
\item zakládajících 16
\item činných (cvičících) 12
\item přispívajících 13
\end{itemize}
Úhrnem \textbf{41}.

\clearpage

%%%%%%%%%%%%%%%%%%%%%%%%%%%%%%%%%%%%%%%%%%%%%%%%%%%%%%%%%%%%%%%%%%%%%%%%%%%%%%%
% ROK 1900
%%%%%%%%%%%%%%%%%%%%%%%%%%%%%%%%%%%%%%%%%%%%%%%%%%%%%%%%%%%%%%%%%%%%%%%%%%%%%%%

\phantomsection
\section{Rok 1900}

\begin{description}
\item[Starostou zvolen:] Josef Vilts.
\item[Náměstkem:] Rudolf Hill.
\item[Náčelníkem:] Adolf Sokol.
\item[Náměstkem:] Ignác Blažej.
\item[Jednatelem:] Vladimír Košťřica.
\item[Ostatní členové:] Ant.\ Besta starší, Josef Tichý, Adolf Bárta, Frant.\ Buček, Josef Bárta, Ignác Blažej, Eduard Lazar.
\item[Přehlížiteli účtů:] Fr.\ Vl.\ Sigmund, Albin Juchelka.
\end{description}

Schůzí výborových konáno 10.

Za člena přistoupil Jan Uhlíř, poštovní expeditor, jenž v~úřadě poštovním zastupoval bratra Sigmunda. Jan Uhlíř, 20letý mladík, rodem z~Ivanovic na Hané, měl veliké hudební nadání. Vzdělával se ve škole skladatele Leoše Janáčka v~Brně a~složil některé písně sólové s~průvodem klavíru i~jiné hudební kusy, které jsme za jeho vedení nacvičili.

V~naší jednotě utvořil se hudební a~pěvecký kroužek, který za řízení br.\ Uhlíře pilně cvičil a~dosáhl v~krátké době obdivuhodné dokonalosti. Jako sopranistka vynikla slečna Marie Coufalíková, provdaná později za učitele Adolfa Sokola a~jako výborný tenorista se osvědčil Leoš Černín, učitel v~Krásném Poli, který obětavě skoro denně k~nám docházel a~s~námi cvičil. Nacvičili jsme několik hudebních kvartet, orkestrových čísel, smíšených sborů a~solových písní a~pak jsme pořádali koncerty v~blízkých i~vzdálenějších obcích. Zpívali jsme toho i~příštího roku v~Porubě, v~Pustkovci, ve Stiebovicích, ve Velké Polomi, Hrabyni, Zábřehu n/O.\ Bohumíně a~v~Příboře. Všude jsme měli veliký úspěch a~byli jsme přijati jako umělci. Zvláště v~Příboře, kde jsme koncertovali před profesory ústavu učitelského, zástupci města a~studenty, sklidili jsme veliký obdiv a~uznání.

Z~programu uvádím některá zpěvní čísla: Z~opery \emph{Prodaná nevěsta} píseň \enquote{Jak možná věřit}, Kvintet \enquote{Rozmysli si, Mařenko,} od Mendelssohna smíšený sbor \enquote{Květomluvu}, \enquote{Hradčany}, \enquote{Hubičku}, některé písně dirigenta Uhlíře (Květenského) a~jiné.

\bigskip

Toho roku pozbyli jsme vynikajícího člena naší jednoty: 2.~června zemřel br.\ MUDr Jan Moravec, lékař v~Klimkovicích, který po několikaleté úsilovné práci zbavil toto město německé správy a~stal se jeho prvním českým starostou. Dr~Moravec byl horlivým národním pracovníkem, oblíbeným lékařem, pravým lidumilem a~velikým příznivcem naší jednoty. Na své přání byl pochován v~Porubě. Účast na jeho pohřbu, jehož se naše jednota 19~členy v~kroji zúčastnila, byla velkolepá. Čest jeho památce!

Jednota zúčastnila se Matiční slavnosti v~M.\ Ostravě 14~členy, národní slavnosti ve Svinově 12 a~výletu místního sboru hasičského 11~členy. Na sletu sokolském v~Olomouci zastupoval jednotu jednatel br.\ Košťřica.

Sehrána byla 2~divadla: \emph{Palicova dcera} z~Tyla a~\emph{Obě nás}.

\bigskip

\noindent Příjem celoroční \hfill 246.25~K\\
Vydání \hfill 198.66~K\\
Zůstatek na rok 1901 \hfill \textbf{47.59~K}

\bigskip

Knihovna sokolská, pořízena většinou z~darovaných knih, čítala 250~svazků. Půjčovali však jen členové cvičící.

\noindent Členů bylo:
\begin{itemize}
\item zakládajících 15
\item činných 15
\item přispívajících 14
\end{itemize}
Úhrnem \textbf{44}.

Z~činných členů cvičilo nejvíce 12, nejméně 4. Cvičení bylo dvakráte v~týdnu: v~sobotu od 9--10~večer, v~neděli od $1\frac{1}{2}$--$2\frac{1}{2}$~odpol.

\clearpage

%%%%%%%%%%%%%%%%%%%%%%%%%%%%%%%%%%%%%%%%%%%%%%%%%%%%%%%%%%%%%%%%%%%%%%%%%%%%%%%
% ROK 1901 - STAVBA SOKOLOVNY
%%%%%%%%%%%%%%%%%%%%%%%%%%%%%%%%%%%%%%%%%%%%%%%%%%%%%%%%%%%%%%%%%%%%%%%%%%%%%%%

\phantomsection
\section{Rok 1901}

\begin{description}
\item[Starostou zvolen:] Josef Vilts.
\item[Jeho náměstkem:] Adolf Bárta.
\item[Náčelníkem:] Vladimír Košťřica.
\item[Jednatelem:] Adolf Sokol.
\item[Pokladníkem:] Josef Bárta.
\item[Hospodářem:] Ignác Blažej.
\item[Ostatní členové:] Ant.\ Besta, Josef Tichý, Eduard Lazar, Rudolf Hill, Fr.\ Vl.\ Sigmund.
\item[Náhradníci:] Josef Ehler, Adolf Besta ml., Albin Juchelka, Jan Uhlíř.
\end{description}

\phantomsection
\subsection{Stavba Sokolovny}

Tento rok je významný pro nás tím, že v~něm byla postavena Sokolovna.

Zároveň však --- bohužel --- vznikla toho roku neshoda mezi členy výboru, která vedla k~vystoupení důležitých jeho členů z~jednoty. To mělo velmi neblahé následky pro celá budoucí léta, neboť z~bývalých bratrů stali se největší protivníci a~nepřátelé jednoty, kteří kde mohli ji škoditi pomluvou, odmlouváním občanů, by nepřistupovali za členy, nechodili na podniky sokolské a~pod.

\textbf{Jak vznikla neshoda?}

Jako z~malé jiskry velký oheň bývá, tak i~zde nepatrná příčina měla veliké následky.

Jak již řečeno, cvičila jednota od roku 1896 v~sále hostince Fr.\ Švidrnocha, který jej pronajal br.\ Rudolfu Hillovi. Br.\ Hill byl člověk přičinlivý, energický, ale i~ctižádostivý, který nesnesl odchylného mínění druhého, ani křivdy skutečné neb i~domnělé. Proto vznikly často mezi ním a~bratry cvičícími spory o~sál, který Hill potřeboval k~jiným zábavám hostinským neb i~k~sušení prádla. Po tu dobu bývalo cvičení přerušováno a~cvičební nářadí odklizeno na půdu. Z~té příčiny se často ozývala touha po vystavení vlastní tělocvičny. Rozhodnutí urychlil br.\ Jan Uhlíř, který --- jak již zmíněno --- byl nejen veliký hudebník, nýbrž i~spiritista, který nás, totiž členy výboru (až na Ed.\ Lazara) a~některé jiné osoby, celkem asi 17, v~tajných schůzkách (seancích) s~naukami spiritistickými seznamoval. V~takových schůzkách, jež řídil medium Uhlíř, byli jsme inteligencemi, to jest duchy zemřelých velikánů českých (hlavně Havlíčka) povzbuzeni k~tomu, abychom sokolovnu brzy vystavěli.

Tak se stalo, že ve schůzi výboru jednoty sokolské, konané dne 9.~února t.r., bylo většinou hlasů usneseno sokolovnu postaviti. Do přípravného stavebního odboru byli zvoleni bratři: Ant.\ Besta, Rud.\ Hill, Adolf Bárta, Vl.\ Košťřica, Ig.\ Blažej.

Ve schůzi 23.~března bylo ujednáno, že se má postaviti sokolovna na zahradě bra Adolfa Bárty, který se uvolil místo ku stavbě darovati, cihlu a~kamení opatřiti, jakož i~financování stavebního nákladu si vzíti na starost. Ant.\ Besta slíbil darovati písek, ostatní členové stav.\ odboru měli postarati jiné stavební potřeby. Stavební plán měl zhotoviti Josef Urbis, stavitel z~Mokrých Lazec.

Avšak již po této schůzi měl Hill o~stavbě jiné mínění. Poněvadž mu majitel hostince Fr.\ Švidrnoch nájem vypověděl a~on se měl koncem roku z~hostince vystěhovati, usiloval u~členů výboru o~to, aby si směl bezprostředně k~budoucí sokolovně přistavěti byt se smíšeným obchodem, v~němž by měl také prodej lihových nápojů.

Výbor však většinou členů byl proti tomu, chtěje míti sokolovnu volnou, na nikom nezávislou, a~rozhodl se postaviti sokolovnu, jež by obsahovala cvičební sál se stálým jevištěm a~dvěma menšími místnostmi pro sborovnu a~kuskovnu. Tím rozhodnutím pohněval si Hilla, který počal na bratry Ant.\ Bestu, Josefa Viltsa a~Josefa Tichého působiti, aby se vyslovili proti stavbě sokolovny, která prý není nutnou, a~provedení stavby zmařili. Jeho úsilí se dařilo.

Kdyžto ve schůzi 23.~března všichni tito bratři se vyslovili pro stavbu, jak v~zápise o~této schůzi jich vlastnoručními podpisy stvrzeno jest, již ve schůzi výboru dne 5.~května, kdy se mělo o~položení základního kamene jednati, scházel již starosta Vilts a~31.~května Josef Tichý. V~jednání byla zřejmá neshoda. Ke schůzi dne 2.~června se žádný z~nich již nedostavil. Přítomni byli: Adolf Bárta, Josef Bárta, Ignát Blažej, Vladimír Košťřica, Fr.\ Vl.\ Sigmund, Adolf Sokol. Tito za přítomnosti stavitele br.\ Jana Sedlaříka z~Lazců schválili jim zhotovený plán a~rozpočet na stavbu sokolovny, znějící na 12.000~K a~stanovili ku položení základního kamene příští neděli 9.~června 1901.

Jednatel br.\ Sokol předložil návrh pamětní listiny, která se potom vložila do základního kamene. Stanoven byl nový stavební odbor, sestávající z~bratří: Adolfa Bárty, Ignáta Blažeje, Ignáta Bárty (otce Josefa a~Adolfa Bárty) a~Adolfa Sokola. Na návrh Josefa Bárty usneseno stavbu prováděti ve vlastní režii. Stavitel br.\ Sedlařík prohlásil, že za plány a~rozpočet stavební ničeho nežádá a~také přejímá zcela bezplatně dozor na stavbu; přeje nám, aby stavba zdárně pokračovala a~co nejdříve svému účelu odovzdána byla.

\phantomsection
\subsection{Slavnost položení základního kamene}

Slavnost položení základního kamene, konaná 9.~června 1901 za účasti asi 900~lidí, z~nich asi 200~členů v~kroji, měla velmi pěkný průběh. Okrsek severní župy mor.\ slezské konal sem pochodové cvičení, jehož se zúčastnily jednoty: Moravská a~Polská Ostrava, Opava, Přívoz, Vítkovice, Paskov, Kunčičky, Michalkovice, Orlová. Našich bratří bylo 17~v~kroji a~10~v~občanském obleku. --- Bratři Vilts, Tichý, Hill, Besta nepřišli. --- Prostná cvičení společná pro omdlenost členů se nekonala. Cvičila pouze závodní družstva z~Opavy a~Mor.\ Ostravy hrazdu a~koně; Kromě toho ostravské družstvo cvičilo kužely a~vzorná prostná II.~sletu.

Slavnostní řeč o~významu Sokola a~důležitosti vlastní tělocvičny pronesl br.\ Dr Jaroslav Pluhař z~Bohumína.

Představenstvo obce, hasičský sbor, ani Čtenářská beseda zastoupeny nebyly. Večer konána zábava v~hostinci Františka Klosa.

Hmotný výsledek byl pěkný:
\begin{itemize}
\item Na vstupném se vybralo 1000~K.
\item Vydání bylo 600~K.
\item Čistého zbylo \textbf{400~K}.
\end{itemize}

Po slavnosti nastala tuhá práce s~prováděním stavby. 11.~června rozděleny funkce:
\begin{itemize}
\item Předsedou stavebního odboru zvolen br.\ Ig.\ Blažej, jemuž přidělen vrchní dozor a~zodpovědnost za všechny předpisy při stavbě, zároveň zplnomocněn pro objednávání a~zadávání prací.
\item Br.\ Adolf Bárta přejal obstarávání peněz a~výplatu veškerých účtů.
\item Br.\ Ignác Bárta převzal denní dozor při práci a~obstarání nahodilých potřeb.
\item Br.\ Adolf Sokol slíbil vésti písemnosti při stavbě, zaznamenávati dělníkům pracovní dny a~hodiny pro výplatu.
\end{itemize}

\phantomsection
\subsection{Stavební materiál}

Kamení vzato od Mikuláše a~Valentina Martínka ze Vřesiny, písek --- 25~fůr --- od Fr.\ Diky z~Poruby a~něco z~Pustkovce, což obstaral a~platil br.\ Ambrož Zdražila. Několik fůr písku daroval také hostinský Fr.\ Klos.

Cihla byla od tří dodavatelů. Něco uděláno a~upáleno na pozemku br.\ Adolfa Bárty, ostatek zakoupeno od firmy Jedlek a~Halourka v~Zábřehu a~rolníka Ferd.\ Kudely ve Svinově. Vápno ze Studénky (vagon) a~ostatek z~Vítkovic od fy Poznanski.

Zednickou práci provedl Fr.\ Kropka, zednický mistr ze Studénky, tesařskou Bartoš ze Studénky, fasádnickou Vincourek z~Mar.\ Hor, pokryvačskou Klíma z~Plesné, klempířskou Dvořák z~Hulvák, štukatérskou L.\ Jirotka z~Mor.\ Ostravy, stolarskou práci firma Ig.\ Blažej v~Porubě, sklenářskou Butora, zámečnickou Zapletal z~Mor.\ Ostravy. Kamna koupena od fy Sýkora v~Praze.

Povozy byly zdarma. Nejvíce povozů bylo od br.\ Ad.\ Bárty. Z~Poruby přispěli povozy: Ambrož Zdražila, Fr.\ Švidrnoch, Klement Calábka, Josef Klos, Adolf Švidrnoch, Josef Besta. Ze Svinova: Adolf Graca, Ed.\ Malík, Jos.\ Urbánek, Fr.\ Hrabovský, Ant.\ Kudela, Ant.\ Rokyta, Alois Silber.

Pozemek pro staveniště $S_a$~145~m² daroval br.\ Adolf Bárta. Plán pro odloučení pozemku zhotovil inženýr V.\ Macourek z~Mor.\ Ostravy, smlouvu trhovou mezi A.~Bártou a~jednotou vyhotovil JUDr Rudolf Hess, advokát v~Klimkovicích.

Celkový náklad na stavbu sokolovny, i~s~vnitřním zařízením činil okrouhle \textbf{10.000~K}, ačkoli rozpočet stavitele Sedlaříka obnášel 12.000~K. Z~toho patrno, že výbor počínal se velmi obezřetně a~opatrně.

\textbf{Stavba započata 7.~června a~ukončena byla počátkem října 1901.}

\phantomsection
\subsection{Otevření sokolovny v~neděli 6.~října 1901}

Počasí bylo nepříznivé, chladno a~veliký vítr, takže se slavnost nemohla konati na zahradě u~sokolovny, jak bylo zamýšleno, nýbrž vše se odbývalo v~sokolovně, která nestačila pojmout účastníky.

Našich bratrů bylo v~kroji 21. Jiné jednoty vyslaly: Přívoz 11, Příbor 7, Opava 5, Vítkovice 5, Dobrá 2, Vsetín 1, Hustopeče 1, Zábřeh 10, Mor.\ Ostrava 1~člena. Mor.\ slezskou župu zastupoval br.\ M.\ Urbánek ze Vřesiny, župu Palackého br.\ Berkopp. Hudba byla z~Frenštátu.

Radosť naše nad dokonáním díla s~takovými překážkami provedeného byla veliká.

\phantomsection
\subsection{Cvičení v~roce 1901}

Cvičení konalo se nepřetržitě po celý rok po 3~dny v~týdnu. Členstvo v~úterý a~ve čtvrtek od 8--9, v~neděli od 5--6. Dorost ve čtvrtek od 7--8, v~neděli od 4--5. Žáci od 1--2~odp.

Do 1.~června cvičeno v~sálu R.\ Hilla. Byvše odtud vypuzeni, cvičili borci ve stodole br.\ Adolfa Bárty až do žní, od 10./7.--18./10.\ v~hostinci Fr.\ Klosa; 16.~října počli cvičit v~sokolovně. Cvičilo se 123~hodinách úhrnem 1100~bratrů, průměrně 9~bratrů, to jest 75\%. Členů dorostu cvičilo ze 14~průměrně 8. V~tom roku zakoupena americká hrazda, kruhy a~šplhadla od fy Vindys v~Praze.

Na venek ukázala se jednota 12~krát:
\begin{itemize}
\item 8.~dubna při zábavě ve prospěch Slezské Ústř.\ (hospor.\ školy v~Kateřinkách) cvičilo 12~bratří prostná, 8~členů bradla.
\item 19.~května konalo 15~bratrů v~kroji a~7~v~obč.\ obleku pochod do V.\ Polomě, kde cvičilo 9~bratří prostná a~skupiny.
\item 16.~června pochodové cvičení do Mor.\ Ostravy.
\item IV.~sletu všesokolského v~Praze zúčastnilo se 9~bratrů, z~nichž 6~cvičilo.
\item 7.~července cvičilo v~Pustkovci při Sokolském večeru 9~bratrů prostná, bradla a~skupiny.
\item 14.~července vyslána deputace k~veř.\ cvičení opavského Sokola v~Přerově.
\item 26.~července konalo 14~členů pochodové cvičení do Zábřehu.
\item V~září zúčastnilo se 10~bratrů v~kroji nár.\ slavnosti v~Kateřinkách.
\item V~září jela deputace k~pohřbu prof.\ Ant.\ Vaňka v~Opavě, jenž byl vydatným národním pracovníkem.
\item V~září vyslána 5členná deputace ku pohřbu br.\ Jos.\ Michálka v~Přívoze.
\end{itemize}

\phantomsection
\subsection{Vzdělávací činnost}

Jak už bylo v~roku 1900 uvedeno, zřízen byl v~jednotě hudebně-pěvecký kroužek, který za řízení br.\ Uhlíře provedl 9~koncertů v~místě i~okolí. Před koncerty proslovy. Koncem roku pořádána v~sokolovně Mikulášská zábava a~hráno drama \emph{Jardina roba}. V~únoru přír.\ hráno ještě v~hostinci Fr.\ Švidrnocha divadlo: \emph{Karel Havlíček Borovský}; před ním proslov br.\ Jana Uhlíře, který hrál tit.\ úlohu. ---

Odebírány časopisy: \emph{Sokol}, \emph{Věstník}, \emph{Prapor}, \emph{Illustr.\ listy}. --- Výpůjček ze sokolské knihovny, čítající na 200~svazků, bylo 450. Ochranitelkou byla nám župa Č.O.\ Sokolské jednota v~Něm.\ Brodě, která však pro nás velmi málo vykonala.

\phantomsection
\subsection{Stav jmění}

\noindent Příjem obnášel se zůstatkem z~min.\ roku \hfill 2214.86~K\\
Vydání \hfill 1568.57~K\\
Přebytek \hfill \textbf{646.29~K}

\bigskip

\noindent\textbf{Soupis majetku:}\\[0.3cm]
\begin{tabular}{lr}
Cena budovy & 9900.---~K \\
Cena pozemku & 906.---~K \\
Cena vnitřního zařízení & 866.46~K \\
Cena tělocvičného nářadí & 442.20~K \\
Cena jeviště s~lampami & 390.---~K \\
\hline
\textbf{Majetek jednoty} & \textbf{12\,504.66~K} \\
\end{tabular}

\bigskip

\noindent Dluh povstalý stavbou sokolovny \textbf{10.000.---~K}

\bigskip

\noindent\textbf{Stav členstva koncem r.~1901:}
\begin{itemize}
\item Zakládajících 12
\item Činných 12
\item Přispívajících 20
\end{itemize}
Úhrnem \textbf{44}.

\clearpage

%%%%%%%%%%%%%%%%%%%%%%%%%%%%%%%%%%%%%%%%%%%%%%%%%%%%%%%%%%%%%%%%%%%%%%%%%%%%%%%
% ROK 1902
%%%%%%%%%%%%%%%%%%%%%%%%%%%%%%%%%%%%%%%%%%%%%%%%%%%%%%%%%%%%%%%%%%%%%%%%%%%%%%%

\phantomsection
\section{Rok 1902}

\begin{description}
\item[Starostou:] Adolf Bárta.
\item[Jeho náměstkem:] Ignác Blažej.
\item[Jednatelem:] Adolf Sokol.
\item[Náčelníkem:] Vladimír Košťřica.
\item[Pokladníkem:] Josef Bárta.
\item[Ostatní členové výboru:] JUDr.\ Jaroslav Pluhař, advokát v~Bohumíně; Lev Červín, učitel v~Krásném Poli; J.\ Vl.\ Sigmund; František Blažej ml.; Ignác Bárta; Ambrož Zdražila.
\item[Náhradníky:] Jan Dolišar, Fr.\ Blažej st., Aug.\ Lazar, Ludvík Vašíček.
\end{description}

V~tomto roku se pilně pracovalo ve všech směrech, ačkoli překážky nás stíhaly při každém kroku. Bývalí vlivní členové, již s~námi při založení jednoty pracovali a~jako členové obecního výboru měli na správu obce i~na občany veliký vliv, nejen že z~jednoty vystoupili, nýbrž jak mohli snahy její mařili a~důvěru k~ní podkopávali.

Vystoupili: Rudolf Hill, Antonín Besta, rolník a~starosta obce, Josef Vilts, rolník, Jan Malík, Jan Sladký, Antonín Balnar, vesměs rolníci. Josef Tichý, rolník, sice neoznámil vystoupení, ale veškeré činnosti se vzdal a~do sokolovny již nevkročil. Jejich nepřátelství jsme cítili při všem. Sokolovna stála, ale návštěva při divadlech a~jiných podnicích v~ní velmi nepatrná. Kdyby nebyli přicházeli hosté z~okolí, hlavně ze Svinova, byla by každá produkce končila deficitem.

Při zábavách nám vadilo, že jsme nemohli hostům podat žádného občerstvení. Posílání pro nápoje do obchodů neb hostinců bylo nevýhodné, ba i~se škodou spojené. Proto na radu a~výzvu našich přátel, zvláště sokolních obcí, podal výbor jednoty žádost na okresní hejtmanství v~Bílovci o~povolení hostinské koncese ku prodeji nápojů (mimo kořalky) a~pokrmů. Leč žádost ta byla zamítnuta hlavně působením představenstva obce Poruby, v~němž zasedali z~předu uvedení pánové, naši bývalí členové. Proti zamítnutí podán rekurs prostřednictvím br.\ Dra Jar.\ Pluhaře, avšak bez výsledku. Bylo třeba ještě mnohaletého přemýšlení a~úsilí, než se cíle dosáhlo.

Výbor vyvíjel velikou činnost. Konal v~roku 14~schůzí, v~nichž jednáno o~tom, jak jednotě získati stálých a~bezpečných příjmů a~povznésti ji v~ohledu mravním. Vydán byl zvláštní řád pro přijímání členů, aby se mezi ně nedostaly živly rušivé.

O~udržení pořádku a~čistoty v~sokolovně se horlivě staral nejstarší člen jednoty staříček Ignác Bárta, jemuž valná hromada zato vyslovila zvláštní díky. Byl kladen za vzor mladým členům, kteří se práci vyhýbali a~pro snahy starších nejevili řádného porozumění.

\textbf{Toho roku zemřel první přítel naší jednoty Br.\ Jan Dedek a~byl pochován ve svém rodišti. Čest jeho památce!}

\phantomsection
\subsection{Cvičení}

Cvičení konáno po celý rok až na srpen, kdy náčelník byl na vojenském cvičení. Cvičení 2--3~krát v~týdnu, celkem 120~hodin za průměrné účasti 8.6~cvičenců. Nejpilnějšími cvičenci byli bři Krejčí a~Doležel. Také dorost cvičil, ačkoli nepravidelně. --- Mezi cvičícími členy bylo 8~dělníků, 2~učitelé, 1~živnostník a~1~úředník.

Na venek se ukázala jednota 6~krát a~sice:
\begin{enumerate}
\item V~červenci při okrskovém cvičení ve Svinově 12~br.\ v~kroji, 16~v~obč.
\item V~srpnu při pozdvižení praporu Sokola v~Mar.\ Horách 10~br.\ v~kroji.
\item Při sletu sokolském v~Mor.\ Ostravě 16~v~kroji, 12~v~obč.\ obleku (2~dny).
\item V~září při národní slavnosti ve Svinově 10~v~kroji, 9~v~obč.
\item V~listopadu při otevření sokolovny v~Přívoze 6~v~kroji, 5~v~obč.
\end{enumerate}

Okrskové cvičení ve Svinově, pořádané na louce za mlýnem bratrů Strnadů měli jsme na starosti sami. Stálo to velikou námahu; v~sobotu před slavností bylo deštivé počasí, které práci znesnadňovalo, proto se pracovalo druhý den již do $\frac{1}{4}$~hodin zrána na úpravě cvičiště. Pro nářadí do Svinova a~M.\ Ostravy jeli bratři Košťřica a~Krejčí, do Zábřeha a~Vítkovic br.\ Doležel. --- Cvičení samo za krásné pohody a~velké účasti se vydařilo znamenitě.

\phantomsection
\subsection{Vzdělávací a~zábavní činnost}

Předsedou zábavního odboru byl Josef Bárta, režiserem Ad.\ Sokol.
\begin{itemize}
\item 8.~května sehrána veselohra \enquote{Brouček z~Lokaye} (1.~kus v~sokolovně) za hojné účasti, hlavně z~okolí.
\item 19.~května sehrána jednoaktovka \enquote{Věno} s~proslovem, básněmi, hudbou a~sborovým zpěvem.
\item 15.~srpna 2~jednoaktovky s~volným vstupem.
\item 15.~října sehrána Stolbova \enquote{Závěť} s~úspěchem plným.
\item 8.~prosince jednoaktovka \enquote{Pletichy}, smíšené sbory a~hudební kvarteto.
\end{itemize}

Časopisy čteny: \emph{Sokol}, \emph{Věstník sokolský}, \emph{Prapor}, \emph{Zdraví}, \emph{Ostravský denník}, \emph{Illustrovaný svět}, \emph{Havlíček}.

Ku členstvu promluveno 9~krát. Nejvýznamnější byly večery na oslavu Mistra Jana Husa a~na památku Bitvy na Bílé Hoře. --- Účast na podnicích sokolských slušná, ačkoli byl přístup k~sokolovně ztížen tím, že nová okresní cesta vesnicí se teprve počala stavěti.

Toho roku se přistavěla k~sokolovně kuželna pro zábavu členů i~hostů, které bylo v~prvních letech hojně používáno.

\noindent Počet členů koncem roku:
\begin{itemize}
\item zakládajících 11
\item činných 12
\item přispívajících 26
\end{itemize}
Úhrnem \textbf{49}.

\clearpage

%%%%%%%%%%%%%%%%%%%%%%%%%%%%%%%%%%%%%%%%%%%%%%%%%%%%%%%%%%%%%%%%%%%%%%%%%%%%%%%
% ROK 1903
%%%%%%%%%%%%%%%%%%%%%%%%%%%%%%%%%%%%%%%%%%%%%%%%%%%%%%%%%%%%%%%%%%%%%%%%%%%%%%%

\phantomsection
\section{Rok 1903}

Zvoleni byli:

\begin{description}
\item[Starostou:] Adolf Bárta.
\item[Jeho nám.:] Ig.\ Blažej.
\item[Náčelníkem:] Vl.\ Košťřica.
\item[Jeho nám.:] Ant.\ Záviš, tajemník obce Polanky.
\item[Jednatelem:] Adolf Sokol.
\item[Ostatní členové výboru:] Josef Bárta, Frant.\ Krejčí (stolař), Ambrož Zdražila, JUDr.\ Zar.\ Pluhař, Cyril Vejtek (zástupce pojišťovny), Emil Valder z~Pustkovce, Ferd.\ Fryčer (tajemník obce Svinova).
\item[Náhradníky:] Ig.\ Bárta, Vincenc Martiník (naduč.\ ve Vřesině), Frant.\ Valder, Robert Valder z~Pustkovce.
\end{description}

Valná hromada schválila změnu stanov v~tom smyslu, aby v~případě rozpuštění spolku připadlo jeho jmění instituci sokolské.

Tento rok jako předešlý byl pln houževnaté práce sokolské, již nám všemožně ztěžovali naši nepřátelé --- bývalí bratři. Výbor konal 11~schůzí. Projednáno bylo 75~došlých dopisů a~odesláno 95~dopisů.

\phantomsection
\subsection{Cvičení}

Sbor cvičitelský sestával z~bratrů: Vl.\ Košťřice, A.\ Záviše, Frant.\ Richtra a~Fr.\ Krejčího. Pomocníkem byl Ludvík Dedoch.

Dny cvičební pro členstvo: středa a~sobota od 8--9~večer (119~cvič.\ hodin), pro dorost: neděle od 1--2~hod.

\phantomsection
\subsection{Účast na různých podnicích}
\begin{itemize}
\item 18.~dubna: Noční výlet do Zábřehu --- 9~bratrů.
\item 14.~června: Při okrskovém cvičení v~Zábřehu cvičilo 9~br.
\item 21.~června: Při slavnosti sokolské v~Přívoze i~členové v~kroji.
\item 28.~června: Slavnosti 10letého trvání hasičského sboru v~Porubě 13~bratrů v~občanském obleku.
\item 29.~června: Společně se Zábřežskou jednotou cvičilo ve Střebovicích 13~bratrů v~kroji.
\item 15.~srpna: Veřejné cvičení v~Porubě. Vypomáhal Přívoz.
\item 16.~srpna: Při župním cvičení ve Frýdku 6~bratrů.
\item 23.~srpna: Při národní slavnosti ve Svinově 7~br.\ v~kroji.
\end{itemize}

\phantomsection
\subsection{Vzdělávací činnost}

Ku členstvu promluveno 8~krát, k~dorostu jednou. O~jubileu básníka Jaroslava Vrchlického promluvil Josef Bárta, který o~Vrchlickém přednášel také ve Střebovicích.

Sehrány byly tyto divadelní hry:
\begin{enumerate}
\item \emph{Nové hnízdo}, veselohra od Lokaye.
\item \emph{Přítel}, veselohra od Vikové-Kunětické.
\item \emph{Šumařova dcera}, drama od Rutha.
\item \emph{Zmatek nad zmatek}, fraška od Kotzebue.
\item \emph{Přišla do rozumu}, vesel.
\item \emph{Praprababiččin kalendář}, vesel.\ od Jonáše.
\item \emph{Služebník svého pána}, drama od Jeřábka. Toto půvabné drama hráno dvakrát.
\end{enumerate}

Také byla přednáška o~M.\ Janu Husovi.

\phantomsection
\subsection{Zpráva pokladní}

\noindent Přijato \dotfill 856~K 68h\\
Vydáno \dotfill 676~K 97h\\
Zbytek \dotfill \textbf{179~K 71h}

\bigskip

\noindent Členů koncem roku:
\begin{itemize}
\item zakládajících 10
\item činných 7
\item přispívajících 18
\end{itemize}
Úhrnem \textbf{35}.

\clearpage

%%%%%%%%%%%%%%%%%%%%%%%%%%%%%%%%%%%%%%%%%%%%%%%%%%%%%%%%%%%%%%%%%%%%%%%%%%%%%%%
% ROK 1904 - 10 LET JEDNOTY
%%%%%%%%%%%%%%%%%%%%%%%%%%%%%%%%%%%%%%%%%%%%%%%%%%%%%%%%%%%%%%%%%%%%%%%%%%%%%%%

\phantomsection
\section{Rok 1904}

O~valné hromadě konané v~sobotu 16.~ledna za účasti 12~členů byli zvoleni:

\begin{description}
\item[Starostou:] Adolf Bárta.
\item[Jeho nám.:] Ignát Blažej.
\item[Náčelníkem:] Vlad.\ Košťřica.
\item[Jednatelem:] Josef Bárta.
\item[Členy výboru:] Ignát Bárta, Adolf Sokol (pokladníkem), Dr.\ Zar.\ Pluhař, Frant.\ Krejčí, Vincenc Martiník, Ambrož Zdražila, Ludvík Dedoch (knihovníkem a~hospodářem).
\item[Náhradníky:] Fr.\ Kudela, Val.\ Gebauer, Ferd.\ Fryčer, Tomáš Gelnar (učitel).
\end{description}

Předsedou zábavního odboru zvolen Josef Bárta.

\phantomsection
\subsection{Jubilejní rok}

Tento rok byl rokem jubilejním, neboť uplynulo od založení jednoty 10~let. Deset let poctivé úmorné práce, která se nesetkala s~takovým výsledkem, jak by byla zasluhovala, která však přece zanechala v~obci naší viditelné stopy vzdělání a~pokroku, jakými se nemohla chlubit žádná z~obcí okolních. Kdyby nebylo bývalo té neblahé roztržky mezi členstvem, byly by výsledky ještě lepší.

Jednota utrpěla velikou ztrátu odchodem náčelníka bratra Vlad.\ Košťřice, který od 1.~září nastoupil učitelské místo ve Svinově. Svou milou povahou a~vytrvalou činností jako náčelník i~jako divadelní ochotník získal si přízně všech. Jeho nástupce učitel Eduard Jetens šel z~počátku v~jeho šlépějích, později však ochaboval, až asi po 4~letech činnosti se vzdal a~z~jednoty vystoupil.

\phantomsection
\subsection{Činnost cvičební}

Cvičitelský sbor se skládal z~bratrů: Košťřice, Krejčího, Záviše a~Richtra. Když dva poslední odešli, nastoupili Dedoch a~Kojera.

V~lednu a~únoru cvičeno v~neděli od 4--5~hod., v~březnu ve středu a~v~sobotu od $\frac{1}{2}$9--$\frac{1}{2}$10~večer. Pravidelně cvičilo 9členné družstvo ve 118~hodinách.

Na veřejnost vystoupeno 6~krát:
\begin{itemize}
\item 5.~června v~Ostravě při slavnosti položení základního kamene ku stavbě sokolovny --- 9~bratří v~kroji, 4~v~obč.\ obleku.
\item 12.~června v~Porubě při slavnosti 10letého trvání jednoty 15~v~kroji. Při prostných cvičili s~námi bratři z~Mar.\ Hor, Přívozu, Zábřehu, Mor.\ Ostravy a~Opavy.
\item 19.~června vyslána 3členná deputace k~veřejnému cvičení Opavského Sokola do Háje a~2členná deputace do Mar.\ Hor.
\item 26.~června zúčastnilo se okrskového cvičení v~Přívoze 6~br.\ v~kroji, 2~v~obč., z~nichž cvičilo 5~br.\ prostná a~skupiny.
\item 3.~července byli na župním sletu v~Pol.\ Ostravě 14~br.
\item 24.~července při okrskovém výletu ve Svinově 15~br.\ v~kroji, cvičilo 7.
\end{itemize}

\phantomsection
\subsection{Činnost zábavní a~vzdělávací}
\begin{itemize}
\item 30.~ledna konán ples. V~lednu hrány \enquote{České Amazonky}, ves.\ od J.K.\ Tyla.
\item 21.~února členská schůze s~přednáškou br.\ Kojery \enquote{Čeho dbáti při cvičení sokolském.}
\item 28.~února přednášel br.\ učitel Emil Jelen ze Vřesiny o~Fr.\ Palackém.
\item 25.~března večer Komenského. Po přednášce E.\ Jelena sehrána divadelní hra \enquote{Sebeurčení} od Ant.\ Rudka.
\item 23.~května sehrány 2~aktovky.
\item 12.~června při slavnosti 10letého trvání promluvil br.\ Ad.\ Sokol o~poměrech naší jednoty a~Dr.\ Jar.\ Pluhař o~povinnostech Sokola a~Čecha. Po slavnosti sehrána večer v~sokolovně Stroupežnického veselohra \enquote{Ochrana Napoleona}. Po divadle přednesl učitel Jaroslav Gabarna z~Mokrých Lhot některá koncertní čísla na housle, přednesl několik znamenitých čísel z~oboru humoristiky. --- Byl to překrásný večer, účast veliká.
\item 25.~srpna zábavní večer: veselohra Štechova \enquote{Maloměstské tradice}, pak ženské sbory s~průvodem klavíru (Drošákovy Dvojzpěvy).
\item 31.~srpna večírek na rozloučenou s~br.\ náčelníkem Vl.\ Košťřicou s~recitacemi, hudbou a~zpěvem.
\item 15.~října hrána veselohra \enquote{Jedenácté přikázání} od Šamberka.
\end{itemize}

Staré jeviště prodáno sokolské jednotě v~Martinově za 100~K. Nové nám vymaloval opět malíř Ant.\ Svoboda z~Příbora, který toho roku maloval zdejší kostel. Projekt jeviště vypracovali Blažej a~Sokol.

Mezi stromky, vysazenými minulého roku na zahradě u~sokolovny, byly upraveny cestičky nákladem 41~K. Z~pokladny za to 19~K, ostatek uhradili členové výboru. Osazené zahradě jsme říkali \enquote{Riegrův park}, poněvadž roku 1903, kdy byla zařízena, zemřel vůdce národa Dr.\ F.\ Rieger.

Toho roku byl také zakoupen od firmy Vojtěch Najš v~Brně obehraný klavír za 330~K.

Počet členů vzrostl hlavně přistoupením dělníků z~továrny Ig.\ Blažeje, založené r.~1903; koncem roku bylo:
\begin{itemize}
\item členů zakládajících 12
\item činných 12
\item přispívajících 36
\end{itemize}
Úhrnem \textbf{60}.

\noindent Příjem byl \hfill 2002.79~K\\
Vydání \hfill 1964.29~K

\clearpage

%%%%%%%%%%%%%%%%%%%%%%%%%%%%%%%%%%%%%%%%%%%%%%%%%%%%%%%%%%%%%%%%%%%%%%%%%%%%%%%
% ROK 1905
%%%%%%%%%%%%%%%%%%%%%%%%%%%%%%%%%%%%%%%%%%%%%%%%%%%%%%%%%%%%%%%%%%%%%%%%%%%%%%%

\phantomsection
\section{Rok 1905}

Valná hromada konala se 29.~ledna za účasti 15~členů. Zvoleni byli:

\begin{description}
\item[Starostou:] Adolf Bárta.
\item[Jeho nám.:] Ig.\ Blažej.
\item[Náčelníkem:] Eduard Jetens.
\item[Jednatelem:] Josef Bárta.
\item[Pokladníkem:] Adolf Sokol.
\item[Ostatní čl.\ výboru:] František Krejčí, Alois Petras (soustružník v~fcy Blažej), Vladimír Košťřica, Ignác Bárta, Václav Zoubek, Josef Čínek.
\item[Náhradníky:] Ambrož Zdražila, Ludvík Dedoch, Ant.\ Bárta, Jos.\ Jirák.
\item[Novinářem:] Ant.\ Studenský.
\end{description}

\phantomsection
\subsection{Cvičení}

Hodin cvičebních bylo 92, průměrně cvičilo 6~bratrů 2--3~týdně. Dorost cvičil br.\ Josef Jirák. Ze cvičenců odešli 3~k~vojsku, takže koncem roku zůstalo jich 10, tedy pořád málo.

\begin{itemize}
\item 4.~června pořádalo se v~Porubě okrskové cvičení. Příjem byl 663.64~K, vydání 591.46, čistého 72.18~K.
\item 25.~června zúčastnila se jednota cvičení v~Mor.\ Horách.
\item V~červenci slavnosti \enquote{Havlíčka} ve Střebovicích.
\end{itemize}

\phantomsection
\subsection{Vzdělání a~zábava}

\textbf{Přednášky:}
\begin{itemize}
\item 30.7.\ Br.\ Adolf Sokol: O~stavu jednoty a~povinnostech Sokola.
\item 26.11.\ br.\ Ed.\ Jetens: O~bitvě na Bílé hoře.
\item 3.12.\ br.\ Josef Bárta: O~naší zeměkouli.
\end{itemize}

\textbf{Divadel pořádáno 7:}
\begin{itemize}
\item 22.~ledna --- \emph{Lapený Samsoněk}, vesel.\ od Svobody.
\item 24.~dubna --- \emph{Olymp}, veselohra od Rutha.
\item 31.~května --- \emph{Maryša}, drama od Mrštíků.
\item 4.~června --- \emph{Závěť}, drama od Stolby.
\item 15.~října --- \emph{Peníze}, drama od Stolby.
\item 19.~listop.\ --- \emph{Charleyova teta}, veselohra přeložená z~angl.
\item 17.~prosince --- \emph{Otec}, drama od Jiráska.
\item 24.~září --- hudebně pěvecká zábava.
\end{itemize}

Při divadle účinkovaly jako hlavní osoby: Marie Sokolová, industr.\ učitelka, Leopoldina Bártová, choť nadučitele, Emilie Zdražilová, provdaná později za Fr.\ Blažeje, ze Svinova slečny Hurníkovy. Z~mužů: Jan Černý, poštovní úředník a~Jan Čech, nádražní úředník ze Svinova, z~Poruby Ed.\ Jetens, Alois Petras aj.

Časopisy předplaceny: \emph{Sokol}, \emph{Věstník sokolský}, \emph{Prapor} a~\emph{Cvičitelské Listy}. Knih vypůjčeno 116.

\phantomsection
\subsection{Dary a~půjčky}
\begin{itemize}
\item Župa mor.-slezská poslala neúročitelnou půjčku \hfill 200.---~K
\item Česká Obec Sokolská neúročitelnou půjčku \hfill 1500.---~K
\item Jednota Sokol na Vsetíně dar na stavbu sokolovny \hfill 15.---~K
\item Arnold Skorkovský, maj.\ továrny na sukna v~Humpolci \hfill 10.---~K
\end{itemize}

Eduard Hrubeš, správce školy v~Ondřichovicích zaslal 200~knih zábavných, pocházejících z~jeho knihovny neb z~Ochotnického divadelního spolku v~Napajedlích.

\bigskip

\noindent Příjem roku 1905 činil \hfill 2020.50~K\\
Vydání \hfill 1733.41~K\\
Zbytek k~1.1.\ 1906 \hfill \textbf{287.09~K}

\bigskip

\noindent Členů:
\begin{itemize}
\item zakládajících --- 7
\item činných --- 10
\item přispívajících --- 30
\end{itemize}
Úhrnem \textbf{47}.

\clearpage

%%%%%%%%%%%%%%%%%%%%%%%%%%%%%%%%%%%%%%%%%%%%%%%%%%%%%%%%%%%%%%%%%%%%%%%%%%%%%%%
% ROK 1906
%%%%%%%%%%%%%%%%%%%%%%%%%%%%%%%%%%%%%%%%%%%%%%%%%%%%%%%%%%%%%%%%%%%%%%%%%%%%%%%

\phantomsection
\section{Rok 1906}

Valná hromada konána 14.~ledna za přítomnosti 18~členů. Zvoleni byli:

\begin{description}
\item[Starostou:] Adolf Bárta.
\item[Jeho nám.:] Ignác Blažej.
\item[Náčelníkem:] Eduard Jetens.
\item[Jednatelem:] Josef Bárta.
\item[Pokladníkem:] Adolf Sokol.
\item[Ostatní členové:] Josef Jirák, Fr.\ Krejčí, Fr.\ Klos st., Ig.\ Bárta (hospodářem), Karel Malík, Ant.\ Bárta (perníkář).
\item[Náhradníky:] Košťřica, Navrátil, Dedoch, Ant.\ Studenský.
\item[Předsedou Vzděl.\ a~zábavního odboru:] Ig.\ Blažej.
\end{description}

\phantomsection
\subsection{Cvičení}

Hodin cvičebních 112, průměrně cvičilo 6~borců 2--3~týdně.

\begin{itemize}
\item Na okrskovém cvičení v~Přívoze cvičilo 6~členů.
\item Na župním v~Mor.\ Ostravě cvičilo 8~členů prostná.
\item Jednota zúčastnila se veřejného cvičení v~Mar.\ Horách, Staré Bělé a~v~Kunčicích.
\item 2.~srpna konalo se veřejné cvičení v~Porubě za účasti jednoty z~Mar.\ Hor. Společně cvičena prostná. Před členstvem cvičil dorost.
\end{itemize}

Vstupné na výletiště bylo 32~hal.\ i~s~národním kolkem. Večer bylo divadlo v~sokolovně; vstupné: I.~místo 60~h, II.~50h, III.~40h, k~stání 20~h.

Z~uvedeného nepatrného vstupného je viděti, jak bylo jednotě těžko sehnati za rok tolik příjmů, aby stačily na úrokování vypůjčeného kapitálu, na opravy sokolovny, nářadí, na otop, světlo a~jiné potřeby, které činily přes 1000~K ročně. Členové konali všechny práce bezplatně sami, konali sbírky při každé příležitosti, prodávali bloky po 10h, aby jen sokolovnu udrželi. Jak to bylo za stávajících poměrů těžko, nemožno si ani představit.

Sokolovna byla po celý den uzavřena, klíč byl u~některého bratra. Kdo chtěl jíti večer do sokolovny, musel napřed sehnati klíč, sokolovnu otevřít, rozsvítit si lampu a~v~zimě si sám zatopil, ovšem bylo-li po ruce dříví a~uhlí. Měl-li chuť na nějaké občerstvení, musel si je buď s~sebou přinést anebo si pro ně buď zajít neb někoho poslat do hostince. Byly to těžké dny a~pomoci žádné; žádosti o~hostinskou koncesi byly vlivem nepřátelského zastupitelstva obce napořád zamítány.

\phantomsection
\subsection{Vzdělávací a~zábavní činnost}
\begin{itemize}
\item 20.~února přednáška profesora Švába z~Mor.\ Ostravy \enquote{O~Svat.\ Čechovi}, jenž slavil toho roku 60té narozeniny. Jednota zaslala básníkovi blahopřání, na něž došla odpověď (poděkování) vlastnoručně psaná.
\item 29.~července přednášel br.\ Košťřica o~Karlu Havlíčku Borovském.
\item 28.~listopadu br.\ Ed.\ Jetens, O~přímém a~nepřímém působení tělocviku na duši.
\item 6.~července zúčastnila se jednota slávy M.\ Jana Husa, pořádané mezi Svinovem a~Porubou; při hranici promluvil redaktor Policar z~Mor.\ Ostravy. Účast z~obou vesnic veliká.
\end{itemize}

\textbf{Divadel bylo 6:}
\begin{itemize}
\item dubna činohra \emph{Obětovaná}.
\item května drama \emph{Emigrant} od Jiráska.
\item června --- \emph{Pan Měsíček, obchodník}, ves.\ od Stroupežnického. (Měsíčka hrál Arnošt Chamrad z~Hrabyně.)
\item 8.~září \emph{Blázinec v~prvém poschodí}, ves.\ od Šamberka.
\item 28.~října --- fraška \emph{Hrůzostrašná noc} od Šamberka a~\emph{All right} od Tölly.
\end{itemize}

Další činnost byla přerušena spálovou epidemií v~obci se rozšířivší.

\bigskip

\noindent Příjem toho roku \hfill 2831.45~K\\
Vydání \hfill 2525.47~K\\
Hotovost na rok 1907 \hfill \textbf{305.98~K}

\bigskip

Poněvadž neplatící členové byli ze seznamu vypuštěni a~někteří odešli buď k~vojsku nebo jinam, bylo koncem roku 1906:
\begin{itemize}
\item členů zakládajících 7
\item cvičících 14
\item přispívajících 9
\end{itemize}
Úhrnem \textbf{30}.

\clearpage

%%%%%%%%%%%%%%%%%%%%%%%%%%%%%%%%%%%%%%%%%%%%%%%%%%%%%%%%%%%%%%%%%%%%%%%%%%%%%%%
% ROK 1907
%%%%%%%%%%%%%%%%%%%%%%%%%%%%%%%%%%%%%%%%%%%%%%%%%%%%%%%%%%%%%%%%%%%%%%%%%%%%%%%

\phantomsection
\section{Rok 1907}

Valná hromada se konala 20.~ledna za přítomnosti 19~členů. Zvoleni byli:

\begin{description}
\item[Starostou:] Adolf Bárta.
\item[Jeho nám.:] Ig.\ Blažej.
\item[Náčelníkem:] Ed.\ Jetens.
\item[Jednatelem:] Josef Bárta.
\item[Pokladníkem:] Adolf Sokol.
\item[Ostatní čl.:] Vl.\ Košťřica, Rudolf Buron, Josef Jirák, Ambrož Zdražila, Ignác Bárta, Antonín Bárta.
\item[Náhradníky:] Navrátil, Krejčí, F.\ Valder, K.\ Malík.
\item[Prohlížitelé účtů:] Ludvík Dedoch, Leop.\ Martiník.
\end{description}

V~minulém roce konána jedna členská schůze a~10~výborových.

\phantomsection
\subsection{Úmrtí bratra Ignáce Bárty}

Dubna zemřel nejstarší člen naší jednoty br.\ Ignác Bárta ve věku 64~let. Byl vzorem pilného a~obětavého Sokola. Staral se všemožně o~udržování pořádku v~sokolovně, při jejíž stavbě dozíral a~vypomáhal, a~účastnil se radou i~skutkem při každém podniku sokolském. Pohřbu dobrého staříčka, jak mu všichni říkali, účastnili se všichni bratři buď v~kroji neb v~obleku občanském.

\phantomsection
\subsection{Cvičení}

Cvičitelský sbor sestával z~bratrů Ed.\ Jetensa, Rud.\ Buroně, Václava Lazara a~Frant.\ Balnara. Když Balnar odešel k~vojsku, nastoupil Robert Blažej.

V~99~cvičebních hodinách cvičilo průměrně 5.44~cvičenců. Dorost cvičil v~38~hodinách, průměrně 12~hochů.

\begin{itemize}
\item Června bylo místní veřejné cvičení za výpomoci jednot zábřežské a~mar.\ horské, s~nimiž cvičena prostná V.~sletu; pak 1~družstvo metací stůl, druhé koně, třetí bradla. Ženský odbor z~Přívozu a~Mar.\ Hor podal ukázku cvičení sletových s~kuželi.
\item 27.~června zúčastnila se jednota V.~sletu všesokolského v~Praze 11~členy, z~nichž 3~cvičili.
\item 2.~června zastoupena byla jednota při slavnosti v~Opavě.
\item Při okrskovém cvičení v~Zábřehu cvičilo 1~družstvo koně.
\item Rovněž se zúčastnili někteří bratři sokolských slavností v~Mar.\ Horách, Místku, Hradci--Podolí, Klimkovicích a~v~Přívoze.
\end{itemize}

K~veřejnému cvičení byly zakoupeny 2~prapory.

\phantomsection
\subsection{Činnost vzdělávací a~zábavní}
\begin{itemize}
\item 3.~března div.\ představení: \emph{Její pastorkyňa}, drama od J.\ Preissové.
\item 7.~dubna --- \emph{Ravuggiolo}, ves.\ od Šamberka.
\item 12.~května --- \emph{Krejčí a~švec}, ves.\ od Stolby.
\item 13.~října --- \emph{Papageno}, vesel.\ od Rutha.
\item 24.~listop.\ --- 2~jednoaktovky.
\item 15.~prosince --- \emph{Palackého třída}, vesel.\ od Šamberka.
\item 31.~prosince Sylvestrovská zábava: \emph{Vzorný vlastenec}, vesel.\ a~výstupy.
\end{itemize}

Návštěva divadel byla slušná, dík dobrým výkonům všech účinkujících.

\phantomsection
\subsection{Opravy a~změny v~sokolovně}

Značným nákladem byla opravena podlaha sálu, kuželna, sklep a~místnost pod jevištěm, kde se dávala houba.

Obě menší místnosti vedle sálu (sborovna a~knihovna) byly spojeny tím, že se dělící zeď odstranila; záchod vedle knihovny zrušen, aby se docílilo většího místa pro schůze.

Zakoupeny činky, stálky a~opraveny žíněnky.

\phantomsection
\subsection{Dary}

Ochranitelka naše, jedn.\ Sokol v~Německém Brodě poslala nám darem 30~K.

Stavitel a~majitel továrny na kamna, Viktor Neusser v~Klokočově, daroval nám pěkná kachlová kamna.

\bigskip

\noindent Počet členů koncem roku 1907:
\begin{itemize}
\item zakládajících --- 7
\item cvičících --- 13
\item přispívajících --- 26
\end{itemize}
Úhrnem \textbf{46}.

\clearpage

%%%%%%%%%%%%%%%%%%%%%%%%%%%%%%%%%%%%%%%%%%%%%%%%%%%%%%%%%%%%%%%%%%%%%%%%%%%%%%%
% ROK 1908
%%%%%%%%%%%%%%%%%%%%%%%%%%%%%%%%%%%%%%%%%%%%%%%%%%%%%%%%%%%%%%%%%%%%%%%%%%%%%%%

\phantomsection
\section{Rok 1908}

Valná hromada konána v~neděli 19.~ledna za účasti 28~členů. Zvoleni byli:

\begin{description}
\item[Starostou:] Adolf Bárta.
\item[Jeho nám.:] Ig.\ Blažej.
\item[Náčelníkem:] Josef Jirák.
\item[Jeho nám.:] Rudolf Buron.
\item[Jednatelem:] Josef Bárta.
\item[Pokladníkem:] Adolf Sokol.
\item[Ostatní:] Vladimír Košťřica (vyslancem do župy a~Č.O.S.), Antonín Bárta, Antonín Besta st., Jakub Halička (obchodník v~Porubě), Ambrož Zdražila.
\item[Náhradníky:] Ed.\ Jetens, Alois Besta, Ludvík Dedoch, Fr.\ Klos.
\item[Prohl.\ účtů:] Josef Vaněk, Fr.\ Novák (účetní).
\end{description}

\phantomsection
\subsection{Cvičení}

Br.\ náčelník Josef Jirák se staral, aby cvičení pozvedl. Poněvadž členů bylo málo, hleděl získat dorost a~žáky. Cvičících členů bylo 13, dorostu 12, žáků 11.

V~zimních měsících v~nevytopeném sálu cvičeno málo, od dubna 6--13~krát měsíčně.

Náčelník podnikl s~bratry vycházku do Kyjovic, kde prohlédli krásnou zámeckou zahradu (dne 19.~dubna).

\begin{itemize}
\item 17.~května zúčastnilo se veřejného cvičení v~Polance, jež pořádala jednota z~Mar.\ Hor 7~br.\ v~kroji, 4~v~obč.
\item 28.~května cvičební a~zábavní večer v~sokolovně. Po cvičení provedeném na jevišti (prostná a~bradla) sehráli mladí členové 2~jednoaktové veselohry \emph{Nechce kouřit} a~\emph{Konec světa}.
\item 8.~června sehrána národní operetta \emph{Lucifer}.
\item 21.~června pořádáno bylo okrskové cvičení ve Svinově, k~němuž všechny přípravy vykonala naše jednota. Cvičilo: 140~cvičenců, 25~žen, 45~dorostenek, 40~žáků. Z~naší jednoty cvičilo 7~bratrů prostná a~skupiny, 2~hrazdu a~2~bradla. --- Úspěch morální i~hmotný byl veliký.
\end{itemize}

\phantomsection
\subsection{Další činnost}
\begin{itemize}
\item 3.~října pořádán večírek na rozloučenou s~odcházejícími bratry Rudolfem Buronem a~Leop.\ Martiníkem, k~nimž promluvil Adolf Sokol.
\item 15.~října pořádána Hedvická zábava s~písněmi a~recitacemi. Bezručovy básně znamenitě recitoval br.\ Gustav Kořený, nadučitel ve Vřesině, který po celou dobu svého působení ve Vřesině jako výtečný herec, zpěvák, hudebník i~recitátor nám neocenitelné služby prokázal. S~ním zároveň účinkoval jako herec i~řečník při našich podnicích mladší učitel vřesinský br.\ Ant.\ Cholský.
\end{itemize}

\phantomsection
\subsection{Jiné podniky}
\begin{itemize}
\item 29.~února 1908 pořádán byl v~sokolovně ples, který nám představenstvo obce (starosta Josef Tichý) teprve na zakročení zemského výboru povolilo.
\item 25.~března sehrána \emph{Palicova dcera}, obraz ze života od Tyla. Před hrou měl Josef Bárta proslov o~spisovateli J.~K.~Tylovi. Tyla hrál br.\ Frant.\ Navrátil; krále Václava Karel Bartek, který byl dobrým hercem a~technickým režisérem.
\item 26.~dubna sehrána byla veselohra \emph{Senec}, čili \emph{Nevinně vinníci} od A.K.
\item 25.~října sehráno \emph{Vykoupení}, drama od Kepkové--Novotné.
\item 15.~listopadu veselohra \emph{Věrný manžel} od Ant.\ Lokaye.
\item 21.~prosince pořádána Sylvestrovská zábava, v~níž připadla hlavní úloha výbornému komiku Ig.\ Navrátilovi, který po mnoho let hrával čelné role vážné i~komické.
\end{itemize}

O~Mistru Janu Husovi přednášel redaktor Otakar Skýpala.

Z~uvedeného vidno, že jsme divadlům věnovali hodně času a~píle, jednak jako prostředku vzdělávacímu a~u~lidu našeho oblíbenému, jednak též z~důvodů finančních, abychom výtěžkem kryli veliká svá vydání.

\phantomsection
\subsection{Další záznamy}

Jeviště jsme propůjčili dvakráte k~divadlu ve Vřesině, pak do hostince Fr.\ Švidrnocha ku slavnosti mateřské školky, která v~létě v~Porubě trvala. Mateřská škola byla zřízena za starostování Antonína Besty, hlavně působením učitelstva a~členů Sokola. Když pak dostali vládu obce do rukou klerikálové, zrušili mateřskou školu 30./12.~1910 a~do uprázdněné místnosti v~Obecním Domě usadili čeledníky! --- Učitelka mateřské školy sl.\ Olga Gregorková osvědčila se nejen jako dobrá učitelka, nýbrž i~jako výborná herečka, která po paní Sokolové převzala repertoár a~vytvořila celou řadu předních ženských úloh; na příklad Máji v~Oblacích, Gardinu robu, Lidušku v~Pasekách atd.

Z~knihovny bylo vypůjčeno 291~knih. Knihovníkem vzorným byl br.\ Josef Vaněk, který knihovnu zrevidoval, nevhodné knihy vyřadil, nový seznam pořídil a~knihy pravidelně půjčoval. Škoda, že ten pilný a~nadaný hoch pak brzy zemřel!

\phantomsection
\subsection{Ochranitelka}

Naše ochranitelka, jednota Sokol v~Německém Brodě slavila toho roku 40leté trvání. Abychom jí k~tomu jubileu blahopřáli a~zároveň ji o~stavu naší jednoty informovali, vyslali jsme jako našeho zástupce do Něm.\ Brodu bratra Vladimíra Košťřicu. Po svém návratu přinesl zprávu, že byl mile přijat, pohoštěn a~že nám jednota N.~Br.\ pošle darem 50~K, což se také stalo.

Při svatbě Ant.\ Besty, obchodníka v~Porubě, vybral jednatel na sokolovnu 25~K 20~h.

Župa mor.\ slezská ujala se chudého synka Černé Hory, jižní Janovice, kterého dala studovat na reálce v~M.~Ostravě a~uložila každé jednotě něčím k~tomu ročně přispět. My jsme zasílali ročně 12~K k~tomu účelu.

\phantomsection
\subsection{Návrat bratra Besty}

Ke konci sluší podotknouti, že br.\ Antonín Besta starší, jeden ze zakladatelů Sokola, který působením bývalého hostinského a~nyní obchodníka v~Porubě, Rudolfa Hilla, z~jednoty v~r.~1902 vystoupil, svou chybu uznal a~kajícně se zase za člena přihlásil. Byl přijat a~od té doby projevuje se opět horlivým Sokolem, který přes své vysoké stáří všech podniků sokolských pilně se zúčastňuje.

\bigskip

\noindent Počet členů koncem roku 1908:
\begin{itemize}
\item zakládajících --- 7
\item cvičících --- 13
\item přispívajících --- 36
\end{itemize}
Úhrnem \textbf{56}.

\clearpage

%%%%%%%%%%%%%%%%%%%%%%%%%%%%%%%%%%%%%%%%%%%%%%%%%%%%%%%%%%%%%%%%%%%%%%%%%%%%%%%
% ROK 1909
%%%%%%%%%%%%%%%%%%%%%%%%%%%%%%%%%%%%%%%%%%%%%%%%%%%%%%%%%%%%%%%%%%%%%%%%%%%%%%%

\phantomsection
\section{Rok 1909}

Valná hromada konala se v~neděli 31.~ledna za účasti 27~členů. Zvoleni byli:

\begin{description}
\item[Starostou:] Adolf Bárta.
\item[Jeho nám.:] Ignác Blažej.
\item[Náčelníkem:] Josef Jirák.
\item[Náměstkem:] Josef Vaněk.
\item[Jednatelem:] Josef Bárta.
\item[Pokladníkem:] Adolf Sokol.
\item[Náhradníky:] Fr.\ Navrátil, Ludvík Dedoch, Vl.\ Košťřica, Jakub Halička.
\item[Knihovníkem a~archivářem:] Josef Vaněk.
\end{description}

\phantomsection
\subsection{Založení Družstva a~Občanské záložny}

K~nejvýznačnějším událostem toho roku dlužno počítati založení Družstva pro udržování sokolského domu v~Porubě a~zřízení Občanské záložny.

Důvody ku založení Družstva byly tyto: Již v~roku 1902, jak uvedeno, jsme se pokoušeli získati hostinskou koncesi pro sokolovnu, leč nepodařilo se to ani v~tomto ani v~příštích letech. Úřady zamítaly naše žádosti také z~toho důvodu, že sokolské jednoty nemají ve stanovách získávati hostinské koncese, ani že to není jejich úkolem. To jsme uznávali, ale zároveň pociťovali nutnou potřebu hostinské koncese, abychom předně mohli při slavnostech a~zábavách poskytnouti svým hostům žádaných pokrmů a~nápojů a~za druhé, abychom z~pronajaté hostinské živnosti měli zabezpečený stálý příjem pro udržování sokolovny.

Proto jsme na radu bratrů J.~Uherka a~Březiny, členů jednoty Sokol v~Přívoze, založili Družstvo pro udržování sokolského domu, jak to před tím učinili také v~Přívoze, a~do stanov jsme dali, že účelem Družstva je převzíti do své správy sokolovnu s~celým majetkem na tak dlouho, až by úplně vyplacena byla. K~dosažení toho účelu bude Družstvo vybírati příspěvky, pořádati zábavy a~snažiti se o~získání hostinské koncese.

Ustavení Družstva schváleno naší jednotou na mimořádné valné hromadě dne 9.~května 1909. Stanovy Družstva byly téhož roku zemskou vládou slezskou schváleny. Brzy po schválení podalo Družstvo žádost o~hostinskou koncesi okresnímu hejtmanství v~Bílovci, které ji po dlouhém klání a~různých intervencích br.\ Ig.\ Blažeje, Ad.\ Bárty a~jiných vlivných osob konečně povolilo a~sice na 10~let a~jen pro členy Družstva.

Leč naše radost z~povolení netrvala dlouho. Naše \enquote{milé} zastupitelstvo obce, v~němž vedl hlavní slovo hostinský Alois Klos, podalo proti povolení hostinské koncese stížnost na zemskou vládu v~Opavě, která ji příznivě vyřídila a~hostinskou koncesi nám odepřela.

Rozčilení naše možno si představit. Leč my nepopustili a~hledali cesty k~uskutečnění naší tužby. Na radu našich přátel obrátili jsme se na říšského poslance barona Rolsberga, velkostatkáře v~Litultovicích, aby intervenoval v~náš prospěch u~ministerstva obchodu ve Vídni, kam jsme žádali o~udělení koncese. Baron Rolsberg, který byl rodem Němec, ale přítelem českého lidu, v~jehož středu žil, slíbil tak učiniti a~slib svůj také splnil. Jenom jemu děkujeme, že nám ministerstvo obchodu (ministrem byl Dr.~Weisskirchner) konečně hostinskou koncesi povolilo v~těch poměrech jako okresní hejtmanství a~sice rozhodnutím ze dne 9.~ledna 1911. Jak povolení koncese účinkovalo na nás a~jak na \enquote{slavné} zastupitelstvo obce Poruby, netřeba tuším vyličovati. Naše devítileté úsilí bylo korunováno zdarem.

\medskip

Občanskou záložnu v~Porubě založili jsme z~těch důvodů, abychom nebyli v~peněžitých záležitostech odkázáni na místní Raiffeisenku, kterou měli v~rukách naši nepřátelé, a~pak proto, aby záložna, až se zmůže, mohla nás hmotně podporovati.

O~uskutečnění Družstva i~záložny má hlavní zásluhu br.\ Ig.\ Blažej, který si úřady osobně i~písemně vyjednával a~informace potřebné si obstarával.

\phantomsection
\subsection{Činnost tělocvičná}

Činnost tělocvičná byla přes usilování náčelníka bratra Josefa Jiráka chabá, poněvadž členové neměli chuti do cvičení.

Cvičebních hodin bylo za celý rok u~členstva 56, prům.\ cvičilo 6~čl. Dorost cvičil 31~hodin, počtem 4--10. Žáků bylo 13; z~nich cvičilo v~13~hodinách průměrně 6.

Veřejně vystoupila jednota:
\begin{itemize}
\item 15.~srpna slavnosti \enquote{Havlíčka} ve Střebovicích 6~bratrů v~kroji.
\item 22.~srpna v~Klimkovicích 8~bratrů v~kroji.
\item 29.~srpna ve Vřesině cvičilo 6~bratrů: Josef Vaněk, Val.\ Lazar, Špaček, Klimek a~Šlapeta, Josef Jirák hrazdu, bradla a~půrnosti.
\item Okrskového cvičení v~Mar.\ Horách se účastnili br.: Vaněk, Vl.\ Zdražila, Josef Jirák.
\item Župního výletu na Lysou Horu zúčastnil se br.\ Jos.\ Vaněk.
\end{itemize}

\phantomsection
\subsection{Vzdělávací a~zábavní činnost}
\begin{itemize}
\item 6.~ledna divadelní hra \emph{Pasekáři} od Sokola-Tůmy v~Porubě a~9.~ledna ve Střebovicích.
\item 24.~ledna jsme měli v~sokolovně ples, který byl zajímavý tím, že se konal bez hostinského. Žádný totiž ze zdejších hostinských nechtěl jít čepovat do sokolovny v~úmyslu, aby ples byl zmařen. Leč nepovedlo se jim. Usnesli jsme se, že dáme na ples vyšší vstupné, totiž pro pána 5~K, pro dámu 2~K, a~všem účastníkům po zaplacení vstupného všechno zdarma. Naše ženy připravily chutné pokrmy, muži obstarali nápoje, a~tak každý účastník dostal dobrou večeři a~nápoje dle libosti. Žádný víc neplatil než vstupné. Hostů bylo dosti a~všichni si libovali, že vyšli dobře, a~že tak příjemný ples ještě nezažili. I~jednotě zbylo něco korun.
\item 14.~března sehrána byla s~velkým úspěchem \emph{Oblaka} od Jar.\ Kvapila.
\item 12.~dubna sehrána fraška \emph{Bílá myška} od Hlavatého.
\item 16.~května sehrána veselohra \emph{Stranický pan Kaprál}.
\item 24.~října pořádána zábava pro prospěch milionového daru pro Ústřední Matici školskou, která vynesla čistých 60~K. Po proslovu o~významu Matice sehrána byla veselohra \emph{Mezi umělci} od Stolby; po ní čísla hudební a~zpěvní (Haasovy Slezské písně a~Bartoníčkovy Slovenské zpěvy).
\item 21.~listopadu sehrána veselohra \emph{Modrá Krev} od Distla. Po divadle podal náčelník br.\ J.\ Kořený zajímavé ukázky lidového zpěvu.
\item 31.~prosince Sylvestrová zábava se 2~aktovkami (\emph{Jen mám příjem} a~\emph{Sutifarka}) a~veselými čísly.
\end{itemize}

Mimo toho byla ještě v~březnu přednáška o~J.~A.~Komenském (Josef Bárta) a~v~červenci přednáška o~Mistru Janu Husovi (Ant.\ Cholský).

V~listopadu a~prosinci pořádány byly třikráte úspěšné přednášky pro členy i~hosty z~různých oborů (češtiny, pravopis, literatura, účetnictví, kalkulace živnostníků a~j.). Přednášeli: J.\ Kořený, Ad.\ Sokol, Josef Bárta. Účast na přednáškách přes 20~členů, výsledek dobrý.

K~výzdobě sokolovny koupeno 5~národních hesel. Vypůjčených knih ze sokolské knihovny bylo 313.

\bigskip

\noindent Členů bylo koncem roku \textbf{58}.

Z~uvedené zprávy vidno, že rok 1909 přinesl bohaté výsledky neúnavné práce vedoucích činovníků sokolských, kteří za svou práci byli známými nepřáteli místními všemožně pronásledováni. K~dosavadním nepřátelům přidružil se jako přední bojovník zdejší kaplan Vilém Schmied, který v~kázaních a~novinářských článcích štval proti Sokolům, pokrokářům a~nevěreckým učitelům; hlavně nadučitel Josef Bárta byl stálým terčem jeho vzteku a~mstivosti.

Než přes všechny překážky šlo se mírně ku předu.

\clearpage

%%%%%%%%%%%%%%%%%%%%%%%%%%%%%%%%%%%%%%%%%%%%%%%%%%%%%%%%%%%%%%%%%%%%%%%%%%%%%%%
% ROK 1910
%%%%%%%%%%%%%%%%%%%%%%%%%%%%%%%%%%%%%%%%%%%%%%%%%%%%%%%%%%%%%%%%%%%%%%%%%%%%%%%

\phantomsection
\section{Rok 1910}

Valná hromada konala se 16.~ledna za přítomnosti 42~členů. Zvoleni byli:

\begin{description}
\item[Starostou:] Adolf Bárta.
\item[Jeho nám.:] Alois Besta.
\item[Náčelníkem:] Robert Blažej.
\item[Náměstkem:] Karel Klimek.
\item[Jednatelem:] Josef Bárta.
\item[Pokladníkem:] Adolf Sokol.
\item[Členy výboru:] Karel Vícha (učitel v~Porubě), Ant.\ Besta st., Frant.\ Navrátil, Antonín Bárta, Josef Jirák.
\item[Náhradníky:] Ig.\ Blažej, Ambrož Zdražila, Fr.\ Kozub, Jakub Halička.
\end{description}

\phantomsection
\subsection{Činnost tělocvičná}

Činnost tělocvičná dosáhla za vedení obětavého a~vytrvalého br.\ Rob.\ Blažeje veliké dokonalosti, o~čemž svědčí získané ceny při okrskových závodech. Družstvo složené z~bratrů: Rob.\ Blažeje, Karla Klimka, Fr.\ Polanského, Vojtěcha Neuwirtha a~Vlad.\ Zdražily dobylo III.~ceny, Rob.\ Blažej získal pro sebe ještě zvláštní ceny.

V~91~hodinách cvičilo průměrně 11~cvičenců.

\begin{itemize}
\item 6.~března pořádán cvičební večírek ve Střebovicích: 12~br.\ cvičilo bradla a~koně.
\item 26.~června zúčastnilo se 5~bratrů okrskových závodů.
\item 3.~července cvičilo při župním sletu v~Orlové 8~br.\ prostná a~půrnosti.
\item 10.~července zúčastnilo se veř.\ cvič.\ v~Klimkovicích 10~bratrů v~kroji výletním, 4~v~obč., 6~bratrů cvičilo bradla a~půrnosti.
\item 21.~srpna cvičilo v~Klimkovicích opět 8~br.
\item 14.~srpna pořádala naše jednota ve Střebovicích s~tamějším spolkem \enquote{Havlíček} národní slavnost, při níž vypomáhaly jednoty z~Klimkovic, Zábřehu a~Mar.\ Hor. Naši cvičili ve 3~družstvech.
\item 12.~června jsme měli veřejné cvičení v~Porubě. Ač slavnost částečně déšť kazil, přece návštěva byla četná a~výsledek uspokojivý.
\item 4.~října pořádali jsme večírek na rozloučenou s~br.\ náčelníkem R.\ Blažejem, který nás opustil za vojenskou povinnost; při večírku cvičilo 6~br.\ hrazdu a~bradla.
\end{itemize}

Cvičení vedl dále br.\ Karel Klimek, rovněž dobrý pracovník.

\phantomsection
\subsection{Činnost zábavní a~vzdělávací}
\begin{itemize}
\item Února pořádán ples, který se zdařil a~vynesl čistého 109~K.
\item Března sehráno drama \emph{Gardina roba} v~Porubě, dubna opakováno drama ve Vřesině, dubna drama v~Klimkovicích, všude s~velkým zdarem morálním i~finančním.
\item 29.~května provedena veselohra K.\ Horkého \emph{Srážka vlaků}.
\item 14.~srpna zábavný večer s~programem hudebně pěveckým. Zpívány: Malátovy národní písně pro 2~ženské hlasy s~průvodem klavíru, solové písně, mužský sbor \enquote{Za krmáše}, smíšené sbory \enquote{Studentská} a~\enquote{Česká selská holubička}.
\item V~září sehrána veselohra \emph{Dcery pana Zajíčka}.
\item 4.~října při večeru na rozloučenou s~br.\ náčelníkem recitovány básně a~zpívány různé písně s~prův.\ klavíru.
\item 17.~října jednoaktovka \emph{Když se ženské smějou}, po ní recitace.
\item 18.~prosince sehráno veršované drama \emph{Na vždy} od Rydla. Jím dokázali svou uměleckou výši a~neúmornou píli. V~kuse hráli: sl.\ Gregorková, J.\ Kořený, Metoděj Zdražila, Karel Bartek, Ant.\ Cholský.
\item 31.~prosince Sylvestrovská zábava, při níž sehrány jednoaktovky \emph{Ženichové} a~\emph{O~půlnoci}. Účast pěkná.
\end{itemize}

\phantomsection
\subsection{Přednášky}

O~cvičebním večírku ve Střebovicích dne 6.~března promluvil Josef Bárta \enquote{O~uvědomění národním a~sokolské práci.}

Br.\ Sokol přednášel mládeži \enquote{O~ceně vzdělání pro život.}

Z~knihovny vypůjčeno 123~knih.

\bigskip

\noindent Členů koncem roku:
\begin{itemize}
\item zakládajících --- 6
\item činných --- 10
\item přispívajících --- 34
\end{itemize}
Úhrnem \textbf{50}.

\clearpage

%%%%%%%%%%%%%%%%%%%%%%%%%%%%%%%%%%%%%%%%%%%%%%%%%%%%%%%%%%%%%%%%%%%%%%%%%%%%%%%
% ROK 1911
%%%%%%%%%%%%%%%%%%%%%%%%%%%%%%%%%%%%%%%%%%%%%%%%%%%%%%%%%%%%%%%%%%%%%%%%%%%%%%%

\phantomsection
\section{Rok 1911}

Valná hromada konaná 15.~ledna za účasti 28~členů. Zvoleni byli:

\begin{description}
\item[Starostou:] Adolf Bárta.
\item[Jeho nám.:] Alois Besta.
\item[Náčelníkem:] Karel Klimek.
\item[Jeho nám.:] František Pavlíček.
\item[Jednatelem:] Rudolf Špaček.
\item[Pokladníkem:] Adolf Sokol.
\item[Ostatní členové:] Josef Bárta, Ig.\ Blažej, Method Zdražila, Ambrož Zdražila, Frant.\ Navrátil.
\item[Náhradníky:] Ant.\ Bárta, Jakub Hališka, Karel Bartek, Josef Vaněk.
\item[Revisory účtů:] Ant.\ Figalla (učitel v~Porubě), Fr.\ Blažej (stolař v~Porubě).
\end{description}

\phantomsection
\subsection{Činnost tělocvičná}

Činnost tělocvičná byla něco slabší než v~minul.\ roce. Cvičebních hodin bylo u~mužů 71, nejvíce 13, nejm.\ 5, prům.\ 8.1; u~dorostu 87, nejvíce 12, nejm.\ 6, prům.\ 9.27. Žáci pro čelné překážky necvičili.

\phantomsection
\subsection{Založení ženského odboru}

Přičiněním bratra Špačka byl založen ženský odbor, který v~měsíci květnu počal cvičit a~píle sester vynikla tak, že se zúčastnil i~veřejných podniků. Cvičebních hodin měl 64, nejv.\ cvičilo 12, nejm.\ 6, prům.\ 8~žen.

\begin{itemize}
\item 14.~května byl cvičební večírek, na němž účinkovaly sestry i~bratři ze Zábřehu n/O. Sestry cvičily prostná, bratři bradla.
\item 15.~června zúčastnili se župních závodů bratři: Karel Klimek, Fr.\ Pavlíček, Eda Jirák, Josif Borek, Ant.\ Panda, Leo Špaček. Ceny obdrželi: Fr.\ Pavlíček velký diplom, Borek malý diplom, celé družstvo pochvalné uznání. Br.\ Klimek dostal při závodech ve vyšším oddílu pochvalné uznání.
\item 16.~července při okrskovém cvičení ve Svinově cvičilo 8~mužů hrazdu a~prostná, 5~žen prostná a~11~dorostenců prostná.
\item 9.~července při veřejném cvičení v~Klimkovicích cvičilo 7~br.\ hrazdu a~půrnosti.
\item 6.~srpna bylo místní veřejné cvičení za spoluúčinkování jednot ze Svinova a~Studénky. Cvičilo 25~dorostenců prostná s~praporky, 20~mužů hrazdu, bradla a~prostná, ženský odbor prostná, sestry ze Studénky americké tance.
\item 20.~srpna cvičili v~Pustkovci bratři hrazdu a~prostná, ženy prostná původní a~dorost prostná s~praporky.
\item Též se jednota zúčastnila cvičení ve Staré Vsi, Zábřehu a~Hrabyni.
\end{itemize}

Výborových schůzí bylo 9, členské schůze 4, v~nich projednávány návrhy na zlepšení stavu jednoty.

\phantomsection
\subsection{Přístavba sokolovny}

Z~důležitých návrhů byl schválen návrh na přístavbu sokolovny, která se jevila nutnou vzhledem ku hostinské koncesi, povolené ministerstvem obchodu ze dne 9.~ledna 1911. Přístavbu si vzalo na starost Družstvo pro udržování sokolského domu, které ji zadalo ku provedení staviteli Raimundu Halcerovi z~Pustkovce. Stavba, která byla započata 14.~září a~ukončena v~listopadu, stála 11\,000~K. Stavba pozůstává dole z~výčepních místností a~nahoře z~bytu hostinského. Místnosti jsou útulné a~na nynější dobu vyhovující.

Hostinská koncese pronajata na rok br.\ Antonínu Bártovi, řezníku v~Porubě. V~sokolovně nastal nový ruch a~radostná práce.

\phantomsection
\subsection{Vzdělávací a~zábavní činnost}

Divadelních představení pořádáno 7:
\begin{itemize}
\item \emph{Strakonický dudák} --- z~výpravné hry J.~K.~Tyla.
\item \emph{Lesní panna} --- z~výpravné hry J.~K.~Tyla.
\item \emph{Samota chorobné květy} od Růžka.
\item \emph{Vojnarka} od Jiráska. Před divadlem přednášel Jos.\ Bárta o~Jiráskovi.
\item \emph{Klub mládenců} od Bałuckého.
\item \emph{České Amazonky} od Tyla.
\item \emph{Veselohra v~Kuchyni}, jednoaktovka.
\end{itemize}

30.~června navštívil nás pěvecký spolek \enquote{Vlastimil} z~Kateřinek a~uspořádal pěknou pěveckou produkci v~sokolovně.

Mimo toho byla Mikulášská a~Sylvestrovská zábava.

\phantomsection
\subsection{Spolupráce s~Dělnickou jednotou}

Dělnické tělocvičné jednotě \enquote{Lassalle}, která se toho roku při místní organisaci dělnické utvořila, byla propůjčena sokolovna i~s~nářadím k~používání dvakráte v~týdnu. Tím jsme chtěli udržeti přátelské styky mezi Sokolem a~Dělnickou organisací, abychom spojenými silami mohli čeliti silně se vzmáhajícímu místnímu klerikalismu.

Rozšířením sokolovny provedenou přístavbou jsme opět utužili svou posici.

\bigskip

\noindent Koncem roku bylo členů:
\begin{itemize}
\item zakládajících --- 8
\item cvičících --- 11
\item přispívajících --- 33
\end{itemize}
Úhrnem \textbf{52}.

\clearpage

%%%%%%%%%%%%%%%%%%%%%%%%%%%%%%%%%%%%%%%%%%%%%%%%%%%%%%%%%%%%%%%%%%%%%%%%%%%%%%%
% ROK 1912
%%%%%%%%%%%%%%%%%%%%%%%%%%%%%%%%%%%%%%%%%%%%%%%%%%%%%%%%%%%%%%%%%%%%%%%%%%%%%%%

\phantomsection
\section{Rok 1912}

Valná hromada konána 28.~ledna za účasti 32~členů. Zvoleni byli:

\begin{description}
\item[Starostou:] Adolf Bárta.
\item[Jeho nám.:] František Kozub.
\item[Náčelníkem:] Ant.\ Figalla, byl vyslancem do župy a~Č.O.S.
\item[Jeho nám.:] Václav Dvořák, stolař u~fy Blažej.
\item[Jednatelem:] Metoděj Zdražila, učitel ve Vřesině.
\item[Pokladníkem:] Adolf Sokol.
\item[Ostatní čl.\ výboru:] Josef Bárta, Alois Besta, Frant.\ Klos pt., Ant.\ Besta st., Jindřich Bernatsik (úředník fy Blažej).
\item[Náhradníky:] Ig.\ Blažej, Fr.\ Pavlíček, Fr.\ Matyáš, Arnošt Gelnar.
\item[Revisory účtů:] Vl.\ Košťřica a~Fr.\ Blažej.
\end{description}

\phantomsection
\subsection{Činnost tělocvičná}

Činnost tělocvičná byla dosti dobrá, zvláště v~I.~pololetí.
\begin{itemize}
\item Mužů zařaděno 8, průměrně cvičilo 5 ve 92~hodinách.
\item Žen zařaděno 8, průměrně cvičilo 6 v~67~hodinách.
\item Dorostu zařaděno 20, průměrně cvičilo 13 ve 94~hodinách.
\end{itemize}
Jeví se tedy potěšitelný přírůstek v~dorostu.

\subsubsection*{Veřejná vystoupení}
\begin{itemize}
\item 10.~března pořádán cvičební večírek, při něm čísla zpěvní.
\item 24.~března cvičební večírek ve Vřesině. Cvičilo 7~bratrů, 6~sester, 8~dorostenců.
\item 28.~dubna vycházka do Hošťálkovic (5~bratrů a~6~sester).
\item 2.~června místní veřejné cvičení za účasti jedn.\ Klimkovské (6), Velkopolomské (16~br.), Staroveské (3~br.), Svinovské (11~br., 8~s., 10~dorost.). Dorost cvičil s~dlouhou tyčí, muži hrazdu (Poruba), bradla (Svinov). Prostná mužů a~žen k~II.~sletu. Cvičení se vydařilo dobře.
\end{itemize}

\phantomsection
\subsection{VI.~slet všesokolský}

VI.~slet všesokolský konal se v~Praze od 29./6.--2./7. Z~naší jednoty se ho zúčastnilo 12~bratrů a~2~sestry.

V~kroji sokolském jeli br.: Ant.\ Bárta, Josif Bartek, Eda Jirák, Ant.\ Figalla, Karel Klimek, Gustav Kořený, Metoděj Zdražila; sestry: Miloslava Bártová, Marie Motýlová.

V~kroji občanském: Adolf Bárta, Ant.\ Besta st., A.\ Besta ml., Václav Dvořák, Fr.\ Pavlíček.

V~Praze cvičili prostná: Figalla, Pavlíček, Eda Jirák a~Mil.\ Bártová. Sletem se velice povzneslo vědomí sokolské u~všech.

\begin{itemize}
\item 28.~července zúčastnila se jednota veř.\ cvičení ve Svinově: 6~br.\ cvičilo prostná a~hrazdu, 7~žen prostná, 12~dor.\ prostná.
\item Také v~Klimkovicích bylo 28~bratrů, z~nich 8~v~kroji, 9~sester, 8~dorostenců na veř.\ cvičení.
\end{itemize}

\phantomsection
\subsection{Vzdělávací a~zábavní činnost}

\begin{tabular}{@{}lrr@{}}
8.4.\ hrána v~sokolovně \emph{Maryša} & hrubý příjem 80~K & čistý 56~K \\
21.4.\ opakována u~Švidrnochů & 76~K & 52~K \\
6.5.\ hrán \emph{Diblík, čertík z~hor} & 50~K & 25~K \\
27./10.\ hráno drama \emph{Karel Modrava} & 63~K & 48~K \\
Mikulášská zábava & & 46~K \\
Sylvestrovská zábava & & 53~K \\
\end{tabular}

\medskip

Z~toho vidět, že ze zábav byl celkem nepatrný zisk hmotný, ač všichni účinkující pracovali zdarma, a~že Družstvu udržování sokolovny působilo veliké obtíže. Na vydání přispěla Občanská záložna obnosem 200~K.

Členů bylo tolik jako minulého roku.

Toho roku položena v~sálu nová podlaha a~sál nově vymalován.

\clearpage

%%%%%%%%%%%%%%%%%%%%%%%%%%%%%%%%%%%%%%%%%%%%%%%%%%%%%%%%%%%%%%%%%%%%%%%%%%%%%%%
% ROK 1913
%%%%%%%%%%%%%%%%%%%%%%%%%%%%%%%%%%%%%%%%%%%%%%%%%%%%%%%%%%%%%%%%%%%%%%%%%%%%%%%

\phantomsection
\section{Rok 1913}

O~valné hromadě konané 16.~ledna zvoleni byli:

\begin{description}
\item[Starostou:] Adolf Bárta.
\item[Náměstkem:] Alois Besta.
\item[Náčelníkem:] Frant.\ Pavlíček (nově přistoupil).
\item[Jednatelem:] František Šedivý, učitel v~Porubě.
\item[Pokladníkem:] Adolf Sokol.
\item[Ostatní čl.\ výb.:] Ignác Blažej, Fr.\ Kozub, Josif Bárta, Ant.\ Figalla, Václav Dvořák, Gustav Kořený (který byl zvolen vzdělávatelem a~vyslancem do župy i~Č.O.S.).
\item[Náhradníci:] Ant.\ Bárta, Fr.\ Navrátil.
\end{description}

\phantomsection
\subsection{Činnost tělocvičná}

Činnost tělocvičná vykazuje slabé výsledky, členů cvičících málo.
\begin{itemize}
\item V~80~cvič.\ hodinách cvičilo průměrně 6~bratrů.
\item V~73~cvič.\ hodinách cvičilo průměrně 7~sester.
\item V~85~cvič.\ hodinách cvičilo průměrně 19~dorostenců.
\end{itemize}

15.~června konáno v~místě okrskové cvičení, které řídil okrskový náčelník br.\ Chalupa, okresní soudce z~Klimkovic. Posudek cvičení v~Sok.\ Věstníku byl pěkný. Muže vedl místonáčelník br.\ Eda Jirák, sestry vedla náčelnice s.\ Slávka Bártová.

Na zakoupení dorostenských krojů věnovala jednota 40~K.

Jednota také pomáhala k~tomu, aby v~Krásném Poli byla založena sokolská jednota. Ku cvičení půjčeny jí stará bradla.

\phantomsection
\subsection{Činnost vzdělávací a~zábavní}

Mimo 1~přednášky pořádáno bylo několik divad.\ představení:
\begin{itemize}
\item 24./3.\ \emph{Směry života}, hra od Svobody.
\item V~květnu: \emph{Závěť Lukavického pána} a~\emph{Hodinou manželé} (Hilbert).
\item 12./10.\ \emph{Ondráš, pán Lysé Hory} v~Porubě.
\item 18./10.\ \emph{Ondráš} opakován ve Svinově.
\item 19./12.\ \emph{Paní ministrová} a~\emph{Zvíkovský rarášek} od Stroupežnického.
\item 26./12.\ Jednoaktovky \emph{Věrnost}, \emph{Před svatbou}.
\item 31./12.\ Sylvestrová zábava.
\item 2./11.\ \emph{Josef Kajetán Tyl}.
\end{itemize}

Souhra všech divadel byla dobrá; mezi herci vynikli: G.\ Kořený, Fr.\ Navrátil, pí Sokolová. Také návštěva divadel byla obstojná.

\phantomsection
\subsection{Úmrtí}

Dne 6.~února 1913 zemřel v~útlém věku 14~let \textbf{Zdeněk Bárta}, syn nadučitele Josefa Bárty, žák III.~třídy českého gymnasia v~Mor.\ Ostravě a~horlivý člen dorostu sokolského. Jsa velice nadaným hudebníkem, hrával při zábavách sokolských jako pianista aneb jako violista hudební kapely, kterou v~sokolovně zařídil a~pro niž složil také několik hudebních kusů.

\begin{novinovy-vystrizek}
Odešel, kdy počínal k~dobru toho lidu svého růsti. Mravenčí píle, vždy v~prvních řadách učení, v~píli i~v~mravech. Gymnasium ostravské ztrácí svého nejschopnějšího žáka, s~ním uložen v~hrob i~velký hudební talent, jenž již záhy zračil se v~bravurní výkonnosti na pianě, v~dovedném ovládání několika hudebních nástrojů. Proces jeho uměleckého vzestupu zůstal utajen mezi privatissimy --- škoda, že veřejnost nemůže uzříti resultát jeho uměleckého cítění. Co jsem se jej naposlouchal, jak vyluzoval ze zažloutlé klaviatury tvrdé, hrotové akkordy odbojných písní národních, českých, srbských, slovanských, ve sterých variacích.

Jako dorostenec Porubského \enquote{Sokola} rád podroboval se dobrovolné kázni a~byl vždy věrný a~oddaný zákonu Tyršovu. Žil těchto pár dnů dle Tyršovy stručné, leč výmluvné devise: \enquote{Věčný ruch}. Všímal si otázek veřejných a~kulturních a~ještě nedávno dal podnět ku zřízení dorostenské besídky v~Sokole Porubském i~byl krátce před smrtí pověřen funkcí knihovníka.

Odžil dny života svého! Drahý Zdeňku můj! Duše skřivánčí! Věř, že dokud dech v~nás, bude žít Tvá v~nitrech našich čistá památka, neb kdykoli by blednout začala, Tvé prázdné místo steskem bolestným ji zas a~hlouběj nám tam vryje.

\hfill\textit{-ř-}
\end{novinovy-vystrizek}

\medskip

V~květnu stihla naši jednotu nová rána úmrtím bratra \textbf{Metoděje Zdražily}, vynikajícího člena naší jednoty, jenž byl v~r.~1912 jejím jednatelem. Byl synem br.\ Ambrože Zdražily, jednoho z~nejstarších a~nejvěrnějších členů naší jednoty a~zdědil po něm oddanost a~lásku k~sokolské myšlence. Byv ustanoven učitelem ve Vřesině, oddal se s~nadšením práci ve škole i~v~obci. Byl milován nejen tou školskou drobotinou, nýbrž i~veškerým občanstvem, v~němž budil zájem pro dobro a~krásno. Pořádal školské výstavy, přednášky i~divadla dětská, a~při tom přicházel k~nám, do své rodné obce, aby s~námi v~Sokolu pracoval k~docílení kulturní a~duchovní samostatnosti našeho lidu. Jako člen pěveckého i~divadelního odboru podal nám pěkné ukázky své píle a~svého umění.

Leč i~ten dobrý člověk byl předmětem šílené nenávisti zdejšího kaplana P.~Schmieda, jenž jej při každé příležitosti i~před školními dětmi urážel a~zesměšňoval, takže musel hledati po dvakráte u~soudu ochrany. Těmito soudními spory byl beztoho pak porušen, že trpěl veliká duševní muka, která vedla k~zápalu mozkových blan a~k~náhlé smrti. --- Ve 22~letech zmařen lidskou zlobou život muže, který oprávňoval k~nejlepším nadějím. Sokoli z~Poruby i~z~okolí, davy učitelů, sdružení pěvců, zástupci spolků, dítky a~občané ze Vřesiny a~Poruby doprovodili Dobrého učitele a~Sokola na porubský hřbitov. --- Kéž jeho život je nám světlým příkladem vytrvalé a~nezištné práce!

Výdajů měla jednota toho roku mnoho, neboť do sálu sokolovny dána nová podlaha a~pořízena nová malba.

\bigskip

\noindent Členů koncem roku \textbf{35}.

\clearpage

%%%%%%%%%%%%%%%%%%%%%%%%%%%%%%%%%%%%%%%%%%%%%%%%%%%%%%%%%%%%%%%%%%%%%%%%%%%%%%%
% ROK 1914
%%%%%%%%%%%%%%%%%%%%%%%%%%%%%%%%%%%%%%%%%%%%%%%%%%%%%%%%%%%%%%%%%%%%%%%%%%%%%%%

\phantomsection
\section{Rok 1914}

Ve valné hromadě, konané 18.~ledna za účasti 18~členů a~3~hostů, byli zvoleni:

\begin{description}
\item[Starostou:] Alois Besta.
\item[Jeho nám.:] Adolf Bárta.
\item[Náčelníkem:] Ant.\ Figalla.
\item[Jednatelem:] Fr.\ Šedivý.
\item[Pokladníkem:] Adolf Sokol.
\item[Ostatní čl.\ výboru:] Fr.\ Kozub, Josef Bartek, Fr.\ Pavlíček, Václav Dvořák, Eda Jirák, Robert Blažej.
\item[Vzdělavatelem:] Josef Bárta.
\item[Knihovníkem:] Václav Dvořák.
\end{description}

Br.\ Adolf Sokol poděkoval dosavadnímu starostovi br.\ Adolfu Bártovi za dlouholeté obětavé vedení jednoty v~dobách nejkritičtějších; jeho zásluhy o~jednotu uznává každý, kdo s~ním od začátku pracoval.

Když br.\ Ad.\ Bárta funkci náměstka starosty nepřijal, byl zvolen náměstkem br.\ Fr.\ Kozub.

\phantomsection
\subsection{Činnost tělocvičná}

Činnost tělocvičná, která v~prvém pololetí šla obvyklou cestou, byla v~druhé polovici vypuknutím války a~nastalou mobilisací přerušována a~mařena. Z~bratrů odešlo do konce roku za vojenskou povinností 19. Z~výboru nastoupili vojenskou službu br.\ Šedivý, br.\ Václav Dvořák, Eda Jirák a~Robert Blažej.

\phantomsection
\subsection{Úmrtí bratra Josefa Vaňka}

Navždy nás opustil br.\ Josef Vaněk, vzpomenutý již v~r.~1908, který od chlapeckých let lnul upřímně k~sokolství, byl vyspělým, neúnavným borcem, správcem knihovny a~svým stálým sebevzděláváním a~čistým životem byl vzorem Sokola. Pohřben byl na zdejším hřbitově za četné účasti 17./XI.

\phantomsection
\subsection{Zábavy a~kulturní činnost}
\begin{itemize}
\item 21.~února masopustní zábava.
\item 15.~března beseda členská s~přednáškou o~Mir.\ Tyršové.
\item 8.~března divad.\ představení: \emph{Zpověď}, drama od Rutha.
\item 13.~dubna hrána veselohra \emph{Na přelíčení} od Krylova.
\item 19.~dubna táž veselohra opakována v~hostinci Fr.\ Švidrnocha ve prospěch Spolku pro ošetřování nemocných v~Porubě.
\item 24.~května konána na oslavu 20letého trvání Sokola zahradní slavnost, po níž sehrána večer v~sokolovně veselohra Bałuckého \emph{To smetí!}.
\item 26.~prosince sehráno Ladeckého drama \emph{Bez lásky} ve prospěch Červeného kříže.
\end{itemize}

Sokolovna byla od 15.~července 1914 pronajata Josefu Lazarovi za roční nájemné 900~K.

\phantomsection
\subsection{Vypuknutí války}

Po vypuknutí války byly mysli naše upjaty k~válečným událostem, neboť hned z~počátku jsme tušili, že se jedná nejen o~pokoření Srbska a~Ruska, nýbrž o~zeslabení všech Slovanů, zvláště rakouských a~v~prvé řadě nás Čechů, proto se naše sympatie nesly ku Trojdohodě. Počáteční veliká vítězství Němců ve Francii, oznamovaná chvástavě velikými písmeny v~novinách, nás skličovala, avšak naděje nám nebrala, neb jsme doufali v~nevyčerpatelné síly Ruska i~prozíravost vypočítavé Anglie.

Když jsme se dověděli, že vlaky s~raněnými zastavují se při průjezdu ve Svinově, tu jsme ihned zahájili podpůrnou akci. Sbírali jsme mezi členy i~nečleny peníze a~potraviny, které jsme dodávali do Svinova a~tam na nádraží podávali raněným vojínům. Naším přičiněním darováno a~odvezeno bylo do Svinova: 10~hl mléka, 250~pecnů chleba, 9~kop vajec, 40~kg másla, 700~koláčů, uzeniny a~cigarety. Na nádraží měli jsme stánek a~kamna, na nichž jsme vařili pro vojáky mléko, kávu a~čaj.

Podělování se zúčastnili sestry: Anna Bártková, Leop.\ Bártová, Slávka Bártová, Anastasie Bártová, Ant.\ Kozubová, Marie Sokolová a~sestry Dostálové. Z~bratrů: Ant.\ Bárta, Jaromír Bárta, Josef Bárta, Adolf Sokol. Podělování trvalo ve dne i~v~noci asi 3~týdny, až konečně ustalo pro chladné a~deštivé počasí.

Téhož roku byl povolán k~vojsku náš všestranný umělec br.\ Gustav Kořený, nadučitel ve Vřesině, jehož odchodem pozbyli jsme našeho nejlepšího člena vzdělávacího a~divadelního odboru.

\phantomsection
\subsection{Bilance sokolského majetku koncem r.~1914}

\begin{tabular}{@{}lr@{\hspace{2cm}}lr@{}}
\multicolumn{2}{c}{\textbf{Aktiva}} & \multicolumn{2}{c}{\textbf{Passiva}} \\
Pokladní hotovost & 58.69~K & Obč.\ zál.\ v~Klimkovicích & 6\,000.---~K \\
Závodní podíly u~Obč.\ záložny v~Klimk. & 300.---~K & Obč.\ zál.\ v~Porubě & 10\,709.17~K \\
Záv.\ pod.\ u~Obč.\ zál.\ v~Třebovici & 100.---~K & Obč.\ zál.\ v~Porubě & 1\,500.---~K \\
Na běžném účtu u~Obč.\ zál.\ v~Porubě & 161.52~K & Ad.\ Bártovi & 2\,889.---~K \\
Hodnota nemovitosti & 21\,281.93~K & Členské Obci Sokolské & 1\,500.---~K \\
Hodnota inventáře & 1\,482.---~K & Úroky Ad.\ Bártovi & 194.40~K \\
U~Fr.\ Míky & 1.15~K & Zed.\ mistru Hrubešovi & 56.20~K \\
U~Jos.\ Lazara & 300.49~K & & \\
\textbf{Celkem} & \textbf{23\,685.78~K} & \textbf{Celkem} & \textbf{22\,839.77~K} \\
\end{tabular}

\bigskip

\noindent Čisté jmění \textbf{846.01~K}

\clearpage

%%%%%%%%%%%%%%%%%%%%%%%%%%%%%%%%%%%%%%%%%%%%%%%%%%%%%%%%%%%%%%%%%%%%%%%%%%%%%%%
% ROK 1915
%%%%%%%%%%%%%%%%%%%%%%%%%%%%%%%%%%%%%%%%%%%%%%%%%%%%%%%%%%%%%%%%%%%%%%%%%%%%%%%

\phantomsection
\section{Rok 1915}

Zvoleni byli:

\begin{description}
\item[Starostou:] Ant.\ Besta, starší.
\item[Jeho nám.:] František Kozub.
\item[Náčelníkem:] Ant.\ Figalla.
\item[Jednatelem:] Josef Bartek.
\item[Pokladníkem:] Adolf Sokol.
\item[Ostatní čl.\ výb.:] Adolf Bárta, Josef Bárta, Richard Heinrich (učitel v~Porubě), Alois Besta, Ambrož Zdražila, Frant.\ Klos pt.
\item[Náhradníky:] Fr.\ Valder, Cyril Dostál, Fr.\ Pavlíček, Ludvík Kos.
\item[Vzdělavatelem:] Josef Bárta.
\end{description}

Naděje naše na brzké ukončení války se nesplnily. Stálými odvody řady naše ještě více prořídly. K~válečným úkonům povoláno bylo opět 9~bratrů, mezi nimi skoro všichni cvičící členové s~náčelníkem br.\ Figallou, jednatelem br.\ Jos.\ Bartkem, i~jeho nástupcem br.\ Heinrichem.

Do června cvičil ještě dorost, pak cvičení přestalo.

\phantomsection
\subsection{Padlí a~zemřelí bratři}

Smrť nám odňala 4~dobré a~milé bratry: Ambrože Zdražilu, Václava Dvořáka, Josefa Sladkého a~Jaromíra Bártu.

\textbf{Ambrož Zdražila}, 60~let starý, byl zakládajícím členem a~setrval věrně v~našich řadách po 20~let. Zúčastnil se v~kroji každého veřejného vystoupení a~byl v~Praze na každém sletu. Své děti brával s~sebou také do Prahy, aby jim ukázal její krásy a~všechny památky české slávy. Byl prodchnut zásadami sokolskými, znal důkladně dějiny českého národa a~již v~letech osmdesátých minulého století, jak na začátku této kroniky poznamenáno, stál mezi těmi, kteří šířili v~naší obci uvědomění národní. Zemřel 14./6.

\textbf{Václav Dvořák}, rodem z~Nadějkova v~Čechách, přišel r.~1908 jako stolařský pomocník do zdejší továrny Ig.\ Blažeje. Od 8.1.1910 byl členem Sokola, horlivým cvičencem, knihovníkem a~dobrým divadelním ochotníkem. Vynikal poctivým smýšlením, milou povahou a~neúnavnou pracovitostí. Sotva si založil mladou domácnost, musel opustit svoji choť i~malou dcerušku, aby se více nevrátil. Dle soukromých zpráv zahynul v~listopadu 1914 na srbském bojišti.

Hrdinskou smrt na ruském bojišti nalezli bratři:

\textbf{Josef Sladký}, nar.\ 27.8.1894 v~Porubě, syn dělníka Ant.\ Sladkého, byl nadaný a~pilný mladík, vychodil s~dobrým prospěchem měšťanskou i~hornickou školu, a~ačkoliv jeho zaměstnání při hornictví jej vysilovalo, přece rád zúčastňoval se sokolské práce jako čilý a~neúnavný borec. Byl též horlivým čtenářem a~abstinentem. Padl v~1.~polovici r.~1915 na haličském bojišti.

\textbf{Jaromír Bárta}, nar.\ 10./9.~1896, syn nadučitele Jos.\ Bárty, vystudoval nižší reálku a~obchodní školu. Na začátku války přejal místo účetního ve zdejší továrně. Vstoupil do Sokola a~pracoval v~divadelním odboru. Již jako student hrával jako výborný houslista se svou sestrou při produkcích v~sokolovně. Byl odveden 15.~března jako 19letý k~vojsku, poslán byl 1.~června na haličskou frontu, kde v~zuřivých bojích byl 9.~června těžce zraněn. Zemřel 12.~června a~pohřben v~Kolomyji na tamějším řeckokatolickém hřbitově.

Škoda těchto slibných, tak předčasně zmařených životů! Zachovejme těmto zemřelým bratrům čestnou paměť a~v~dobrých skutcích je následujme!

\phantomsection
\subsection{Různá činnost}
\begin{itemize}
\item 14.~února sehrány 2~veselohry: \emph{Dvě jizvy} a~\emph{Ochrana Napoleona I.}
\item 14.~března byl večírek na rozloučenou s~odvedenými bratry.
\item 5.~září hrána vesel.\ \emph{Bílý lístek} v~Porubě.
\item 12.~září opakována ve Svinově.
\item 26.~září opakována v~Klimkovicích, všude s~dobrým výsledkem.
\item 7.~listopadu hrána \emph{Veselohra} od Jeřábka.
\end{itemize}

Z~čistého výtěžku věnováno Červenému kříži 46.50~K, měšťankám v~Klimkovicích a~ve Svinově po 20~K.

Při pořádání divadel nám vypomáhali br.\ Oldřich Vaca a~Karel Bartek ze Svinova, učitel Lacurka z~Heřmanic, učitel Jos.\ Martiník ze Vřesiny a~učitelka Marta Součková.

Z~35~členů zůstalo doma 12 a~6~sester. Zbylí členové se snažili, aby pořádáním divadel udrželi život jednoty a~opatřili příjmy k~udržení sokolovny.

Za odešlého jednatele vedla agendu sestra Slávka Bártová.

\clearpage

%%%%%%%%%%%%%%%%%%%%%%%%%%%%%%%%%%%%%%%%%%%%%%%%%%%%%%%%%%%%%%%%%%%%%%%%%%%%%%%
% ROK 1916
%%%%%%%%%%%%%%%%%%%%%%%%%%%%%%%%%%%%%%%%%%%%%%%%%%%%%%%%%%%%%%%%%%%%%%%%%%%%%%%

\phantomsection
\section{Rok 1916}

O~valné hromadě konané 2.~dubna za účasti 12~členů a~2~hostů byli zvoleni:

\begin{description}
\item[Starostou:] Antonín Besta st.
\item[Náměstkem:] Frant.\ Kozub.
\item[Náčelníkem:] František Besta, zámečník.
\item[Členy výboru:] Miloslava Bártová (jednatelkou), Cyril Dostál (pokladníkem), Josef Bárta (vzdělávatelem), Adolf Bárta (hospodářem), Adolf Sokol, Alois Besta, Frant.\ Klos, Frant.\ Pavlíček.
\item[Náhradníky:] Fr.\ Valder, Ludvík Kos, Marie Motýlová, Herma Kysková (knihovníkem).
\item[Vyslancem do župy:] Fr.\ Besta.
\end{description}

Z~počátku roku cvičili dorostenci, protože mužů nebylo, a~ženy. Leč nové odvody odváděly nám poslední zbytky cvičenců k~vojsku, takže cvičení vázlo, a~když v~sokolovně ubytována byla vojenská posádka, přestalo časem úplně.

Z~výboru a~starších členů sloužili u~vojska: Frant.\ Kozub, Adolf Bárta, Alois Besta, Ignác Blažej. Josef Bárta a~Adolf Sokol byli odvedení, leč jako učitelé byli vyreklamováni.

V~sokolovně bylo nevlídně; ubytovaní polští a~rusínští vojáci dělali nepořádek a~nečistotu. V~sále nabity do stěn ohromné hřebíky, na nichž visely vojenské cáry ubohých, špinavých vojínů, kteří teskně vzpomínali svých vzdálených domovů.

Přes to snažil se vzdělavatel, aby aspoň občasnými divadly udržel jakýsi život v~sokolovně, a~nacvičil za pomoci přátel již v~minulém roku uvedených tyto divadelní hry, jež byly v~sokolovně sehrány:
\begin{itemize}
\item 1./1.\ \emph{Následky dostaveníčka}, veselohra od Khünla.
\item 5./3.\ Veselohry: \emph{Dynamit} a~\emph{Rozbitá sklenice}.
\item 24./4.\ Veselohra od Rutha: \emph{Sestřenka}.
\item 8./10.\ Slavnostní schůze.
\end{itemize}

Po čas divadel vyklidili vojáci se souhlasem velícího důstojníka sokolovnu. Z~výtěžku divadel musela býti z~nařízení okresního hejtmanství odvedena část Červenému kříži.

\clearpage

%%%%%%%%%%%%%%%%%%%%%%%%%%%%%%%%%%%%%%%%%%%%%%%%%%%%%%%%%%%%%%%%%%%%%%%%%%%%%%%
% ROK 1917
%%%%%%%%%%%%%%%%%%%%%%%%%%%%%%%%%%%%%%%%%%%%%%%%%%%%%%%%%%%%%%%%%%%%%%%%%%%%%%%

\phantomsection
\section{Rok 1917}

Valná hromada nebyla konána a~pak podrželi funkce zbylí členové výboru.

Poměry byly zrovna tak neutěšené jako v~minulém roce, život v~sokolovně skoro utuchl, poněvadž se členů i~bývalých hostů dlela většina u~vojska. Pro nedostatek petroleje i~uhlí málo se svítilo a~topilo, takže večery každý raději trávil doma.

Protože hostinský Josif Lazar konal též vojenskou službu, převzal sokolovnu 1.~ledna obchodník Cyril Dostál, který však se jí nemohl věnovati pro zaměstnání ve svém obchodě smíšeným zbožím; proto se 15.~prosince t.r.\ vzdal hostinství ve prospěch Karla Lisníka, dříve hostinského v~Polance.

\phantomsection
\subsection{Činnost tělocvičná}

Činnost tělocvičná byla velmi nepatrná; částečně cvičily sestry, z~nichž s.\ Slávka Bártová a~s.\ Herma Kysková zúčastnily šermovnického kursu a~nacvičných sjezdů ve Vítkovicích, pak sokolských večírků v~Polance, v~Lúboři a~Klimkovicích.

\phantomsection
\subsection{Činnost divadelní}
\begin{itemize}
\item 6.1.\ sehrána veselohra \emph{Láska kvete v~každém věku}.
\item 21.1.\ táž veselohra opakována ve prospěch Spolku pro ošetřování nemocných v~Porubě.
\item 9.4.\ veselohra \emph{Tapínkovy nálety}.
\item 27.5.\ veselohra \emph{Žabec} od Stružného.
\item 26.12.\ sehrána \emph{Gazdina roba}, drama G.\ Preissové.
\item 30.12.\ opakována \emph{Gazdina roba} ve Velké Polomi.
\item 17.6.\ a~26.8.\ pořádán koncert na zahradě u~sokolovny.
\end{itemize}

Z~divadel a~koncertů pořádaných v~r.~1916 a~1917 hrazeny byly různé výdaje sokolské. Mimo to bylo odvedeno:
\begin{itemize}
\item Červenému kříži \hfill 108.50~K
\item Na Účetní dny (ve prospěch vojska) \hfill 64.---~K
\item Sirotčímu spolku v~Klimkovicích \hfill 130.---~K
\item Matici Opavské \hfill 105.50~K
\item Matici Osvěty Lidové \hfill 39.---~K
\end{itemize}

Nájemné z~hostinské živnosti bylo v~letech válečných jen 400~K ročně.

\bigskip

\noindent Členů koncem roku bylo:
\begin{itemize}
\item zakládající --- 3
\item cvičících --- 5
\item přispívajících --- 16
\end{itemize}
Úhrnem \textbf{24}.

\clearpage

%%%%%%%%%%%%%%%%%%%%%%%%%%%%%%%%%%%%%%%%%%%%%%%%%%%%%%%%%%%%%%%%%%%%%%%%%%%%%%%
% ROK 1918
%%%%%%%%%%%%%%%%%%%%%%%%%%%%%%%%%%%%%%%%%%%%%%%%%%%%%%%%%%%%%%%%%%%%%%%%%%%%%%%

\phantomsection
\section{Rok 1918}

Valná hromada konala se 30.~března za účasti 18~členů. Do výboru byli zvoleni:

\begin{description}
\item[Starostou:] Antonín Besta, st.
\item[Náměstkem:] Adolf Bárta.
\item[Náčelníkem:] Frant.\ Besta.
\item[Jednatelem:] Josef Bartek.
\item[Pokladníkem:] Ludvík Dedoch.
\item[Ostatní:] Adolf Sokol, Josef Bárta, Alois Besta, Frant.\ Klos pt., Frant.\ Valder.
\item[Náhradníky:] Fr.\ Klos ml., Fr.\ Pavlíček, Marie Motýlová, Herma Kysková.
\item[Vyslancem do župy:] Adolf Sokol a~Fr.\ Besta.
\end{description}

Tento rok pak významný pro náš národ, kdy konečně po 300leté porobě a~ponížení dosáhl své samostatnosti, přinesl i~naší jednotě oživení a~vzpružení k~nové radostné činnosti. Projev spisovatelů českých i~památná přísaha pražská budily radostný ohlas po širých vlastech československých, sílily naději na brzké ukončení války a~na lepší příští našeho národa. Soukromé zprávy, jež si věrní Čechové potajmu navzájem sdělovali, potvrzovaly, že zánik říše rakousko-uherské je neodvratný, že přičiněním našich slavných emigrantů Masaryka, Beneše aj., a~hrdinnými boji našich legionářů získali jsme obdiv a~uznání celé Dohody, jež nám zajišťuje úplnou samostatnost.

Toto vědomí nás naplňovalo hrdostí, sebevědomím a~budilo novou chuť ku práci. Starší bratři byli již doma a~mladší se pomalu vraceli.

\phantomsection
\subsection{Činnost cvičební}
\begin{itemize}
\item Muži cvičili 73~hodin, průměrně 10½.
\item Žen cvičilo 5.
\item Dorost cvičil 69~hodin, průměrně 14.
\end{itemize}

Června přednášela sestra Slávka Bártová o~významu tělocviku sokolského. Července br.\ Oskar Brázda o~Mistru Janu Husovi.

\phantomsection
\subsection{Veřejné cvičení}

Srpna pořádáno veřejné cvičení se slavností na zahradě u~sokolovny. Muži cvičili bradla, hrazdu a~prostná, ženy chór s~věnci a~soustavná prostná, dorost prostná, skok letmo přes šňůru, dorostenky Honičku.

Po cvičení pronesl nadšenou řeč o~sokolské práci a~povinnosti Čecha JUC Kahánek z~Vítkovic. Po proslovu zpívalo obecenstvo s~nadšením \enquote{Kde domov můj!} a~\enquote{Hej Slované!} Návštěva přes nepříznivé dopolední počasí byla slušná, čistý výnos ze slavnosti 768~K.

Srpna zúčastnili se bratři i~sestry veř.\ cvičení v~Krásném Poli. Bylo 6~bratrů a~3~dorostenky v~Háji.

Září zúčastnilo se památného táboru na Ostré Hůrce 30~bratrů a~8~sester. Měli jsme krásně vypravený alegorický sokolský vůz. Účast slezského lidu na táboru byla obrovská, nadšení veliké. Z~četných řečníků nejkrásněji promluvil poslanec Udržal a~spisovatel Jaroslav Kvapil. Dojemná a~povznášející chvíle byla, když z~tisíce úst nesla se slavnostní přísaha, že neustaneme, dokud nepozdravíme samostatnost českého národa!

29.~září zúčastnili se 4~br.\ a~1~sestra veř.\ cvičení v~Olomouci.

\phantomsection
\subsection{Jiná činnost}

Pořádána byla 1~schůze členská a~11~schůzí výborových. V~červnu byl pořádán zahradní koncert porubské kapely. V~srpnu sehráli naši studenti v~Porubě a~ve Svinově operettu \emph{Z~českých mlýnů} ve prospěch sokolovny. Příjem 400~K.

3.~listopadu hrána veselohra \emph{V~Ohni}. O~Dušičkách uspořádali jsme sbírku ve prospěch Českého srdce a~pozůstalých po českých legionářích.

Dne 10.~listopadu byla z~našeho podnětu založena v~sokolovně odbočka Čes.\ Srdce; předsedou zvolen Josef Bárta.

Dne 8.~prosince uspořádána Mikulášská zábava s~výtěžkem 322~K. 14.~prosince vojenský koncert ve prospěch pozůstalých po čes.\ legionářích. 26.~prosince sehrána byla \emph{Maryša}, drama bratří Mrštíků.

V~prosinci byla propůjčena sokolovna ku žákovskému představení.

\phantomsection
\subsection{Úmrtí}

\textbf{Br.\ Jan Konečný}, nar.\ 9.~listop.\ 1896 v~Porubě, syn pekařského mistra zdejšího, byl upřímně oddán věci sokolské, slouže jí jako vytrvalý cvičenec od žákovských let. Byl odveden v~roku 1915, vrátil se koncem března t.r.\ na dovolenou a~zúčastnil se s~plným zájmem sokolské valné hromady dne 30.~března, při níž zaplatil příspěvky na celý rok. Avšak hned po valné hromadě onemocněl zápalem plic, zaviněným dle lékařského dobrozdání dlouholetými útrapami vojenskými, a~přes veškeré opatrování starostlivé matky, podlehl zákeřné nemoci 10.~dubna 1918.

\textbf{Br.\ Vladislav Bárta}, nar.\ 26.~října 1897 v~Porubě, syn nadučitele, vrátil se po půlleté namáhavé vojenské službě 28.~října 1918 domů, těšil se ještě několik dnů z~našeho osvobození, avšak zhoubná nemoc --- chřipka --- toho roku zle řádící, stihla i~jej a~zlomila jeho mladý život 12.~listopadu 1918. Vypomáhal před svým odchodem k~vojsku při pořádání divadelních představení a~při hudebních produkcích.

Těžce se loučilo členstvo Sokola s~oběma dobrými bratry, kterým nebylo dopřáno pracovati dále v~osvobozené vlasti. Čest jejich památce!

\phantomsection
\subsection{Naše osvobození}

Zpráva o~radostném převratu došla do naší obce v~úterý 29.~října 1918 o~7.~hodině ráno. Ihned vyvěšeny prapory na škole, sokolovně a~jiných veřejných budovách a~konány přípravy k~důstojné oslavě tohoto dne. Ředitel továrny br.\ Ig.\ Blažej objednal telefonicky z~Mor.\ Ostravy řečníka, redaktora Ostravského Denníka Knolka a~místní hudbu. Také pozvána byla obec Vřesina.

O~3.~hodině odp.\ seřadil se veliký průvod občanů porubských a~vřesinských před Obecním Domem a~šel za zvuku hudby a~zpěvu národních písní cestou ku Svinovu. Setkav se za vesnicí s~objednaným řečníkem, vrátil se zástup zpět ku středu vesnice, kde před Obecním domem byla připravena řečnická tribuna. Redaktor Knolek vylíčil krásnými a~dojemnými slovy radostný převrat, který se stal věkopamátného dne 28.~října 1918, a~jaká radost panuje v~matičce Praze a~v~celém národě. Nabádal ku klidu, pořádku a~k~práci. Lid přijal tuto zprávu se slzami radosti v~očích a~vzrušen do nejvyšší míry provolal slávu našim osvoboditelům a~zapěl s~nadšením naši národní hymnu \enquote{Kde domov můj!} Tak ještě dlouho hrála hudba a~lid se veselil, že udeřila hodina našeho osvobození a~že se vrátila k~nám vláda věcí našich, jak po ní toužil věštec národa, slavný náš Jan Amos Komenský.

A~jistě Sokolstvo má velký podíl na našem osvobození, neboť hned od jeho vzniku byl v~něm živen odpor k~našim utlačovatelům a~buzena touha po svobodě. Ze sokolských řad povstaly hlavně naše legie, které po boku čtyřdohody bojovaly statečně proti odvěkým nepřátelům našim.

Kéž Sokolstvo i~nadále zůstane štítem národa a~ochráncem jeho svatých práv!

\phantomsection
\subsection{Počet členstva}

Z~vojny se vrátilo 42~členů, takže koncem roku čítala jednota členů:
\begin{itemize}
\item zakládajících --- 3
\item činných: bratrů 19, sester 5
\item přispívajících: bratrů 30, sester 14
\end{itemize}
Úhrnem \textbf{71}.

\clearpage

%%%%%%%%%%%%%%%%%%%%%%%%%%%%%%%%%%%%%%%%%%%%%%%%%%%%%%%%%%%%%%%%%%%%%%%%%%%%%%%
% ROK 1919
%%%%%%%%%%%%%%%%%%%%%%%%%%%%%%%%%%%%%%%%%%%%%%%%%%%%%%%%%%%%%%%%%%%%%%%%%%%%%%%

\phantomsection
\section{Rok 1919}

Valná hromada byla konána 19.~ledna za účasti 42~členů. Zvoleni byli:

\begin{description}
\item[Starostou:] Ant.\ Besta, st.
\item[Náměstkem:] Adolf Bárta.
\item[Náčelníkem:] Frant.\ Besta.
\item[Náčelnicí:] Slávka Bártová.
\item[Jednatelem:] Josef Bartek.
\item[Pokladníkem:] Běta Chlebková.
\item[Ostatní:] Ignác Blažej, Josef Bárta, Fr.\ Klos ml., Fr.\ Pavlíček, Karel Klimek, Alba Elötygrová.
\item[Náhradníky:] Emil Mika, Bohuš Bárta, Fr.\ Valder, Leop.\ Martiník.
\item[Prohlížitelé účtů:] Frant.\ Kozub, Alois Besta.
\item[Vzdělavatelem:] Adolf Sokol a~Anna Bártková.
\item[Knihovníkem:] Herma Kysková.
\item[Hospodářem:] Ant.\ Bárta.
\item[Zástupci do župy:] Fr.\ Pavlíček, Fr.\ Besta ml.
\item[Zástupci do Č.O.S.:] Ant.\ Figalla, Adolf Sokol.
\end{description}

O~valné hromadě přijaty návrhy:
\begin{itemize}
\item Zakládající členové mají platit jednou pro vždy 50~K (dříve 20~K).
\item Vzdělavatelé mají pořádat měsíčně přednášky pro členy.
\item K~letošnímu jubileu, 25tiletému trvání jednoty, mají se konat přípravy.
\end{itemize}

\phantomsection
\subsection{Branná četa}

V~měsíci lednu byla dle nařízení župy mor.\ slezské ustavena branná četa, jejímž velitelem byl zvolen br.\ Ignác Blažej a~jeho zástupcem br.\ Adolf Bárta. Četa čítá 20~členů a~má pušky. Branné čety byly zřízeny u~všech sokolských jednot, aby včas potřeby bránily mladou republiku proti nepřátelům vnitřním i~cizím, dokud by nové vojsko nebylo řádně zorganisováno. Naše četa sešla se několikrát ku cvičení, ale činnost její brzy ustála. Překážkou byla také zima v~tělocvičně, kde se pro nedostatek uhlí nemohlo topit. Též scházely puškám řemeny.

Branné čety byly v~příštích letech zrušeny a~pušky odvedeny vojenskému velitelství v~Opavě.

\phantomsection
\subsection{Zábavy a~činnost}
\begin{itemize}
\item V~únoru pořádán byl sokolský ples, jehož se zúčastnilo přes 100~osob. Příjem ze vstupného a~z~bufetu 1\,104.40~Kč.
\item 16.~března uspořádán humoristický večírek, na němž účinkoval populární ostravský komik Helmid.
\item 23.~března pořádán se zdarem tělocvičný večírek, z~něhož zbylo čistého 166.80~Kč.
\item O~svátcích velikonočních hráno s~úspěchem drama spisovatele Fr.\ Sokola-Tůmy \emph{Gorali}, jež 21.~dubna opakováno v~Klimkovicích.
\item V~červenci konána Husova oslava společně s~DTJ.
\end{itemize}

\phantomsection
\subsection{Sokolské závody a~cvičení}

Sokolských závodů v~červenci zúčastnilo se 41~bratrů, z~nichž 8~dostalo známku A~(velmi schopni), 23~známku B~(méně schopni), 10~známku C~(neschopni).

Br.\ náčelník se 3~bratry zúčastnili se pak měsíčního vojenského cvičení v~Brně.

25.~května konána v~Opavě slavnost \enquote{Střežení stráží svobody}, při níž cvičilo Sokolstvo a~vojsko. Od nás byli tam činní 4~bratři a~4~dorostenky.

10.~srpna konáno veřejné cvičení na zahradě u~sokolovny, jež se nad očekávání zdařilo. S~velikým zájmem přihlíželi četní účastníci na vzorně provedená čísla cvičícího členstva, zesíleného bratry z~polních jednotek. O~zábavní program se staral br.\ Otakar Bárta, jenž se svou plováckou hudbou, sestavenou ze studentů, vytěžil mnoho pro spolkovou pokladnu.

Hrubý příjem z~veřejného cvičení byl přes 7\,000~Kč, čistý výtěžek 4\,090~Kč.

Sestra Alba Elötygrová ušila pro žactvo zdarma 18~úborů.

\phantomsection
\subsection{Další činnost}

Slavnosti \enquote{Vítání Moravců v~Opavě} (to jest obyvatelů Hlučínska nám mírovou smlouvou přiděleného) zúčastnilo se mnoho našich členů v~neděli 31.~srpna.

Exhumace bratra Kotka, zastřeleného v~Mor.\ Ostravě za války rakouskou justicí, zúčastnili se 4~bratři a~2~sestry.

Ženských závodů žup moravských v~Kroměříži zúčastnila se sestra Chlebková a~docílila 95~bodů.

21.~září pořádala odbočka Vřesina za naší účasti veřejné cvičení ve Vřesině, jež se pěkně vydařilo.

Za režie br.\ Otakara Bárty sehráli studující s~některými našimi členy o~prázdninách v~sokolovně divadelní hru \emph{Prašivé švadlenky} s~velkým úspěchem. Hra byla po opakování v~sokolovně sehrána také v~Třebovicích a~Svinově. Větší část výtěžku věnována Družstvu pro udržování sokolovny.

Oslavy výročí naší samostatnosti pořádané župou 28.~října v~Mor.\ Ostravě zúčastnilo se 9~bratrů, z~nich 7~v~kroji, 1~sestra a~7~dorostenců.

7.~prosince sehrána veselohra Stružného \emph{Hanselínova kometa}; 26.~prosince Tůmovy \emph{Pasekáři}, kteří byli v~hostinci Švidrnochově opakováni.

\phantomsection
\subsection{Rozloučení s~bratrem Adolfem Sokolem}

Na rozloučenou s~br.\ Adolfem Sokolem, jenž obdržel místo řídícího učitele v~Polance u~Klimkovic, uspořádán byl členský večírek, na němž cvičili bratři, sestry, dorost i~žáci. Zásluhy br.\ Sokola o~naši jednotu, jejímž členem byl od r.~1896, zastávaje v~ní důležité funkce, ocenil v~delší řeči br.\ Josef Bárta. Br.\ Sokol zřejmě dojat poděkoval za poctu i~za knihu jemu věnovanou a~přál jednotě pro budoucnost všeho zdaru.

Téhož roku rozloučili jsme se též s~br.\ Karlem Klimkem, horlivým cvičencem, jenž byl před válkou i~náčelníkem. Odešel z~továrny zdejší na nové působiště do Kunčic pod Radhoštěm.

Br.\ František Klos byl 10~dní v~nácvičném kursu. Dorostencům objednáno 16~párů cvič.\ kalhot; polovici nákladu hradila jednota. Zakoupeno 50~sokolských zpěvníčků a~objednáno 20~sokolských sborníků.

Jednota zúčastnila se veřejných cvičení jednot v~Opavě, Velké Polomi, Vřesině, Svinově a~Pustkovci.

Jeden bratr a~sestra byli při otevření sokolovny v~Hrušově. Také pohřbu bratra legionáře v~Klimkovicích zúčastnilo se 6~bratrů v~kroji.

Od firmy Thonet v~Bystřici zakoupeno přičiněním br.\ Josefa Bartka 30~kuželů za 1\,550~Kč.

Dělnické tělocvičné jednotě povoleno cvičit v~sokolovně dvakrát v~týdnu za poplatek 1~Kč za každé cvičení.

Družstvo pro udržování sokolovny pořádalo taneční hodiny, které vynesly 203.80~Kč.

Sokolská knihovna čítá 240~svazků.

\phantomsection
\subsection{Legionáři}

Z~našich členů byli ve válce legionáři:

\textbf{Eda Jirák} sloužil v~Itálii. Byl dvakrát raněn. Obdržel italský bronzový a~válečný kříž a~československý válečný kříž.

\textbf{František Šedivý}, učitel, sloužil v~Rusku. Byl telegrafistou u~štábu, později v~Pograničné v~Mandžurii. Měl hodnost desátníka.

\textbf{Josef Hurník}, učitel, byl napřed v~Rusku, později odjel do Francie, kde bojoval v~Alsasku a~v~Argonnách. Také se zúčastnil bojů o~Těšínsko.

Z~bývalých dorostenců byli legionáři:
\begin{enumerate}
\item Emil Widhalm, žel.\ zřízenec, bojoval ve Francii v~Poissonu a~Vouziers.
\item Jan Jelínek, stolařský pomocník, byl ruským legionářem.
\item Václav Besta, čalouník, byl také ruským legionářem.
\end{enumerate}

Br.\ Antonín Hrbáč, dělník, byl těžce zraněn při plebiscitních bojích na Těšínsku.

\clearpage

%%%%%%%%%%%%%%%%%%%%%%%%%%%%%%%%%%%%%%%%%%%%%%%%%%%%%%%%%%%%%%%%%%%%%%%%%%%%%%%
% ROK 1920
%%%%%%%%%%%%%%%%%%%%%%%%%%%%%%%%%%%%%%%%%%%%%%%%%%%%%%%%%%%%%%%%%%%%%%%%%%%%%%%

\phantomsection
\section{Rok 1920}

Valná hromada konána 18.~ledna 1920 o~3~hod.\ odpol. Přítomno 29~členů. Zvoleni byli:

\begin{description}
\item[Starostou:] Adolf Bárta.
\item[Náměstkem:] František Kozub.
\item[Náčelníkem:] František Jirák (25.~února zvolen br.\ František Besta).
\item[Náměstkem:] František Besta (25.~února zvolen br.\ Antonín Klos).
\item[Náčelnicí:] Běta Chlebková.
\item[Místonáčelnicí:] Alba Elötygrová.
\item[Jednatelem:] Josef Bartek.
\item[Pokladníkem:] Antonín Figalla.
\item[Ostatní:] Ignác Blažej, Alois Besta, Josef Bárta, Antonín Bárta, Antonín Klos ml., Jan Kopřiva.
\item[Náhradníky:] Ludvík Kočí, Adolf Valder, Emil Mika, František Krejčí.
\item[Vzdělavatelé:] Antonín Figalla, Josef Hurník (učitelé).
\item[Režisér:] Alois Petras.
\item[Knihovnicí:] Herma Lysková.
\item[Vyslancem do župy:] František Jirák.
\item[Vyslancem do Č.O.S.:] Antonín Figalla.
\item[Hospodářem:] Antonín Bárta.
\item[Revisory účtů:] Teofil Pivoň, František Kudela.
\end{description}

Na valné hromadě byly přijaty návrhy:
\begin{enumerate}
\item aby byl zřízen cestovní fond k~VII.~sletu do Prahy (br.\ Bartek),
\item aby byl zřízen fond na zakoupení praporu (br.\ Karas),
\item aby byly opatřeny kroje pro členstvo i~dorost, příp.\ na splátky (br.\ Klos),
\item aby si bratři tykali v~místnosti i~veřejnosti, a~vykání by se trestalo pokutou ve prospěch cestovního fondu (br.\ Ignác Blažej).
\end{enumerate}

Jednota v~tomto roce vyvinula nadprůměrnou činnost. Mimo řádné valné hromady konala se 25.~února 1920 mimořádná valná hromada, která po resignaci zvoleného náčelníka br.\ Fr.\ Jiráka zvolila náčelníkem br.\ Františka Bestu a~jeho náměstkem br.\ Antonína Klosa.

Župa provedla nové rozdělení jednot dle okrsků. Náš okrsek obsahuje jednoty: Svinov, Polanka, Klimkovice, Poruba, Krásné Pole a~Plesná s~příslušnými odbočkami. Č.O.S.\ vzala na vědomí zřízení odbočky ve Vřesině.

Poněvadž byla v~obci zřízena osvětová komise, která má pečovati o~vzdělání lidu četbou a~přednáškami, usneseno darovati jí sokolskou knihovnu až na odborné a~divadelní knihy.

V~roce konáno 14~výborových a~7~členských schůzí.

Zakoupeny byly 2~žíněnky, pořízeny kruhy na letním cvičišti a~opravena bradla.

\phantomsection
\subsection{Přednášky}
\begin{enumerate}
\item Božena Němcová (br.\ Josef Hurník)
\item Bílá Hora (br.\ Josef Hurník)
\item Národní přísaha a~sokolský slib
\item Tyrš a~jeho zásady
\item T.~G.~Masaryk jako motiv náboženský a~sociální (br.\ Antonín Figalla)
\item J.~A.~Komenský (br.\ učitel Josef Vavrečka)
\end{enumerate}
Proslovů před šikem bylo 23 (br.\ Josef Hurník). U~hranice Husovy vyslechli členové proslov československého faráře Františka Šibora z~Radvanic.

\phantomsection
\subsection{Divadla}
\begin{enumerate}
\item \emph{Hnízdo v~bouři}
\item \emph{Socialisace}
\item \emph{Její pastorkyně}
\end{enumerate}

Do okolních lesů podniknuto 5~vycházek se zpěvem.

\phantomsection
\subsection{Veřejné cvičení}

Veřejné cvičení bylo 11.~července. Ačkoli je částečně kazil déšť, který zavinil menší účast lidí, přece dopadlo celkem dobře po stránce morální i~hmotné.

Zábavní část svěřena br.\ Otakaru Bártovi, který svůj úkol provedl k~úplné spokojenosti. Velikou veselost způsobilo rozdávání párků pod zvonem. Muzeum sestavil br.\ Josef Hurník a~rybolov na pstruhy br.\ Karas. V~Kavárně \enquote{Jokohama}, upravené pod kuželnou, hostily sestry Miloslava Hurníková a~Herma Lysková. Při cvičení účinkovalo 18~hudebníků po 50~Kč. Vstupné: občan 5~Kč, občanka 3~Kč.

\noindent Hrubý příjem činil \hfill 6\,899~Kč 70~h\\
Vydání \hfill 3\,237~Kč 10~h\\
Zisk \hfill \textbf{3\,662~Kč 60~h}

Zisku bylo použito na splacení dluhu v~Občanské záložně v~Porubě.

\phantomsection
\subsection{VII.~slet sokolský v~Praze}

Sokolský slet byl velikou událostí v~životě Sokolstva celé republiky. Vzpružil novou činnost a~nové nadšení ve všech jednotách. I~u~nás se jevil velký zájem pro slet a~konány přípravy k~důstojné representaci. Účast na sletu byla značná; přes polovici členstva jelo do Prahy, aby vidělo vzkříšení a~nový rozmach Sokolstva, tolik pronásledovaného za světové války, a~aby uvidělo svoboditele národa, svého milovaného prezidenta bratra T.~G.~Masaryka.

9~bratrů jelo v~kroji, 23~v~občanském obleku a~jeden dorostenec. Sester jelo 9 a~4~dorostenky. Pisatel těchto pamětí, jenž jel o~2~dni později, byl přítomen poslednímu cvičení na Letenské pláni a~odnesl si z~něho nezapomenutelný dojem. Byl to obraz nevylíčitelné krásy, kdy za zvuku mistrné hudby zaplnilo rozsáhlé prostranství tisíce bratrů a~sester, by podali ukázky své vyspělosti a~sokolské kázni.

Obrovský potlesk nadšených tisíců účastníků po každém cvičení byl zaslouženou odměnou všem účinkujícím, kteří nedbajíce únavy a~velikého parna, po 3~dni stáli pevně na svých místech, aby ukázali celému světu sílu a~krásu myšlenky Tyršovy, myšlenky sokolské.

\phantomsection
\subsection{Přístavba sokolovny}

Již po několik let se pociťovala nedostatečnost dosavadních místností sokolovny, která zvláště při větších schůzích, divadlech a~plesech se ukázala malou. Proto se stále uvažovalo o~přístavbě, avšak pro slabý finanční stav (dluh na sokolovně obnášel ještě 11\,000~Kč) se provedení stále oddalovalo.

Konečně po mnohých úvahách a~úsilí některých bratrů, zvláště Josefa Bárty, který na půjčkách a~darech od bratrů i~přátel získal 36\,000~Kč, rozhodlo se Družstvo pro udržování sokolovny pro rozšíření sokolovny. Ve schůzi konané 20.~března 1920 sestaven stavební odbor, v~němž byli bratři: Ignác Blažej, Adolf Bárta, Alois Besta, Josef Bárta, Antonín Figalla a~František Jirák. Rozpočet činil 52\,000~Kč, avšak pro neustálé zdražování všeho staviva byl překročen.

Přístavbou a~adaptací byly provedeny tyto změny: Velký sál byl prodloužen tím, že se odstraněním dělící zdi připojily k~němu 2~menší místnosti, sloužící dříve za sborovnu a~knihovnu. Vedle velikého sálu přistavěn malý sál a~vedle jeviště šatna se záchody. Také upravena a~rozšířena byla místnost pod jevištěm a~upraveny sklepy. Též zahrada u~sokolovny mezi stromky byla vyrovnána a~pískem vysypána. Stará kuželna, již hodně chatrná, byla odstraněna.

Tyto práce byly provedeny a~ukončeny v~roce 1921.

\noindent Novostavba malého sálu a~šatny stála \hfill 57\,994.18~Kč\\
Oprava starých budov \hfill 15\,891.11~Kč\\
Úprava zahrady \hfill 995.---~Kč\\
\textbf{Úhrnem} \hfill \textbf{74\,880.29~Kč}

\phantomsection
\subsection{Zveřejnění a~rozšíření hostinské koncese}

Dosud byla v~sokolovně omezená hostinská koncese. Nápoje, pivo a~víno, dostávali jenom členové Družstva, pouze při větších podnicích: divadlech a~výletech smělo se veřejně čepovat. Poněvadž tato koncese, povolená na 10~let, vypršela koncem r.~1919, staralo se předsednictvo Družstva pro udržování sokolovny o~obnovení hostinské koncese, její zveřejnění a~rozšíření.

Podanou žádost však okresní hejtmanství v~Bílovci nevyřídilo dle našeho přání a~povolilo nám koncesi omezenou jako dosud a~jen na 6~let, to jest do r.~1926. S~tím se Družstvo nespokojilo a~podalo na zemskou vládu v~Opavě stížnost a~žádost o~úplnou veřejnou koncesi hostinskou, která byla Družstvu udělena výnosem ze dne 9.8.1920, III-646 na dobu 20~let, to jest do konce roku 1939.

Proti povolené koncesi podalo Společenstvo hostinských v~Klimkovicích stížnost na ministerstvo obchodu a~zároveň podalo žádost České Obci sokolské v~Praze, by tato zakročila přímo svým vlivem, aby rekurs Společenstva hostinských byl ve prospěch Společenstva příznivě vyřízen.

Ministerstvo obchodu zamítlo však rekurs podaný Společenstvem hostinských v~Klimkovicích a~potvrdilo výnos slezské zemské vlády rozhodnutím ze dne 27.~pros.\ 1920, č.~48-700.

Poněvadž v~žádosti Společenstva hostinských zaslané České Obci sokolské bylo plno nepravdivých údajů a~urážek předáků sokolských, proto poslali zástupci Sokola Společenstvu hostinských výzvu, aby nepravdivé údaje a~urážky odvolalo, neučiní-li pak, že bude podána proti němu žaloba. Poněvadž výzva neměla účinku, podalo 12~členů výboru Sokola žalobu u~okresního soudu v~Klimkovicích na Společenstvo hostinských v~Klimkovicích.

Poněvadž bylo dvojí stání, mnoho svědků a~několik právních zástupců, přišel proces velice draho Společenstvu hostinských, které bylo odsouzeno k~pokutě a~ku placení všech soudních útrat, jež šly do tisíců. Trpce na tento spor bude vzpomínati zdejší hostinský Alois Klos, který jako náměstek předsedy Společenstva jej vyvolal.

Že byla na sokolské straně veliká radost po povolené koncesi i~vyhraného sporu, netřeba připomínat.

\phantomsection
\subsection{Nájemci hostinské koncese}
\begin{enumerate}
\item Antonín Bárta, jenž provozoval v~sousedním domě živnost perníkářskou. Platil 3~Kč z~1~hl vyčepovaného piva. Nemoha živnost hostinskou řádně zastávat, vzal si za zástupce Františka Míku.
\item Od 1.~července 1914 do 31.12.1916 byla hostinskou Štěpánka Lazarová, choť Josefa Lazara. Platila ročně 900~Kč.
\item Od 1.~ledna do 15.~prosince 1917 byl nájemcem restaurace Cyrill Dostál, zdejší obchodník smíšeným zbožím. Obchod v~sokolovně šel bídně, nájemné mu bylo sníženo na 400~Kč.
\item Od 15.~prosince 1917 do 31.~března 1920 převzal restauraci Karel Lisník, úředník vítkovických železáren.
\item Od 1.~dubna do 31.~srpna 1920 byl nájemcem restaurace Leopold Mojžíšek, hostinský z~Místku, který nabídl roční nájem 2\,600~Kč. Poněvadž obdržel nádražní restauraci v~Rohatci, vzdal se hostince u~nás.
\item Od 1.~září t.r.\ je hostinskou zase paní Štěpánka Lazarová.
\end{enumerate}

\phantomsection
\subsection{Jiné události}

Taneční vínek pořádaný 8.~ledna vynesl čistého 544~Kč.

Pěkný byl večírek pořádaný 7.~března na oslavě narozenin p.~prezidenta Masaryka. Po proslovu br.\ Figally následovalo velmi zdařilé cvičení žaček, žáků a~dorostu. Dorostenky zacvičily tance na národní písně. Mezi jednotlivými oddíly byly recitace, na konec skupiny a~živý obraz. Čistý zisk z~večírku 170~Kč byl odveden cestovnímu fondu do Prahy.

Pro černé neštovice sebráno u~členů 50~Kč.

Klubu čsl.\ sociální mládeže propůjčena ke schůzím místnost vedle nálevny.

Br.\ starosta Adolf Bárta přenechal jednotě ku cvičení část zahrady (louky), přiléhající k~sokolovně, kteráž část byla ohraničena plotem.

\bigskip

\noindent Počet členstva:\\
Mužů 55, sester 13 = 68.\\
Dorostenců 18, dorostenek 9 = 27.\\
Žáků 24, žaček 13 = 37.

\clearpage

%%%%%%%%%%%%%%%%%%%%%%%%%%%%%%%%%%%%%%%%%%%%%%%%%%%%%%%%%%%%%%%%%%%%%%%%%%%%%%%
% ROK 1921
%%%%%%%%%%%%%%%%%%%%%%%%%%%%%%%%%%%%%%%%%%%%%%%%%%%%%%%%%%%%%%%%%%%%%%%%%%%%%%%

\phantomsection
\section{Rok 1921}

Valná hromada konala se 16.~ledna za účasti 41~členů. Zvoleni byli:

\begin{description}
\item[Starostou:] Adolf Bárta, náměstkem: František Kozub.
\item[Náčelníkem:] František Jirák, náčelnicí: Vlasta Blažejová.
\item[Jednatelem:] František Šedivý.
\item[Pokladníkem:] Josef Vavrečka, učitel (po jeho odchodu do Perného: Adolf Valder).
\item[Ostatní členové výboru:] Ignác Blažej, Josef Bárta, František Besta, Oldřich Studenský, Alba Bestová (dříve Elötygrová).
\item[Náhradníci:] Adolf Valder, František Pospíšil, Donát Grünbaum st., Emil Mika.
\item[Revisoři účtů:] Eda Jirák a~Jan Kopřiva.
\item[Vzdělavatel:] Josef Hurník.
\item[Hospodář:] Antonín Figalla.
\item[Knihovnicí:] Herma Lysková.
\item[Vyslancem do Č.O.S.:] Ignác Blažej.
\item[Vyslancem do župy:] František Jirák.
\end{description}

Přijaty návrhy: Členské příspěvky buďtež zvýšeny na 24~Kč ročně. Dorost měsíčně 1~Kč, žactvo 50~h nepovinně. Noví členové budou přijati po šestiměsíční zkoušce.

V~tomto roku byla ukončena přístavba a~přestavba sokolovny.

\phantomsection
\subsection{Osobní události}

Z~jednoty vystoupila čilá sokolská pracovnice sestra Běta Chlebková, účetní zdejší firmy Ignác Blažej a~spol., jež přijala místo v~Radvanicích. Odchodem jejím utrpěla jednota citelnou ztrátu. Do Mariánských Hor se vrátil br.\ Rosenzweig, řezník, a~k~Vřesinské železniční jednotě přešel Oldřich Studenský.

Ve vřesinské pobočce zemřel br.\ Martiník v~mladém věku.

V~sousední jednotě plesenské zemřel náhle 8.~listopadu br.\ Alois Řiháček, tamější učitel. Byl zakladatelem jednoty v~Plesné a~jejím vůdcem. Pro svou kulturní činnost byl zvolen okrskovým vzdělavatelem. Veliká účast sokolských jednot z~celého okolí i~z~naší jednoty svědčila o~tom, jak byl br.\ Řiháček oblíben a~jak velikých zásluh si získal o~sokolství.

\phantomsection
\subsection{Činnost tělocvičná}

Činnost tělocvičná byla uspokojivá. Cvičitelský sbor měl 5~členů. Cvičil 32~hodin, průměrně 4.

\begin{itemize}
\item Mužů cvičilo 10, cvičili 85~hodin, průměrně 8.
\item Mužský dorost čítal 15~členů, cvičil 91~hodin, průměrně 10.
\item Žáků cvičilo 31, cvičili 97~hodin, průměrně 19.
\item Žen cvičilo 7 v~69~cvičebních hodinách.
\item Dorostenek 6 v~90~hodinách.
\item Žákyň 33 ve 100~hodinách.
\end{itemize}

Dva bratři absolvovali župní kurs cvičitelský, jeden se podrobil župní cvičitelské zkoušce. Ženy necvičily v~srpnu a~září pro nemoc sestry náčelnice.

V~březnu byla tělocvičná beseda, na níž cvičily dorostenky a~žáci. Besedou oslaveny byly narozeniny p.~prezidenta.

V~dubnu byla vycházka 25~žáků do lesa.

8.~května pořádala se členská jarní beseda.

7.~srpna pořádáno okrskové cvičení, které bylo velmi zdařilé. Hrubý příjem činil 8\,000~Kč, čistý výnos přes 3\,000~Kč.

\phantomsection
\subsection{Župní slet a~zájezd na Slovensko}

Župního sletu v~Opavě zúčastnilo se 11~bratrů v~kroji, 6~bratrů cvičilo prostná, 15~hry. Dorostenek 7, z~nich obdržely diplomy: Vlasta Blažejová, Slávka Lazarová a~Anička Grünbaumová.

Zájezdu na Slovensko zúčastnili se 4~bratři: Emil Mika, Václav Jirák, Rudolf Sladký, Břetislav Blažej. Jednota jim dala podporu 100~Kč.

Nejpilnějšími cvičenci v~tom roku byli Břetislav Zdražila a~Miloslav Kozub.

Zakoupeny byly: \emph{Cvičební večery pro dorost}, \emph{Rozvrh cvičební pro dorost}, \emph{Rejová prostná v~kruhu}, \emph{Rytmická cvičení}.

Knihy \emph{Masaryk osvoboditel} objednáno 10~výtisků, z~nich 3~darovány nejlepším cvičencům.

Garanční fond 600~Kč, upsaný na všesokolský slet, darován na stavbu Tyršova domu v~Praze.

\phantomsection
\subsection{Činnost vzdělávací}

Přednášek bylo 9:
\begin{enumerate}
\item V~lednu: Palacký a~Denis (br.\ Josef Hurník)
\item V~lednu: O~Tyršovi --- dorostencům (br.\ Josef Hurník)
\item V~únoru: Člověk a~vesmír (br.\ Josef Hurník)
\item V~dubnu: Od Bajkalu do Čech (br.\ Josef Vavrečka)
\item V~červnu: 21.~červen 1621 (br.\ Josef Hurník)
\item V~srpnu: O~Jiráskovi (br.\ Václav Klos)
\item V~říjnu: Můj názor na sokolství (br.\ Leo Grünbaum)
\item V~listopadu: Vlastnosti dobrého Sokola (br.\ Josef Hurník)
\item V~listopadu: K.~H.~Borovský (br.\ Josef Bárta)
\end{enumerate}

Ženám a~dorostu přednášely sestry:
\begin{itemize}
\item Jarmila Sokolová: \enquote{O~vzdělání}
\item Anna Grünbaumová: \enquote{Názor Masaryka na ženu}
\end{itemize}

V~červenci zúčastnila se jednota jako obyčejně Husovy oslavy na pozemku u~Vřesiny. Proslovů před šikem pronesli br.\ vzdělavatel a~náčelník celkem 38.

\phantomsection
\subsection{Divadla}
\begin{itemize}
\item V~březnu za režie br.\ Františka Šedivého sehrána veselohra \emph{Poslední muž}.
\item V~květnu za režie br.\ Josefa Martiníka \emph{Pygmalion}.
\item V~červnu za režie br.\ Hochmanna z~Klimkovic sehrána v~přírodě v~Záhumenici \enquote{u~lípy} \emph{Maryša} bratří Mrštíků. Účinkovali členové jednoty porubské, klimkovské a~vřesinské. K~velikému úspěchu hry vydatně přispěl známý ostravský ochotník Ladislav Jirotka. Hra byla v~červenci opakována. Ku provedení této působivé hry, jež byla vzorně vypravena a~co nejdokonaleji provedena (hlavní úlohu hrála pí Anna Hochmannová z~Klimkovic), sešlo se veliké množství obecenstva z~celého okolí, jež bylo hrou nadšeno. Pro obecenstvo připravily všechny 3~jednoty bufety. Náš bufet vynesl čistého 701~Kč.
\item V~srpnu sehráno Jiráskovo drama \emph{Otec}.
\item V~prosinci za režie br.\ L.~Grünbauma sehrána pohádka Stružného \emph{Slečna prokuristka}.
\end{itemize}

Na konci roku uspořádána pěkná silvestrovská zábava, jež vynesla 402.76~Kč.

Časopisy odbírány: \emph{Věstník ČOS} (13~výt.), \emph{Sokol} (2~výt.), \emph{Sokolice} (1~výt.), \emph{Sokolské besedy} (14~výt.), \emph{Skřivánek} (5~výt.), \emph{Župní věstník} (23~výt.), \emph{Nový Lid} (27~výt.).

Loutkové divadlo (sádrové figurky) koupeno od jednoty Martinov za 500~Kč.

Ministerstvo sociální péče udělilo nám na naši žádost státní subvenci 4\,500~Kč, jimiž splacen dluh v~Občanské záložně v~Klimkovicích.

Br.\ Vladimír Košťřica, učitel a~člen Sokola ve Svinově, přistěhoval se do Poruby, stal se členem zdejší jednoty a~uplatnil se hlavně v~Družstvu pro udržování sokolovny tím, že zavedl v~něm řádné účetnictví a~sestavil bilanci.

\bigskip

\noindent Zpráva pokladní:\\
Příjem za rok 1921 činil 3\,268.20~Kč\\
Vydání 3\,287.01~Kč

Podotýká se, že jednota Sokol účtuje jenom členské příspěvky, příjmy za odznaky, legitimace, pohlednice, dary a~ze členských večírků. Ostatní účetnictví, jež vyžaduje udržování sokolského majetku, vede Družstvo pro udržování sokolovny.

\bigskip

\noindent Stav členstva na konci roku:\\
Mužů 64, žen 17 --- úhrnem \textbf{81}.

\clearpage

%%%%%%%%%%%%%%%%%%%%%%%%%%%%%%%%%%%%%%%%%%%%%%%%%%%%%%%%%%%%%%%%%%%%%%%%%%%%%%%
% ROK 1922
%%%%%%%%%%%%%%%%%%%%%%%%%%%%%%%%%%%%%%%%%%%%%%%%%%%%%%%%%%%%%%%%%%%%%%%%%%%%%%%

\phantomsection
\section{Rok 1922}

Valná hromada konána 14.~ledna za účasti 46~členů. Zvoleni byli:

\begin{description}
\item[Starostou:] Adolf Bárta.
\item[Náměstkem:] František Kozub.
\item[Náčelníkem:] František Klos.
\item[Náčelnicí:] Božena Blažejová.
\item[Jednatelem:] František Jirák.
\item[Pokladníkem:] Donát Grünbaum.
\item[Vzdělavatelem:] Josef Hurník.
\item[Další členové výboru:] František Šedivý (matrikářem), František Besta (hospodářem), Vladimír Košťřica, Ignác Blažej, Adolf Valder, Antonín Klos.
\item[Náhradníky:] Antonín Hrbáč (zvolen také zapisovatelem), Alois Besta, Miloslava Hurníková, Eduard Jirák.
\item[Revisoři účtů:] Jindřich Legerský, Alois Petras.
\item[Režisérem:] Leoš Grünbaum (také knihovníkem).
\item[Vyslanci do župy:] František Šedivý.
\item[Vyslanci do Č.O.S.:] Ignác Blažej.
\end{description}

Režie loutkového divadla svěřena Františku Šedivému.

\phantomsection
\subsection{Poděkování Josefu Bártovi}

Člen jednoty Josef Bárta obdržel 16.~února 1922 od výboru Tělocvičné jednoty Sokol v~Porubě přípis tohoto znění:

\begin{quote}
\textit{Bratru Josefu Bártovi.}

Při poslední schůzi výboru jednoty bylo s~překvapením konstatováno, že nenacházíš se mezi nově zvolenými členy výboru. Nepřítomnost Tvá, Tebe, který jsi po celou dobu trvání jednoty naší stál na předním místě a~vždy úsilovně pracoval pro blaho a~rozkvět této, nelekaje se ani nejhorších překážek a~úskoků ze strany odpůrců našich, vytrvav v~práci nevděčné a~často málo doceněné, byla trpce pohřešována a~pátráno po příčině Tvého pominutí.

Výbor přesvědčil se, že nebylo zde osobního odporu proti Tobě, ale z~uznání Tvého podlomeného zdraví a~přesílení v~práci nevolila Tebe část členstva, chtějíc Ti dopřát aspoň jednoročního odpočinku, bys s~nově nabytými silami mohl opět pracovat ku zdaru jednoty, zvláště když bylo vzato také v~úvahu, že převzetím funkce pokladníka v~Družstvu pro udržování sokolovny přibylo Ti nové zodpovědné práce.

Výbor pokládá za svou povinnost vzdáti Tobě, milý bratře, upřímný sokolský dík za veškerou Tvou úspěšnou činnost a~těší se, že aspoň radou budeš vždy nápomocen v~ušlechtilém díle a~po roce opět ochotně propůjčíš síly své k~práci nám společné dle zásady Tyršovy, kterou jsi se vždy řídil: \enquote{Ni zisk, ni slávu,} a~přeje Ti plného zotavení po přestálých trampotách.

Na zdar!

\hfill Za Tělocvičnou jednotu Sokol v~Porubě --- František Jirák v.r., jednatel.
\end{quote}

\phantomsection
\subsection{Úmrtí bratra Antonína Hrbáče}

Dne 6.~října zemřel sympatický člen naší jednoty br.\ Antonín Hrbáč v~mladém věku 24~let na plicní neduh, který u~něho vypuknul následkem zranění při plebiscitních bojích na Těšínsku. Br.\ Hrbáč byl nadaným mladíkem, jenž se v~jednotě uplatnil jako zapisovatel při výborových schůzích a~při zábavách jako humorista. Četná účast na pohřbu svědčila o~jeho oblibě.

Na výlohy pohřební sebrali členové mezi sebou 400~Kč, které odevzdali jeho chudým rodičům.

\phantomsection
\subsection{Tělocvičná činnost}

Tělocvičná činnost byla velmi uspokojivá. Župní slet v~Mor.\ Ostravě vyžádal si mnoho práce a~příprav, se kterými počato již v~březnu.

Rozřazovacích závodů v~šestiboji zúčastnilo se 15~bratrů a~6~dorostenců. Vyzval br.\ František Jirák.

Při okrskových rozřazovacích na prostná (21.5.) prošlo 11~mužů a~2~dorostenci.

V~jednotě konalo se 4.~června veřejné cvičení; 16~bratrů cvičilo prostná, 2~družstva nářadí (kůň, bradla). 6~dorostenců cvičilo kruhy, 8~živé pyramidy. 21~žáků pálkovaná, 17~hru \enquote{Malý cvičitel}, 24~různé hry.

Na župním sletu v~Mor.\ Ostravě 2.~července bylo 9~bratrů v~kroji, 25~v~občanském obleku. 8~bratrů cvičilo prostná, 1~na bradlech, 1~v~šestiboji (František Jirák).

Zájezdu na Těšínsko (29.6.) zúčastnili se 3~bratři, kteří cvičili prostná. V~Karviné byli 2~bratři v~kroji, 1~v~občanském.

Dne 19.~listopadu konala se v~sokolovně tělocvičná akademie. Muži cvičili hrazdu a~rej rohovnický, žáci skupiny na žebřinách a~hry.

Župních závodů 17.~prosince zúčastnilo se 6~bratrů (Eda Jirák, František Jirák, Antonín Klos, František Besta, Břetislav Zdražila, Václav Endlicher), kteří docílili 75\,\% dosažitelných bodů.

Pilně cvičily v~roce ženy, dorostenky a~žákyně, které se také zúčastnily veřejných produkcí.
\begin{itemize}
\item Na veřejném cvičení v~Porubě cvičilo 20~žákyň.
\item Na krajském sletu v~Mor.\ Ostravě 16~žákyň, 3~dorostenky závodily, 3~cvičily prostná, 5~žen cvičilo prostná, 1~závodila.
\item Na akademii 19.~listopadu cvičilo 28~žákyň, 4~dorostenky a~4~ženy.
\end{itemize}

\phantomsection
\subsection{Vzdělávací činnost}

Vzdělávací činnost řízená vzdělavatelem br.\ Josefem Hurníkem přinesla hojné ovoce. Proslovů před šikem bylo 46, přednášek 10.

Přednášeli:
\begin{itemize}
\item Br.\ Josef Hurník: Žena v~životě občanském a~sokolském; Kritika a~sebekritika; Lidová výchova a~její poměr k~sokolstvu; Život a~dílo Fügnerovo; Nové úkoly sokolstva.
\item Br.\ František Šedivý: Dějiny sokolstva.
\item Br.\ František Jirák: O~sokolské organizaci; Sokolstvo v~praktickém životě; Proč cvičíme.
\item Br.\ František Kočí: Jak prospěti vlasti a~národu.
\end{itemize}

Výročí 75.\ narozenin spoluzakladatele sokolstva Jindřicha Fügnera bylo vzpomenuto slavnostní mimořádnou členskou schůzí s~přednáškou vzdělavatele. Ostatní program vyplněn recitacemi, zpěvními a~hudebními čísly.

Toho roku konány také ideové závody mužů, žen i~dorostu. Pro každý odbor sestaveno 20~otázek.

Výročí Husovo oslaveno u~hranice na \enquote{Kočí Hůrce} u~Vřesiny přednáškou br.\ Vladimíra Košťřici.

\phantomsection
\subsection{Divadla}
\begin{itemize}
\item 15.1.\ \emph{Na faře}
\item 13.2.\ \emph{Láska divy tvoří}
\item 16.4.\ \emph{Lucerna} (Alois Jirásek)
\item 27.8.\ \emph{Kulatý svět} (Šamberk)
\item 28.10.\ \emph{Staříček Holuša} (Sokol-Tůma)
\item 8.12.\ \emph{Macíkova dovolená}
\item 26.12.\ \emph{Ženská vojna}, opereta Ed.\ Marhuly. Operetu nacvičil Josef Bárta.
\end{itemize}

Výtěžek z~divadel činil 3\,880~Kč.

K~výpravě jeviště namalováním nových dekorací velmi přispěl br.\ Arnošt Kozák, kreslíč u~firmy Ignác Blažej, který všechny práce vykonal bez odměny. Také vypravil jeviště pro loutkové divadlo.

\bigskip

\noindent Zpráva pokladní vykazuje: Příjmu 4\,292~Kč, vydání 3\,864.88~Kč, zbytek 427.12~Kč.

\bigskip

\noindent Stav členstva k~31.12.1922: Mužů 76, žen 20.

\clearpage

%%%%%%%%%%%%%%%%%%%%%%%%%%%%%%%%%%%%%%%%%%%%%%%%%%%%%%%%%%%%%%%%%%%%%%%%%%%%%%%
% POMĚRY PŘED ZALOŽENÍM SOKOLA
%%%%%%%%%%%%%%%%%%%%%%%%%%%%%%%%%%%%%%%%%%%%%%%%%%%%%%%%%%%%%%%%%%%%%%%%%%%%%%%

\phantomsection
\section{Rok 1923}

\begin{description}
\item[Jednatelem:] Vladimír Košťřica.
\item[Pokladníkem:] Donát Grünbaum ml.\ (účetní).
\item[Vzdělavatelem:] Antonín Figalla.
\item[Knihovníkem a~archivářem:] Josef Hurník.
\item[Kronikářem:] Josef Bárta.
\item[Matrikářem:] František Šedivý.
\item[Přísedící:] Ignác Blažej, Leoš Grünbaum, Rudolf Sladký.
\item[Náhradníky:] Oldřiška Kozubová, Emil Mika, Arnošt Kozák, Alois Petras.
\item[Hospodářskými dozorci:] Otakar Bárta, Donát Grünbaum st., Adolf Valder.
\item[Do rozhodčího soudu:] Vladimír Košťřica, Antonín Bárta, Josef Bárta, Ignác Blažej.
\item[Vyslanci do župy a~do Č.O.S.:] František Šedivý a~Ignác Blažej.
\end{description}

\phantomsection
\subsection{Tělocvičná činnost}

Činnost tělocvičná poklesla; náčelník marně se namáhal dostat do cvičení mladíky alespoň do 26~let, jak to je přáním ČOS. Jen několik jich vytrvalo a~docílilo úspěchu. Ženský sbor se rozešel, cvičily jen dorostenky a~žačky. Za ochuravělou sestru Vlastu Blažejovou vedla žačky sestra Sláva Hurníková.

Dorostence vedl br.\ František Jirák a~žactvo br.\ Richard Grünbaum.

V~tom roku byla u~všech cvičících provedena lékařská prohlídka.

Pro malý počet cvičenců nebylo pořádáno ani veřejné cvičení ani akademie, čímž způsobena jednotě finanční ztráta.

\phantomsection
\subsection{Vítězství na II.~mezisletových závodech v~Praze}

Radostnou událostí v~tomto roce je vítězství cvičícího družstva na Druhých mezisletových závodech v~Praze ve dnech 6.--8.~července. V~družstvu byli bratři: Eda Jirák, Donát Grünbaum, František Jirák, František Besta, Antonín Klos a~Břetislav Zdražila.

Toto družstvo zúčastnilo se již 29.4.\ okrskového rozřazovacího závodu v~Porubě, 13.5.\ župního rozřazovacího závodu v~Příboře a~27.5.\ krajského závodu v~Žilině na Slovensku, kde všude docílilo dobrých výsledků.

Čestně obstálo družstvo v~matičce Praze, získavši malý diplom a~lipový věnec družstva jako II.~cenu v~mezisletových závodech ČOS. Ze 178~družstev se umístilo na 104.~místě a~dosáhlo 475.62~bodů, což je 83.44\,\% v~nižším oddíle.

Jednotlivě závodili tři bratři mezi 1\,340~závodníky:
\begin{itemize}
\item Donát Grünbaum --- 54.~místo
\item Eda Jirák --- 73.~místo
\item Antonín Klos --- 77.~místo
\end{itemize}
Každý z~nich obdržel velký diplom.

Br.\ František Jirák dosáhl z~90~bodů 75.75, br.\ František Besta z~90~bodů 72.50, br.\ Břetislav Zdražila z~90~bodů 62.50.

Všem závodníkům, kteří tak čestně reprezentovali naši jednotu v~Praze, poděkoval a~vzletnými slovy ocenil jednatel br.\ Antonín Figalla jejich vytrvalou, obětavou a~námahavou práci na členské schůzi 27.~října.

25.~července zúčastnila se naše jednota veřejného cvičení v~Pustkovci.

Župní kurs her navštěvoval od 31.8.--2.9.\ bratr Richard Grünbaum.

Dorostenky (počtem 12) za vedení sestry Slávky Lazarové cvičily 73~hodin. Proslovů bylo 7. Žákyně (16) měly 90~cvičebních hodin a~2~vycházky.

\phantomsection
\subsection{Vzdělávací činnost}

Vzdělávací činnost byla uspokojivá. Přednášek bylo 9, proslovů k~žákům a~dorostu 29.

Přednášeli:
\begin{itemize}
\item Br.\ Vladimír Košťřica: O~nových sokolských stanovách
\item Br.\ Antonín Figalla: Sociologie prezidenta Masaryka; Historie sokolských závodů
\item Br.\ František Šedivý: Volný duch --- volná myšlenka
\item Br.\ Jan Krtička (Svinov): Vliv náboženství na národ
\item Br.\ Otakar Šindler (Třebovice): Umění a~Miroslav Tyrš
\item Br.\ Josef Bárta (2~přednášky): Historie naší jednoty za prvních 20~let
\end{itemize}

U~Husovy hranice vyslechli členové přednášku o~Mistru Janu Husovi, přednesenou hostujícím řečníkem.

Proslovy k~žactvu a~dorostu pronesli bratři: učitel Josef Hurník 12, učitel Antonín Figalla 10, učitel František Šedivý 7.

\phantomsection
\subsection{Divadla}
\begin{enumerate}
\item Opereta \emph{Ženská vojna} opakována 5.~ledna.
\item Veselohra \emph{Potas a~Perlmutter} za režie br.\ Leoše Grünbauma, sehrána 2.~dubna.
\item Drama \emph{Gejzír} sehrál 28.5.\ pohostinsky Sokol Moravská Ostrava III.
\item Veselohra \emph{Ořechy} sehrána za režie br.\ Otakara Bárty.
\item 15.~července sehrána opereta \emph{Ženská vojna} v~přírodě v~Záhumenici \enquote{u~lípy}, na témž místě, kde byla hrána Maryša. Zpěvy a~hudbu nacvičil kapelník Chamrád z~Klimkovic. Účastníků asi 1\,000. Morální úspěch pěkný, finančně slabší. Výtěžek z~bufetu činil 754~Kč.
\item 16.~prosince sehráno drama \emph{Navždy} za režie br.\ Figally.
\item 28.~prosince sehrána za režie sestry Miloslavy Hurníkové veselohra \emph{Vilouškův svatební den}, která se velmi líbila a~přinesla pokladně čistého 645~Kč.
\end{enumerate}

Agrární dorost, většinou členové Sokola, sehrál Stolbovu veselohru \emph{Vodní družstvo}, z~níž věnoval Sokolu 100~Kč.

Čistý výnos z~veselohry \emph{Ořechy} věnován místnímu Spolku pro ošetřování nemocných.

Ideové školy ve Svinově zúčastnilo se 5~bratrů.

\phantomsection
\subsection{Osobní události}

Koncem března odešel z~Poruby br.\ Arnošt Kozák, úředník fy Ig.\ Blažej, jenž jako dovedný malíř a~konstruktér prokázal jednotě cenné služby při pořádání divadel, zvláště v~přírodě. Pro loutkové divadlo sestavil jeviště a~namaloval dekorace úplně zdarma. Také účinkoval jako tenorista při pěveckých produkcích. Jednota se s~ním rozloučila na členské schůzi 24.~března a~darovala mu na památku román Sokola-Tůmy \emph{Na Kresách}.

V~září odešel z~Poruby do Opavy br.\ učitel Vladimír Košťřica, dobrý pracovník ve finančním hospodářství Družstva pro udržování sokolského domu a~velmi milý a~zasloužilý člen naší jednoty, zvláště při prvém pobytu v~Porubě v~letech 1899--1904.

29.~září slavili sňatek br.\ Leo Grünbaum se sestrou Boženkou Blažejovou. Sezdával je československý duchovní Ignác Olšanský ve vyzdobené třídě zdejší školní budovy. Oddavky ve škole chtěl všemožným způsobem zmařit zdejší římskokatolický farář Al.\ Gee, což se mu však nepodařilo.

Jisté rozladění a~roztrpčenost mezi členstvem jednoty způsobily obecní volby pro různé nazírání na kandidáty starostenství. Později se ukázalo, že dobře jednali ti, kteří umožnili zvolení soc.\ demokrata Ludvíka Dedocha, bývalého člena naší jednoty, jenž se osvědčil energickým a~prozíravým starostou obce.

\bigskip

\noindent Stav členstva koncem roku: mužů 73, žen 19. Celkem \textbf{92}.

\clearpage

%%%%%%%%%%%%%%%%%%%%%%%%%%%%%%%%%%%%%%%%%%%%%%%%%%%%%%%%%%%%%%%%%%%%%%%%%%%%%%%
% ROK 1924
%%%%%%%%%%%%%%%%%%%%%%%%%%%%%%%%%%%%%%%%%%%%%%%%%%%%%%%%%%%%%%%%%%%%%%%%%%%%%%%

\phantomsection
\section{Rok 1924}

O~valné hromadě konané v~sokolovně 6.~ledna o~3.~hod.\ odpoledne za přítomnosti župního starosty bratra Hugona Sovíčka byli zvoleni:

\begin{description}
\item[Starostou:] Adolf Bárta.
\item[Náměstkem:] František Kozub.
\item[Náčelníkem:] Eda Jirák.
\item[Náčelnicí:] Anna Grünbaumová.
\item[Jednatelem:] Antonín Figalla.
\item[Vzdělavatelem:] František Šedivý.
\item[Pokladníkem:] Donát Grünbaum ml.
\item[Knihovníkem a~archivářem:] Josef Hurník.
\item[Ostatní členové výboru:] Antonín Klos (mlynář), Arnošt Kozák, František Jirák, Vlasta Blažejová, Josef Bárta.
\item[Náhradníky:] Jarmila Sokolová, Rudolf Buron, Alois Petras, František Besta.
\item[Režisérem divadla:] Miloslava Hurníková.
\item[Vyslanci do župy:] Adolf Valder a~Fr.\ Besta.
\item[Vyslanci do Č.O.S.:] Josef Hurník a~Eda Jirák.
\item[Revisory účtů:] Josef Bartek a~Ludvík Kočí.
\item[Do smírčího soudu:] Ig.\ Blažej, Jos.\ Bárta, Fr.\ Kozub, Alois Besta a~Jarmila Sokolová.
\end{description}

Členské příspěvky stanoveny: cvičící člen 12~Kč, ostatní členové 24~Kč ročně.

V~roku konána 1~mimořádná valná hromada, 4~schůze členské a~11~schůzí výborových. Účast nadprostřední.

\phantomsection
\subsection{Činnost tělocvičná}

Účast na cvičení z~počátku slabá, později lepší. Cvičily 4~složky: žáci, dorost, dospělí a~cvičitelský sbor.

Žáci měli 85~cvičebních hodin a~před šikem 14~proslovů. Dostávali zdarma časopis \emph{Vzkříšení!} Cvičili na veřejném cvičení v~Porubě 14, v~Klimkovicích 12, ve Vřesině 3, na župním sletu v~Mar.\ Horách 12.

Dorost měl 79~hodin. Cvičil v~Porubě (12), v~Klimkovicích (9), ve Vřesině (3), na župním sletu (4). Je pojištěn proti úrazu, platí 10~Kč ročního příspěvku, začež dostává Dorostenské Besedy. V~létě měl dvě vycházky do lesa.

Muži měli 70~cvičebních hodin. Cvičili na okrskovém v~Klimkovicích (3), ve Vřesině (6), v~Porubě 13~prostná a~1~družstvo nářadí. Župních závodů v~šestiboji prostém zúčastnilo se jedno družstvo a~jeden člen v~oddílu nižším. Nejstarší cvičenec Franta Jirák byl na prvém místě z~jednoty a~v~župě na 6.~místě ze 70~závodníků. Družstvo získalo diplom a~jednotliví závodníci také po jednom.

Cvičitelský sbor byl velmi pilný, měl 47~cv.\ hodin.

Župního výletu do Hlučína zúčastnilo se 15~bratrů.

Žačky, počtem 18, měly 78~cvič.\ hodin. Župního cvičení se zúčastnilo 12.

Dorostenek bylo 12. Cvičily v~66~hodinách dosti pilně a~zúčastnily se taktéž všech podniků.

Nejmenší činnost vykazují ženy, ze 6~zapsaných cvičily průměrně 3 a~to jen od června do srpna. Zúčastnily se veřejného cvičení v~Porubě a~župního sletu.

Čistý výnos z~veřejného cvičení v~Porubě byl 1\,073~Kč.

\phantomsection
\subsection{Činnost vzdělávací}

Proslovů bylo u~členstva 10, u~dorostu 9, u~žactva 15.

Pro členstvo konáno 5~přednášek, a~sice:
\begin{itemize}
\item Život a~dílo Bedřicha Smetany
\item Základní zásady sokolské
\item Co třeba připomenouti v~den 28.~října
\item Vl.\ Kratochvíl a~Josef Kotek, první hrdinové naší revoluce (tyto přednášky měl br.\ Fr.\ Šedivý)
\item Proč je žena Sokolkou (přednesla p.\ Miloslava Hurníková)
\end{itemize}

Na paměť stých narozenin našeho velikého skladatele Bedřicha Smetany pořádána 23.~března Smetanova oslava, sestávající z~přednášky a~koncertu (ukázky jeho tvorby --- orchestr a~sólové písně). Účast občanů prostřední. Na vstupném vybráno 359~Kč.

Divadla byla hrána za režie s.\ Miloslavy Hurníkové, a~sice:
\begin{enumerate}
\item 16.2.\ \emph{Mrak}, drama od Olgy Scheinpflugové. Příjem 555~Kč, vydání 128~Kč.
\item 18.2.\ \emph{Spravedlnost}, drama od téže. Příjem 622~Kč, vydání 294~Kč.
\item 20.4.\ \emph{Soupeřníci z~českých lesů}. Příjem 909~Kč, vydání 227~Kč.
\item 21.9.\ \emph{Ach, ta láska}, veselohra od Stolby. Příjem 607~Kč, vydání 164~Kč.
\item 13.12.\ Mikulášská zábava. Příjem 352~Kč.
\item 26.12.\ \emph{Batoch}. Pohostinsky sehrál svinovský Sokol.
\end{enumerate}

Výsledek divadel mravní i~hmotný byl velmi pěkný.

\phantomsection
\subsection{Různé}

Náš milovaný president Masaryk navštívil Slezsko 22.~června. Byl všude nadšeně vítán. Naše jednota zúčastnila se uvítání ve Svinově na nádraží, kde se zastavil a~pak pokračoval v~jízdě autem říšskou silnicí směrem k~Opavě.

Na dluh sokolovny splaceno toho roku 9\,800~Kč.

\phantomsection
\subsection{Úmrtí}

Zemřeli dva záslužní členové bratři: Cyril Dostál a~Antonín Besta.

\begin{novinovy-vystrizek}
\textbf{Župní věstník.}

\textbf{Poruba.} V~poslední době utrpěli jsme ztrátu dvou členů, kteří navždy opustili naše řady: 28.~srpna 1924 zemřel br.\ Cyril Dostál, zdejší snaživý a~v~celém okolí známý obchodník, podlehnuv v~nejlepším mužném věku zákeřné nemoci. Byl dobrým bratrem a~podporovatelem sokolských podniků. --- Dne 29.~září doprovodili jsme ku hrobu nejstaršího svého člena, spoluzakladatele naší jednoty, br.\ Antonína Bestu, jenž zemřel jako selský výměnkář ve stáří 75~let. Br.\ Ant.\ Besta byl oddaným, upřímným Sokolem, jenž v~jednotě zastával různé funkce a~o~její rozvoj až do konce jevil zájem, odkázav jí v~závěti 200~Kč. Zúčastnil se každého sletu v~Praze, posledně ještě v~roce 1920 jako stařeček 71letý se vypravil na dalekou cestu, aby viděl naposledy velkolepý obraz sokolského nadšení a~neúnavné práce. Br.\ Besta získal si také jako dlouholetý starosta obce Poruby velikých zásluh o~její rozkvět, přičiniv se hlavně o~zřízení nové okresní cesty středem Poruby. Své pokrokové smýšlení projevil i~tím, že před svou smrtí přestoupil k~církvi československé. Řada Sokolů a~hojně účastníků z~celého okolí ocenili o~jeho pohřbu jeho pevný charakter a~jeho poctivou práci. Čest památce obou bratrů! --- \textbf{(Divadlo.)} Dramatický odbor Sokola sehrál za režie sestry Miloslavy Hurníkové 21.~září s~plným úspěchem morálním i~hmotným Štolbovu veselohru \enquote{Ach ta láska}. --- Na listopad připravuje Scheinpflugovo drama \enquote{Mrak}.

\hfill J.~B.
\end{novinovy-vystrizek}

\bigskip

\noindent\textbf{Stav členstva 31.~prosince 1924:}\\
Mateřská jednota Poruba: muži 45, žen 9.\\
Pobočka Vřesina: muži 25, žen 5.\\
Celkem mužů 70, žen 14. \textbf{Celkem 84.}

\clearpage

%%%%%%%%%%%%%%%%%%%%%%%%%%%%%%%%%%%%%%%%%%%%%%%%%%%%%%%%%%%%%%%%%%%%%%%%%%%%%%%
% ROK 1925
%%%%%%%%%%%%%%%%%%%%%%%%%%%%%%%%%%%%%%%%%%%%%%%%%%%%%%%%%%%%%%%%%%%%%%%%%%%%%%%

\phantomsection
\section{Rok 1925}

O~valné hromadě, konané v~neděli 4.~ledna 1925 za přítomnosti 40~členů byli zvoleni:

\begin{description}
\item[Starostou:] Adolf Bárta.
\item[Náměstkem:] František Kozub.
\item[Náčelníkem:] Eda Jirák, náměstkem Franta Jirák.
\item[Náčelnicí:] Herma Lysková, vedoucí dorostenek Slávka Lazarová, vedoucí žen Alba Bestová.
\item[Vzdělavatelem:] František Šedivý.
\item[Členové výboru:] Antonín Figalla (jednatel), Adolf Valder, Frant.\ Jirák, Mojmír Bárta, Antonín Klos, Josef Hurník, Frant.\ Besta, Marie Pivoňová.
\item[Náhradníky:] Teofil Pivoň, Ant.\ Práta, Rudolf Sladký, Emil Mika.
\item[Režisér divadla:] Rudolf Sladký.
\end{description}

V~roku 1925 konáno bylo 6~schůzí výborových a~4~členské.

\phantomsection
\subsection{Činnost tělocvičná}

Činnost tělocvičná byla prostřední, jak vidět z~následujícího přehledu:
\begin{itemize}
\item Cvičitelský sbor --- 6~mužů průměrně --- cvičil 59~hodin, úhrnem 283.
\item Muži --- 6~průměrně --- cvičili 83~hodin, úhrnem 528.
\item Dorost --- 10~průměrně --- cvičil 79~hodin, úhrnem 564.
\item Žáci --- 15~průměrně --- cvičili 72~hodin, úhrnem 839.
\end{itemize}

Dorostenky za vedení horlivé sestry Mařenky Pivoňové konaly své povinnosti vzorně. Při okrskových závodech v~Polance 31.8.\ získaly dvě první místa, tři dorostenky navštěvovaly okrskový kurs. I~žačky svým dětským veselím často oživovaly místnosti tělocvičné.

U~mužů byl jakýsi nelad. Zvolený náčelník a~jeho zástupce vzdali se již v~únoru své funkce, a~tak vedl cvičení cvičitelský sbor za hlavní činnosti br.\ Ant.\ Klose. Na okrskových závodech v~Polance získalo družstvo 2.~cenu.

\phantomsection
\subsection{Činnost vzdělávací}

Proslovy: u~členstva 11, u~dorostu 10, u~žactva 5. Přednášky 4.

\phantomsection
\subsection{Divadla}
\begin{enumerate}
\item 15.2.\ \emph{Pes a~Kočka}.
\item 15.3.\ \emph{Vina}, drama od Jar.\ Hilberta.
\item 12.4.\ \emph{Druhé mládí}.
\item 20.9.\ \emph{Falešná kočička}, veselohra od Skružného.
\item 15.11.\ \emph{Devátá louka}.
\item 28.12.\ \emph{Zlatá rybka}, veselohra od Stolby.
\end{enumerate}

29.3.\ Koncert \enquote{Pěveckého sdružení z~Kopřivnice} za řízení Jana Uhlíře, v.\ poštmistra v~Kopřivnici. O~J.\ Uhlířovi je zmínka v~r.\ 1901. Pěvecké sdružení, jím založené a~řízené, podalo ukázky dobré secvičenosti i~pěkného výběru cenných skladeb. Docílilo plného úspěchu.

V~lednu --- jako každého roku --- byl pořádán ples. Hrubý příjem z~divadel, plesu i~koncertu byl 5\,958~Kč, čistý zisk 3\,175~Kč.

\phantomsection
\subsection{Jiné zprávy}

Do cvičebního sálu koupeny nástěnné hodiny. Jeviště opraveno nákladem 1\,062~Kč.

Místnímu hasičskému sboru povoleno cvičit v~sokolovně prostná cvičení.

Počalo se častěji hráti na loutkovém divadle pro děti. Návštěva při nízkém vstupném 50\,h--1\,Kč dosti dobrá.

Srpna zahájena jízda na nové dráze Svinov--Vřesina. V~sokolovně, jež byla zvenku krásně vyzdobena, konala se slavnostní hostina pro zástupce státních, zemských i~obecních úřadů, jakož i~všech činitelů, kteří měli zásluhy o~výstavbu této dráhy, která jest velikým dobrodiním pro naši obec i~celý náš kraj.

\bigskip

\noindent\textbf{Stav členstva:}\\
Jednota Poruba: 52~mužů, 14~žen.\\
Pobočka Vřesina: 25~mužů, 5~žen.\\
Celkem 77~mužů, 19~žen = \textbf{96~členů}.

\clearpage

%%%%%%%%%%%%%%%%%%%%%%%%%%%%%%%%%%%%%%%%%%%%%%%%%%%%%%%%%%%%%%%%%%%%%%%%%%%%%%%
% ROK 1926
%%%%%%%%%%%%%%%%%%%%%%%%%%%%%%%%%%%%%%%%%%%%%%%%%%%%%%%%%%%%%%%%%%%%%%%%%%%%%%%

\phantomsection
\section{Rok 1926}

Valná hromada konána v~neděli 3.~ledna 1926. Aklamací byli zvoleni:

\begin{description}
\item[Starostou:] Adolf Bárta.
\item[Náměstkem:] František Kozub.
\item[Náčelníkem:] František Jirák.
\item[Náčelnicí:] Vlasta Blažejová.
\item[Vzdělavatelem:] František Šedivý.
\item[Do výboru:] Ruda Sladký, Mojmír Bárta, Leoš Grünbaum, Adolf Švidrnoch, Ant.\ Figalla, Miloslava Hurníková, Rudolf Buron, Frant.\ Besta.
\item[Náhradníky:] Josef Petras, Tonda Klos, Miloslav Kozub, Miloslava Lazarová.
\item[Režiséři:] Leoš Grünbaum, Miloslava Hurníková, Ruda Sladký.
\item[Vyslanci do župy:] Ant.\ Klos, Adolf Valder.
\item[Revisoři:] Arnošt Osladil, Břetislav Blažej.
\item[Jednatelem:] Ant.\ Figalla.
\item[Zapisovatelem:] Adolf Švidrnoch.
\item[Pokladníkem:] Miloslav Kozub.
\item[Matrikářem:] Frant.\ Šedivý.
\item[Hospodářem:] Ruda Sladký.
\item[Archivářem:] Leoš Grünbaum.
\end{description}

Usneseno, aby členské příspěvky byly sníženy pro přispívající členy na 18~Kč, pro cvičící na 12~Kč ročně.

Výborových schůzí bylo 6, členských 3 za poměrně malé účasti.

\phantomsection
\subsection{Činnost tělocvičná}

Vzpruhou k~lepší účasti na cvičení byla příprava na VIII.~všesokolský slet v~Praze, jehož se naše jednota v~hojném počtu zúčastnila. Na sletu cvičilo: dorostenců 8, dorostenek 6, mužů 9, ženy 3. Celkem bylo v~Praze na sletu 48~osob, počet dosud nebývalý; u~všech zanechala návštěva Prahy trvalý, mohutný dojem.

Na okrskových závodech získal dorost 71\,\% bodů (V.~místo), muži 50\,\% bodů (II.~místo).

Župního zájezdu v~Opavě zúčastnilo se průvodu i~cvičení: 5~žáků, 12~žákyň, 8~dorostenců, 10~dorostenek, 6~mužů, 4~ženy.

Občanská záložna v~Porubě darovala dorostencům na cestu do Prahy 350~Kč a~půjčila 226~Kč. Na úhradu cestovného sehráno 11~loutkových divadel (řídil Eda Jirák) a~pořádán cvičební večírek s~proslovem br.\ Jos.\ Hurníka.

\phantomsection
\subsection{Činnost vzdělávací}

Proslovů sokolských u~členů 9, u~dorostu 11, u~žáků 1, proslovů jiných 4, u~dorostu 4, u~žáků 1.

Přednášek bylo 5, a~sice:
\begin{itemize}
\item v~lednu: Účel a~cíl sokolských sletů
\item v~únoru: Dílo a~význam spisovatele Fr.\ Sokola-Tůmy
\item v~březnu: Kdy je člověk v~pravdě samostatným
\item v~září: Nástin dějin tělocviku
\item v~říjnu: Smysl českých dějin
\end{itemize}

\phantomsection
\subsection{Divadla}
\begin{enumerate}
\item 28.~února: \emph{Soucit}, drama od F.\ Sokola-Tůmy.
\item 4.~dubna: \emph{Choulostivá historie}.
\item 16.~května: \emph{Léto} od Šrámka.
\item 25.~září: \emph{Cácorka}.
\item 24.10.: \emph{Oblaka}, hra od Kvapila.
\item 14.~listopadu: \emph{Zdravá nemocná}.
\item 15.~prosince: \emph{Madla z~Cihelny} --- pohostinské hry herců ostravského národního divadla.
\end{enumerate}

Mezi režiséry povstaly intriky a~řevnivost, které divadelnictví velmi poškodily. Sestře Miloslavě Hurníkové bylo znemožněno provésti výpravnou hru \emph{Vesnin štítek}, k~níž učinila čelné přípravy. Návštěva divadel upadla. Pohostinské hry, ač měly velkou návštěvu, pro vysokou režii nepřinesly velkého finančního výsledku. \emph{Madla z~cihelny} pro svůj příliš realistický obsah neuspokojila našeho obecenstva.

Z~projektovaného divadla v~přírodě (kus \emph{U~bílého koníčka}) pro neochotu herců také sešlo.

Na zahradě u~sokolovny sehrála Mládež českých socialistů s~velikým úspěchem lidovou operettu \emph{Z~českých mlýnů}.

29.~května závodil náš divadelní odbor v~Orlové divadelním kusem \emph{Pes a~kočka} a~obdržel diplom.

Pěvecký odbor Sokola v~Mar.\ Horách pořádal 18.~dubna v~naší sokolovně koncert, který byl výběrem skladeb i~dokonalým podáním velmi zdařilý.

\phantomsection
\subsection{Padesátiny bratra starosty Adolfa Bárty}

Br.\ Adolf Bárta, který je starostou jednoty od roku 1902 s~přestávkou válečných let, slavil 15.~března t.r.\ své padesátiny. V~uznání jeho zásluh byl mu uspořádán slavnostní večírek s~blahopřejnými projevy zástupců Sokola. Za dorost přednesl Zdeněk Janíček tuto básničku, složenou pisatelem:

\begin{verse}
Po dlouhé a~tuhé zimě přišlo jaro k~nám,\\
skřivan s~jásotem se vznáší k~nebes výšinám,\\
písní jeho probouzí se zvolna spící kraj,\\
brzy zapestří se květy zahrada i~háj.\\
Letos máme radost větší jako jiný krát,\\
poněvadž dnes můžem bratru svému blahopřát;\\
bratru milému nad jiné, každý z~nás to ví,\\
jakou láskou všichni lneme k~bratru Bártovi,\\
který dvacet let už stojí v~čele jednoty,\\
prožívá s~ní doby štěstí, též i~trampoty;\\
blahu jejímu vždy dával zdraví, statky, čas,\\
na něho se obracíme s~důvěrou vždy zas.\\
Sešli se dnes za něho dorost, bratři Sokoli,\\
že je váženým a~ctěným v~celém okolí,\\
že se dožil padesátky v~plné svěžesti\\
a~že má dva velké dary: zdraví a~štěstí.\\
Proto v~dnešní významný den pozvedáme hlas:\\
Nechť přej mu dále zdaru, život plný krás!
\end{verse}

Bratr starosta byl zřejmě dojat a~potěšen tou pozorností, poděkoval za všechny projevy a~pohostil přítomné členstvo.

\begin{figure}[H]
\centering
\adjustimage{width=\textwidth, height=\textheight, keepaspectratio}{images/stara_kronika_transcript_fotky/2.png}
\caption{Skupinová fotografie členů Sokola, 1926 v~Porubě}
\end{figure}

\bigskip

\noindent\textbf{Stav členstva:}\\
Jednota Poruba: 53~mužů, 18~žen.\\
Pobočka Vřesina: 16~mužů.\\
\textbf{Celkem 87~členů.}

\clearpage

%%%%%%%%%%%%%%%%%%%%%%%%%%%%%%%%%%%%%%%%%%%%%%%%%%%%%%%%%%%%%%%%%%%%%%%%%%%%%%%
% ROK 1927
%%%%%%%%%%%%%%%%%%%%%%%%%%%%%%%%%%%%%%%%%%%%%%%%%%%%%%%%%%%%%%%%%%%%%%%%%%%%%%%

\phantomsection
\section{Rok 1927}

Valná hromada konána 30.~ledna za účasti 37~členů a~přítomnosti br.\ Kupky, zástupce župy. Po přečtení provolání České Obce Sokolské k~valným hromadám podány zprávy a~vykonána volba výboru lístky. Poněvadž Bohuš Bárta, jenž obdržel nejvíce hlasů, volby nepřijal, byl ve druhé volbě 20~hlasy volen:

\begin{description}
\item[Starostou:] František Šedivý, učitel.
\item[Náměstkem:] Rudolf Buron, dělník.
\item[Náčelníkem:] Karel Klos.
\item[Náčelnicí:] Anna Lazarová.
\item[Vzdělavatelem:] Frant.\ Mrázek, učitel ve Svinově.
\item[Do výboru:] Josef Hurník, uč., Frant.\ Besta, zámečník, Herma Kubalová, Miloslav Kozub, Leoš Grünbaum, Eda Jirák, Teofil Pivoň, Ruda Sladký.
\item[Náhradníky:] Antonín Figalla, Antonín Klos, Milada Kozubová, Adolf Valder.
\item[Vyslanci do župy:] jednatel a~náčelník.
\item[Revisory účtů:] Adolf Švidrnoch, Adolf Valder.
\item[Archivářem:] Leoš Grünbaum.
\end{description}

Ve výborové schůzi po valné hromadě konané zvoleni:
\begin{description}
\item[Jednatelem:] Ant.\ Figalla.
\item[Pokladníkem:] Miloslav Kozub.
\item[Hospodářem:] Rudolf Buron (též novinářem).
\item[Matrikářem:] Fr.\ Šedivý.
\item[Archivářem:] Leoš Grünbaum (též knihovníkem).
\end{description}

Usneseno půjčovat jednotlivým spolkům jeviště i~sál za poplatek 70\,Kč; sokolníci mimo toho dají své korporaci za čištění sálu 10~Kč.

\phantomsection
\subsection{Činnost tělocvičná}

Zvolenému náčelníku vydatně pomáhal jeho bratr Antonín Klos, takže činnost tělocvičná neochabla.
\begin{itemize}
\item Cvičitelský sbor: 5~čl., týdně 2~hod., průměrná docházka 5,5.
\item Muži: 12~čl., týdně 6~hod., průměrná docházka 9,4.
\item Dorost: 9~čl., týdně 5~hod., průměrná docházka 6,7.
\item Žáci: 10~čl., týdně 4~hod., průměrná docházka 8,4.
\end{itemize}

Žáci cvičili v~některých měsících málo, jelikož neměli vedoucího.

Při pořádání štafetového běhu v~Klimkovicích umístili se muži na 2.~místě. Při župním štafetovém běhu, pořádaném u~příležitosti 25letého výročí zájezdu ČOS do M.~Ostravy, na trati Fulnek--Mor.\ Ostrava zúčastnil se 1~bratr.

Při položení základního kamene k~sokolovně ve Svinově běželo 7~bratrů a~2~dorostenci, časově doběhli na trati Poruba--Svinov třetí ze sedmi. Do základního kamene vložena adresa naší jednoty s~tímto věnováním:

\begin{verse}
\textit{Svinovskému Sokolu:}\\[0.5em]
Nadšení a~vytrvalost,\\
již jste dosud projevili,\\
dovede Vás, bratři i~sestry,\\
jistě k~vytyčenému cíli.\\
Až pak v~nové sokolovně\\
sejdete se k~nové práci,\\
ať Vám Vaši obětavost\\
Osud plným zdarem splácí.\\[0.5em]
\hfill (J.~B.)\\
2.~října 1927. Sokol Poruba u~Svinova.
\end{verse}

21.~srpna konáno veřejné cvičení, na kterém cvičilo 19~mužů, 12~žen, 19~dorostenců, 21~dorostenek, 28~žáků, 44~žaček. V~průvodu šli v~kroji: 22~muži, 7~žen, 17~dorostenců, dorostenek 22; 31~žáků, 50~žaček. Průběh cvičení byl dobrý, i~finanční výsledek uspokojil.

Okrskového cvičení v~Dolní Lhotě zúčastnili se 4~muži a~3~dorostenci, pak 4~ženy a~4~dorostenky.

Zájezdu do Košic na Slovensku se zúčastnili: bratři Josef Buron a~Antonín Krejčí. Z~žen Milada Kozubová a~Anča Lazarová.

Župní výlet na Štramberk obeslán 1~bratrem, 1~dorostencem a~2~dorostenkami.

V~Pustkovci vypomáhalo 10~dorostenek.

\phantomsection
\subsection{Činnost vzdělávací}

Činnost vzdělávací byla šířena bratrem Františkem Mrázkem, rodákem porubským, jenž působí nyní jako odborný učitel na měšťanské škole ve Svinově. Svůj úkol vzdělavatele pojímal vážně a~snažil se své mravní názory a~svůj idealismus vnésti i~mezi členstvo naší jednoty, k~němuž přicházel s~krátkými proslovy časovými i~povšechně vzdělávacími.

Delší přednášky, pečlivě vypracované, přednesl 3, a~sice:
\begin{itemize}
\item 9.~března: Život a~dílo p.\ pres.\ T.~G.\ Masaryka
\item 21.~září: Petr Bezruč šedesátníkem
\item 12.~listopadu: Veselou myslí dobudeš světa
\end{itemize}

Před veřejným cvičením měl pěkný proslov o~významu Sokola pro český národ.

V~dubnu vykonána vycházka do atelieru našeho slezského malíře Val.\ Držkovice v~Pustkovci.

V~květnu konána jarní osvětová škola ve Svinově, jíž se zúčastnilo 5~našich bratrů. V~červnu byla v~Porubě ideová škola pro dorost a~starší žactvo.

\phantomsection
\subsection{Divadla}

Divadelní činnost byla tento rok dosti chudá. V~1.~pololetí nehráno vůbec, ve 2.~pololetí sehrála naše jednota 2~kusy, a~sice:
\begin{enumerate}
\item 18.~září: \emph{Zabitý}, komedie od Olgy Scheinpflugové za režie br.\ Jos.\ Hurníka.
\item 4.~prosince: \emph{Ta třetí}, veselohra od Krylova za režie br.\ L.\ Grünbauma.
\end{enumerate}

Svinovský Sokol za režie br.\ Oldřicha Vácy sehrál 25.~prosince hru \emph{Šofér} jako revanche za naše hostování u~nich v~říjnu kusem \emph{Zabitý}.

Příjem z~divadel činil 759,60~Kč.

Našeho jeviště používaly jiné korporace jako: soc.\ dem.\ Dělníci, spolek Bratrství (českosl.\ církev ve Svinově) a~dramatický Dělnický odbor.

Též místní osvětová komise konala své přednášky v~sokolovně po celou dobu svého trvání.

Na loutkovém divadle sehrán v~březnu \emph{Benátský kupec}.

\phantomsection
\subsection{Jiné zábavy}

V~lednu pořádán ples, v~září \enquote{Vinobraní}.

Na počest p.\ presidenta Masaryka, který byl 27.~května toho roku zvolen po třetí presidentem, byl pořádán obcí lampionový průvod s~hudbou, jehož se všechno pokrokové občanstvo v~hojném počtu zúčastnilo.

\phantomsection
\subsection{Úmrtí}

V~tomto roku odešlo navždy mnoho znamenitých českých mužů: P.\ Jan Hiller, vzdělavatel ČOS a~redaktor Věstníku sok., p.\ JUDr.\ Václav Zendl, župní starosta, Jiří Váchal, Václav Horák, Dr.\ Pavel Blaho, slovenský poslanec, evang.\ biskup Zoch, vynikající pracovník a~přítel českého národa.

Nejvíce se nás však dotklo úmrtí našeho člena, milého bratra Antonína Bárty, který 30.~dubna 1927 po krátké nemoci neočekávaně nás opustil.

\begin{novinovy-vystrizek}
\textbf{NÁŠ ŽALOV.}

\textbf{Poruby}

\dag~Antonín Bárta z~Polanky u~Svinova.

Po krátké nemoci podlehl v~mužném věku 45~let. Od jinošských let byl oddaným přítelem sokolské myšlenky, zúčastňuje se rád a~ochotně všech jejich podniků. Na jeho morální i~hmotnou pomoc mohli jsme vždy určitě počítati. Jak dobrým byl Sokolem, takým byl i~manželem, bratrem, občanem. Proto není divu, že na pohřbu jeho 2.~května 1927 byla taková ohromná účast a~že při výstižné řeči br.\ starosty Fr.\ Šedivého zalily se slzami nejen oči členů jeho rodiny, nýbrž hořkým smutkem naplnila se hruď i~nesčetných jeho přátel. Vždyť je tak málo upřímných, milých a~poctivých lidí, jakým byl zesnulý. Bude nám všude a~dlouho scházeti. Zvláště bolestnou a~nenahraditelnou ztrátu utrpěla jeho ženuška, s~níž prožil v~ideálním manželství pouhých 8~let, a~jeho miláček, malá pětiletá Maruška, kterou si přivedl před třemi lety ze sirotčince, aby jí byl druhým otcem.

\begin{verse}
V~srdcích našich zvučí teskná melodie,\\
že Tě, drahý bratře, černá zem již kryje,\\
zarmoutils nás všechny, Tonečku náš drahý,\\
že jsi odešel nám, žel, tak příliš záhy!
\end{verse}

\hfill J.~B.

Uveřejněno ve Věstníku 14.5.1927.
\end{novinovy-vystrizek}

\bigskip

\noindent\textbf{Stav členstva 31.~prosince 1927:}\\
Poruba: 58~mužů, 18~žen.\\
Pobočka Vřesina: 14~mužů.\\
Celkem 72~mužů, 18~žen = \textbf{90~členů}.

U~vojska slouží Miroslav Blažej a~Jaroslav Kozub.

\clearpage

%%%%%%%%%%%%%%%%%%%%%%%%%%%%%%%%%%%%%%%%%%%%%%%%%%%%%%%%%%%%%%%%%%%%%%%%%%%%%%%
% ROK 1928
%%%%%%%%%%%%%%%%%%%%%%%%%%%%%%%%%%%%%%%%%%%%%%%%%%%%%%%%%%%%%%%%%%%%%%%%%%%%%%%

\phantomsection
\section{Rok 1928}

Ve valné hromadě konané 22.~ledna byli zvoleni:

\begin{description}
\item[Starostou:] Fr.\ Šedivý.
\item[Místostarostou:] František Kozub.
\item[Náčelníkem:] František Mika.
\item[Náčelnicí:] Milada Kozubová.
\item[Vzdělavatelem:] Frant.\ Mrázek.
\item[Jednatelem:] Ant.\ Figalla.
\item[Pokladníkem:] Vladimír Blažej.
\item[Režisérem:] Arnošt Osladil.
\item[Ostatní členové výboru:] Josef Hurník, Adolf Valder, Ant.\ Klos, Bohuš Bárta, Sláva Lazarová.
\item[Náhradníky:] Rudolf Buron, Emil Mika, Adolf Švidrnoch, Marie Pivoňová.
\item[Revisory účtů:] Donát Grünbaum ml.\ a~Adolf Valder.
\item[Vyslancem do župy a~kronikářem:] Josef Bárta.
\item[Matrikářem:] Fr.\ Šedivý.
\item[Do smírčího soudu:] Jos.\ Bárta, Ig.\ Blažej, Adolf Bárta, Donát Grünbaum st., Fr.\ Klos st.\ a~náčelník F.\ Mika.
\end{description}

V~r.\ 1928 bylo schůzí výborových 13, členských 5. Zapisovateli na schůzích byli Josef Barturek, Vladimír Blažej a~Arnošt Osladil.

Agendu jednatelskou vedl Josef Bárta za Ant.\ Figallu, který se zúčastnil šestiměsíčního kursu pro vyučování na lidových hospodářských školách.

\phantomsection
\subsection{Činnost tělocvičná}

Činnost tělocvičná nebyla stejnoměrná; v~některých měsících byla dobrá, v~jiných slabá.

Br.\ Antonín Klos zúčastnil se v~Praze kursu ČOS, pořádaného pro cvičitele. Okrskového kursu pro vedoucí žactva ve Svinově zúčastnili se 2~bratři a~2~dorostenci. Župní školy prostého tělocviku v~Mar.\ Horách zúčastnili se 2~bratři.

22.~dubna byla pořádána akademie s~tímto programem:
\begin{enumerate}
\item Proslov
\item Žačky: Útvarová prostná
\item Smíšený sbor: Slezská hymna
\item Žáci: Prostná (vojáci)
\item Recitace
\item Dorost: Cviky na hrazdě
\item Dorostenky: Rej svící
\item Dorost: Antičtí běžci
\item Muži: Bradla
\item Zpěv
\item Recitace
\item Ženy: Prostná
\item Muži: Čtverylka
\item Smíšený sbor: Sokolská
\end{enumerate}
Všechna čísla byla pěkně provedena a~líbila se.

Června konány závody dorostenců a~žáků III.~okrsku. Naši dorostenci se umístili na 1.~místě a~získali diplom. Dorostenec Zdeněk Janíček dobyl 2.~místa. Žáci získali 4.~místo.

Br.\ Teofil Pivoň čestně se umístil na mezisletových závodech ČOS v~Praze, získav 79,25\,\% bodů. Při župních závodech byl na 27.~místě.

Naše pobočka Vřesina konala 5.~srpna ve své obci veřejné cvičení za výpomoci našich členů a~3~sester ze Svinova. Cvičení bylo zdařilé.

Krajského sletu v~Prostějově a~v~Olomouci činně se zúčastnily 2~sestry a~4~dorostenky.

Sestra Anna Lazarová, vedoucí žaček, odstěhovala se v~srpnu do Svinova, proto vedení žaček převzala sestra náčelní Milada Kozubová.

Naše jednota se čelně zúčastnila slavnostního položení základního kamene sokolovny ve Svinově 1.~července t.r.

Do zdravotního kursu, pořádaného župou v~M.~Ostravě, byli vysláni bratři František Mrázek a~Fr.\ Mika.

Cvičení žáků bylo zanedbáváno, poněvadž nebylo mezi členy řádného vedoucího, kdo by se jim věnoval.

\phantomsection
\subsection{Činnost vzdělávací}

Činnost vzdělávací byla slabší než v~předešlých letech; řídil ji br.\ František Mrázek, který na jarní vzdělávací škole přednášel o~sokolských otázkách za účasti 30~dorostenců a~bratrů.

Narozeniny presidenta Masaryka oslaveny přednáškou a~recitacemi. --- Bratra Jana Čapka vzpomenuto přednáškou u~hořící hranice za Vřesinou. Na jeho pomník ve Štramberku věnovala jednota župou jí předepsaný obnos.

Br.\ Jan Kameníček, odborný učitel ze Svinova, přednášel na členské schůzi o~L.~N.\ Tolstém.

Br.\ Josef Hurník, zdejší učitel, měl 2~přednášky:
\begin{itemize}
\item O~slibu sokolském a~národní přísaze
\item O~Palackém a~Denisovi
\end{itemize}

Br.\ Fr.\ Mrázek přednášel o~alkoholismu.

\phantomsection
\subsection{Divadla}

Divadelní činnost nebyla rovněž uspokojivá. Sehrány byly jen 2~veselohry za režie br.\ Arnošta Osladila:
\begin{enumerate}
\item 23.~září: \emph{Moderní ženu za každou cenu}.
\item 25.~prosince: \emph{Bujná krev}.
\end{enumerate}

V~prvé veselohře vynikli představitelé hlavních úloh: Milada Bártová, Rudolf Ryš (ze Svinova) a~náš výborný komik br.\ Ruda Sladký.

Pěkná byla akademie, pořádaná 27.~října na oslavu naší samostatnosti, jejíž program sestavil a~z~velké části nacvičil br.\ Josef Bárta. Program sestával z~přednášky (br.\ Fr.\ Mrázek), ze zpěvních a~hudebních čísel, recitací a~jednoaktovky \emph{Shledání}. --- Houslová sóla s~doprovodem klavíru hrál jako host p.\ Gustav Vilts, úředník z~Mar.\ Hor.

Sylvestrovská zábava, řízená a~nacvičená Josefem Bártou měla pěkný průběh a~úspěch. Jednoaktovka \emph{Cesta k~práci} od Olivy pěkně pobavila, taktéž i~zpěvy s.\ Zdenky Křížkové z~Háje a~bratra Miroše Zaňusky se líbily. Jako herec i~konferenciér bezplatně účinkoval František Navrátil, jenž v~dřívějších letech náležel k~našim nejlepším ochotníkům.

7.~října pořádáno v~sokolovně \enquote{Vinobraní} s~taneční zábavou.

Ve dnech 28.--30.~září pořádána byla v~sokolovně ovocnářská výstava s~včelařskou exposicí.

\begin{novinovy-vystrizek}
\textbf{PORUBA U~SVINOVA.} Jednota Sokol nás potěšila pěkným divadlem. Veselohra \enquote{Moderní ženu za každou cenu}, sehraná 23.~září, náleží ke hrám, které nejenom pobaví, ale i~poučí. Jest jemná, prostá vší frivolnosti a~banálnosti a~proto lze ji venkovským jednotám dobře doporučiti. Všichni účinkující dobře zhostili se svých úloh, zvláště představitelé hlavních úloh (Milada Bártová, Rudolf Ryš, Ruda Sladký) sklidili zasloužené uznání. Přáli bychom si, aby se hrálo častěji a~nedělaly se několikatýdenní přestávky. --- \textbf{Ovocnářská výstava} s~včelařskou exposicí, pořádaná v~posledních 3~dnech záříjových, byla pro naši obec velikou událostí. Naše milá Sokolovna nemohla skoro ani pojati všechny návštěvníky z~blízkého i~dalekého okolí, kteří přišli potěšit se pohledem na spoustu překrásného ovoce, vyloženého na vkusně dekorovaných stolech. 60~vystavovatelů zaslalo sem nejkrásnější plody svých zahrad. I~včelařská výstavka poskytla mnoho zajímavého a~poučného. Zvláště vábil vystavený med velkovčelaře Fr.\ Kudely z~Čavisova, který za svou exposici obdržel první cenu. Překrásné skupiny jiřin, kolekce zavařenin, ovocné šťávy a~vína (firma Šmuk), různé zahradnické potřeby i~strojky k~lisování ovoce a~j.\ (fa Šafrata), diagramy a~obrazy ak.\ malířky H.~Salichové-Hálové krásně doplňovaly výstavu a~učinily ji cennou a~zajímavou. Posudky odborníků zněly velmi lichotivě a~místní osvětová komise, která výstavu aranžovala, může míti radost z~krásného kulturního činu. --- A~ještě něco zvláštního chystá se v~Porubě: 28.~října bude na náměstí u~křižovatky okresních silnic odhalen pomník Svobody, jímž bude ohromný bludný balvan, 3\textonehalf{}~m dlouhý, 1\textthreequarters{}~m vysoký, vážící asi 90~metr.\ centů. Tento obrovský kámen, sestávající ze švédské žuly, byl namáhavým způsobem vydobyt z~potůčku na západním konci Poruby, v~němž klidně odpočíval statisíce let. Přijďte se podívat na slavnostní odhalení v~den svátku svobody.

\hfill J.~B.
\end{novinovy-vystrizek}

\phantomsection
\subsection{Jiné události}

Jednota Sokol Opava vydala k~28.~říjnu na památku 10tiletého trvání naší samostatnosti pěkný spis \emph{Rozmach sokolské myšlenky na Opavsku}, který za pomoci jednot sestavil br.\ Ludvík Novák, profesor v~Opavě. Příspěvky týkající se naší jednoty sestavil a~dodal jemu Josef Bárta. 10~výtisků odebrali naši členové.

Nové pianino za 9\,000~Kč bylo koupeno od hudební firmy Vladimír Kupka v~Mar.\ Horách, která převzala náš starý klavír za 1\,700~Kč.

Firmě Ig.\ Blažej, továrna na nábytek v~Porubě, blahopřáno ku 25tiletému trvání. Její majitel br.\ Ignát Blažej je od začátku význačným a~zasloužilým členem naší jednoty. Pod jeho obezřetným vedením vyrostl podnik z~malých počátků na jeden z~největších v~našem kraji, neboť zaměstnává na sto dělníků a~úředníků. Prodejna nábytku v~Mor.\ Ostravě dobře konkuruje jiným, většinou německým firmám. Přejeme jí dalšího úspěchu.

Restaurace v~sokolovně byla od 1.~srpna do 31.~prosince 1930 pronajata manželům Martinovi a~Bertě Bichlerovým. Dosavadní nájemci Josef a~Štěpánka Lazarovi odešli do Svinova na vlastní hostinec.

\textbf{Sbírky.} Na podporu Bulharů, postižených zemětřesením, bylo zasláno 136~Kč, na vyhořelou sokolovnu v~Hošťálkové 50~Kč.

\phantomsection
\subsection{Rozloučení se starostou Adolfem Bártou}

Našemu dlouholetému starostovi br.\ Adolfu Bártovi a~jeho choti sestře Marii Bártové při jeho odchodu do Svinova odevzdala deputace naší jednoty pěkný památník, jejž ilustroval p.\ Dvorský, architekt u~fy Ig.\ Blažej. Věnování, které podepsali členové výboru za Družstvo pro udržování sokolovny a~jež sestavil zástupce jednatele Josef Bárta znělo takto:

\begin{quote}
Bratru Adolfu Bártovi, dlouholetému starostovi Těl.\ jednoty Sokol v~Porubě a~jeho věrné družce sestře Marii Bártové věnuje na důkaz lásky a~vděčnosti Porubský Sokol.

\begin{verse}
Kdo své vlasti slouží, práva její hájí,\\
K~tomu dobří lidé úctu, lásku mají.\\
Že ze spolků Sokol nejvíc pro vlast dělá,\\
o~tom přesvědčena veřejnost je celá.\\
I~Vy Sokolu jste mnoho věnovali,\\
Svou obětavostí mnohým příklad dali.\\
Na pozemku Vašem sokolovna stojí,\\
do ní přicházíme jako k~máti svojí.\\
Nalezne v~ní každý, po čem srdce touží,\\
zdraví zachová si, život si prodlouží.\\
Dokud sokolovna naše bude státi,\\
budem vždycky s~láskou na Vás vzpomínati.
\end{verse}
\end{quote}

Ve výborové schůzi konané 8.~listopadu bylo vytýkáno br.\ starostovi Šedivému, že odmítá účinkovat při sokolských divadlech, zatím co hraje a~režíruje divadla v~Dělnické tělocvičné jednotě. Tím se cítil br.\ Šedivý uražen a~po výměně názorů prohlásil, že se vzdává funkce starosty Sokola. Následkem toho zastupoval jednotu až do valné hromady místostarosta br.\ František Kozub.

\phantomsection
\subsection{Památník Svobody}

Památník Svobody --- bludný kámen, jenž je umístěn u~křižovatky okresních cest, blízko Obecního domu, byl namáhavě půjčenými stroji z~vítkovických železáren dobyt z~potoka porubského na severozápadní části Poruby. Jeho dobývání a~převezení z~potoka až na nynější místo řídil br.\ Adolf Valder a~slavnostní řeč o~významu kamene jako pomníku naší samostatnosti měl při jeho odhalení 28.~října br.\ učitel Josef Hurník, jenž o~uskutečnění postavení tohoto památníku jako referent v~obecním zastupitelstvu má značné zásluhy.

Na pomníku je bronzový znak československý a~nápis \enquote{Svornost v~obci, mír ve státě!} 1918--1928. Kéž se tímto nápisem řídí nejen místní, nýbrž i~všichni občané naší republiky!

\bigskip

\noindent\textbf{Stav členstva 31.~prosince 1928:}\\
Poruba: 68~mužů, 20~žen.\\
Vřesina: 29~mužů.\\
\textbf{Celkem 117~členů.}

\clearpage

%%%%%%%%%%%%%%%%%%%%%%%%%%%%%%%%%%%%%%%%%%%%%%%%%%%%%%%%%%%%%%%%%%%%%%%%%%%%%%%
% ROK 1929
%%%%%%%%%%%%%%%%%%%%%%%%%%%%%%%%%%%%%%%%%%%%%%%%%%%%%%%%%%%%%%%%%%%%%%%%%%%%%%%

\phantomsection
\section{Rok 1929}

Valná hromada konána v~neděli 11.~ledna 1929 za účasti 31~členů a~5~hostů. Župu zastupoval jednatel br.\ Dr.\ Sálek, advokát v~Mor.\ Ostravě, jenž před ukončením valné hromady obšírně promluvil o~úkolech sokolstva v~osvobozeném státě a~o~jeho poměru k~vládě, církvím a~politickým stranám.

Zvoleni byli:
\begin{description}
\item[Starostou:] František Kozub.
\item[Místostarostou:] Josef Bárta.
\item[Náčelníkem:] Antonín Klos.
\item[Náčelnicí:] Milada Kozubová.
\item[Vzdělavatelem:] Leoš Grünbaum.
\item[Jednatelem:] Josef Hurník.
\item[Pokladníkem:] Ant.\ Figalla.
\item[Archivářem:] Leoš Grünbaum.
\item[Matrikářem:] Adolf Valder.
\item[Hospodářem a~novinářem:] Rudolf Buron.
\item[Režisérem:] Josef Bárta.
\item[Zapisovateli:] Vladimír Blažej a~Inoš Osladil.
\item[Členy bez funkce:] Marie Pivoňová a~Martin Bichler.
\item[Revisoři účtů:] Karel Klos, Miloslav Kozub.
\item[Náhradníky:] Alois Petras, Fr.\ Pavlíček, Bohuš Bárta, Emil Mika.
\item[Zdravotníkem:] Frant.\ Mrázek.
\item[Do smírčího soudu:] Jos.\ Bárta, Ad.\ Bárta, Ig.\ Blažej, Donát Grünbaum st., Fr.\ Klos st.
\end{description}

\phantomsection
\subsection{Tělocvičná činnost}

Tělocvičná činnost byla prostřední; cvičení žactva i~dorostu pro nedostatek vedoucích bylo zanedbáváno. Br.\ náčelník A.\ Klos pro stavbu mlýna i~přípravy k~sňatku nemohl se cvičení věnovati, jeho náměstek F.\ Mika docházel do večerního kursu. Přes to však vykazuje jednota některé úspěchy.

Na župních závodech umístilo se čtyřčlenné družstvo na 8.~místě. Vratislav Blažej zúčastnil se v~Mor.\ Ostravě třídenního kursu pro vedoucí žactva.

Větším nebo menším počtem zúčastnila se jednota cvičení v~Dolní Lhotě, Vřesině, okrskového cvičení ve Svinově a~sletů v~Místku a~Orlové.

Pěkné bylo místní veřejné cvičení za výpomoci svinovských členů. Hrubý příjem činil 5\,602~Kč, vydání 3\,393~Kč, čistý zisk 1\,669~Kč.

\phantomsection
\subsection{Činnost vzdělávací}

Činnost vzdělávací vykazuje pěkné výsledky zásluhou jednatele bra Jos.\ Hurníka, jenž dílem sám obstarával proslovy pro dorost a~žactvo, i~přednášky na členských schůzích, nebo vhodné řečníky odjinud získával. Zastupoval pak zvoleného vzdělavatele bra Leoše Grünbauma, který omluviv se nedostatkem času z~října své funkce se vzdal.

Proslovů pro dorost a~žactvo bylo 14, pro členstvo 6.

Na členských schůzích podával br.\ Hurník přehled nejdůležitějších časových událostí ze sokolského i~veřejného života.

Přednášky:
\begin{itemize}
\item 20.~února: \enquote{Moje cesta do Anglie a~Francie} --- přednášel br.\ Fr.\ Dostalík, učitel ze Třebovic.
\item 13.~března na oslavu narozenin p.\ presidenta přednášel br.\ Josef Hurník na téma \enquote{Masaryk náš vůdce}.
\item 24.~dubna odborný učitel Vlad.\ Hyl ze Sl.\ Ostravy přednášel o~své cestě do Palestiny a~Egypta.
\item 29.~května přednášel br.\ Josef Hurník \enquote{O~jednotné diferencované škole}.
\item 29.~srpna přednášel br.\ J.\ Hurník na téma \enquote{Z~historie Ostré Hůrky}.
\item 9.~prosince br.\ Josef Bárta: \enquote{O~činnosti sokolské v~republice i~za hranicemi}.
\end{itemize}

Týž měl také proslov při dětské besedě, pořádané 15.~prosince, k~níž s~dětmi nacvičil vhodné písně a~recitace. Po produkci byly děti, jež chodily do cvičení, poděleny dárky.

Členové jednoty zúčastnili se také oslavy Husovy, konané každého roku na kopci u~Třesiny a~v~září veliké slavnosti na Ostré Hůrce, kde byl odhalen za účasti tisíců slezského lidu \enquote{Památník slezského odboje}. --- Na památné Ostré Hůrce u~Chabičova pořádány byly v~letech 1869--1898--1918 mohutné tábory, na nichž český lid protestoval proti germanisaci Slezska a~dožadoval se rovnoprávnosti ve školství i~v~úřadech. Nejmohutnější byl tábor r.\ 1918, pořádaný těsně před převratem, na němž ústy všech řečníků již směle voláno po samostatném československém státě.

\phantomsection
\subsection{Divadla}

Divadel bylo sehráno 6, a~sice:
\begin{enumerate}
\item 10.3.\ \emph{Jak Terka přišla do rodiny}.
\item 30.4.\ \emph{Loupežníci v~českých lesích}.
\item 29.9.\ \emph{Babí léto}.
\item 13.10.\ \emph{Babí léto} opakováno.
\item 1.12.\ \emph{Ta naše Máňa}.
\item 25.12.\ \emph{Firma Piškot, Chaplin a~spol.}
\end{enumerate}

Divadla byla nastudována br.\ Inošem Osladilem a~měla pěkný úspěch morální i~finanční. Na vstupném (5\,Kč, 4, 3, 2) se vybralo 4\,059~Kč. Nejvíce se líbila hra \emph{Babí léto}, k~níž zpěvy nacvičil a~klavírní doprovod obstaral učitel Lumír Bárta.

Jako zpěvačka se velmi uplatnila sl.\ Zdenka Křížková z~Háje, která zde přes pole byla návštěvou u~svého strýce a~nám při divadlech i~akademiích hrou a~zpěvem vypomáhala.

Mimo ní vypomáhali našim hercům členové jednoty Sokol ze Svinova (Milada Mártová, Oldřich Vaca, Rudolf Ryš), ze Vřesiny sestry Jarmila Mártová a~Anděla Hučíková, z~Polanky s.\ Jarmila Kavalová, z~Vítkovic p.\ Mikšaník. Z~místních pánové: Vilibald Klos, Slávek Klos, Fr.\ Navrátil.

\phantomsection
\subsection{Různé}

Na Lapkovu sokolovnu zaslali jsme 50~Kč, Muzeální společnosti pro Slovensko 10~Kč.

Sletu v~Plzni zúčastnili se Ladislav a~Filomena Martínkovi.

Horlivý cvičenec Teofil Pivoň odstěhoval se s~matkou a~sestrou do Ořechova, kde si zakoupil selskou usedlost.

Náš milý náčelník Tonda Klos oženil se s~Mařenkou Pivoňovou, naší dlouholetou horlivou sokolkou, z~čehož máme všichni radost a~přejeme jim hodně zdaru do budoucího života.

\phantomsection
\subsection{Úmrtí}

V~tomto roku odešli navždy 3~nám známí zasloužilí sokolští pracovníci:
\begin{enumerate}
\item Dr.\ František Zelený, bývalý starosta naší župy;
\item br.\ Jan Jandárek, ředitel měšťanské školy ve Svinově, jenž mnoho let horlivě působil nejen pro rozvoj českého školství ve Svinově, nýbrž i~v~samém jejím Sokolu. I~v~naší obci několikráte přednášel.
\item Br.\ Donát Grünbaum, zdejší poštmistr, dobrý člověk a~upřímný Sokol.
\end{enumerate}

Všichni byli pohřbeni v~ostravském krematoriu za velké účasti Sokolů i~přátel.

\begin{novinovy-vystrizek}
\textbf{Donát Grünbaum z~Poruby.} Sokol v~Porubě u~Svinova želí předčasného skonu svého dobrého bratra, jenž jako poštmistr 10~let působil v~obci a~po celou dobu byl oddaným a~obětavým členem jednoty. Zvláště v~hudebním kroužku rád vypomáhal a~pro svou milou a~ušlechtilou povahu byl všemi ctěn a~milován. Zajisté i~jednoty Napajedlech (jeho rodiště), Pohořelicích a~Svinově, kde dříve působil, mají na něho nejlepší vzpomínky. Odešel ve věku 55~let a~byl zpopelněn v~ostravském krematoriu 1.~června. Za jednotu naši vřelými slovy se s~ním rozloučil místostarosta br.\ Jos.\ Bárta, obřady vykonal československý farář Pařík. Bratře Donáte, děkujeme Ti za všechnu Tvou lásku a~práci v~jednotě, kterou jsi tak rád konal a~ujišťujeme Tě naší vděčnou a~trvalou pamětí!

\hfill J.~B.
\end{novinovy-vystrizek}

\bigskip

\noindent\textbf{Stav členstva koncem roku 1929:}\\
V~Porubě: 67~mužů, 20~žen.\\
V~pobočce Vřesina: 35~mužů, 5~žen.\\
\textbf{Celkem 127~členů.}

\clearpage

%%%%%%%%%%%%%%%%%%%%%%%%%%%%%%%%%%%%%%%%%%%%%%%%%%%%%%%%%%%%%%%%%%%%%%%%%%%%%%%
% ROK 1930
%%%%%%%%%%%%%%%%%%%%%%%%%%%%%%%%%%%%%%%%%%%%%%%%%%%%%%%%%%%%%%%%%%%%%%%%%%%%%%%

\phantomsection
\section{Rok 1930}

Valná hromada konána 19.~ledna za účasti 35~členů a~4~hostů, tj.\ cca 40\,\% členstva. Župu zastupovala sestra M.\ Dvořáková z~Mar.\ Hor, která připomněla členům povinnost k~jednotě, ocenila vykonanou práci a~povzbudila k~další činnosti.

Aklamací byli zvoleni:
\begin{description}
\item[Starostou:] Josef Bárta.
\item[Jeho náměstkem:] František Kozub.
\item[Náčelníkem:] Vratislav Blažej.
\item[Náčelnicí:] Milada Kozubová.
\item[Jednatelem:] Adolf Švidernoch.
\item[Pokladníkem:] Ant.\ Figalla.
\item[Vzdělavatelem:] Josef Bárta.
\item[Matrikářem:] Otakar Schindler.
\item[Režisérem:] Inoš Osladil.
\item[Ostatní členové:] Josef Hurník, Ant.\ Krejčí, Bohuš Bárta, Vláďa Blažej.
\item[Náhradníci:] Bohumil Chvíla (novinářem), Rudolf Buron (hospodářem), Adolf Valder, Břetislav Zdražila.
\item[Revisoři účtů:] Miloslav Kozub, Karel Klos.
\item[Smírčí soud:] Adolf Bárta, Ing.\ Marx, Frant.\ Kozub, Frant.\ Klos st., Miloslava Hurníková.
\item[Zapisovatelem:] Zdeněk Janíček.
\end{description}

Členský příspěvek pro cvičící 12~Kč ročně, pro necvičící 24~Kč ročně.

Tento rok možno označiti za rok klidné, úspěšné práce a~shody, která nebyla ničím porušena. Starosta Josef Bárta byl zároveň vzdělavatelem a~od července do konce roku zastupoval také jednatele Adolfa Švidernocha, který na delší dobu z~Poruby se vzdálil.

\phantomsection
\subsection{Tělocvičná činnost}

Tělocvičná činnost za vedení sympatického a~horlivého náčelníka br.\ Vrat.\ Blažeje se pěkně vyvíjela, po jeho odchodu za vojenskou povinností poněkud ochabla. Také náčelní sestra Milada Kozubová se sestrami Vlastou Hrubou a~Lidou Bestovou, jež jí obětavě pomáhaly, měla pěkné výsledky.

27.~dubna pořádána tělocvičná akademie, na níž cvičilo: 6členné družstvo mužů nízkou hrazdu, 5~dorostenců prostná s~praporky, 6~žáků řemeslníky; 6~žen různosti, 12~dorostenek rej, 14~žaček hru. Návštěva obecenstva i~úspěch uspokojivý.

Pěkně se vydařilo také veřejné cvičení, pořádané 20.~července. Průvod s~hudbou od Obecního domu k~sokolovně byl četný a~velmi malebný. V~kroji bylo 20~mužů, 10~žen, 8~dorostenců, 14~dorostenek, 28~žáků a~18~žákyň. Na letním cvičišti u~sokolovny provedeno zdařilé cvičení za účasti četného obecenstva.

Nářadí cvičila 3~družstva, z~nichž vyniklo starší družstvo na bradlech. Mužský dorost cvičil skupinové prostná, 12~žáků prostná a~pak společně 22~žáků předvedlo hry. Žen cvičilo 10; 7~z~Poruby a~3~z~Velké Polomě. Dále cvičilo 14~dorostenek a~14~žaček.

\phantomsection
\subsection{Jiné podniky}

13.~dubna zúčastnilo se šestičlenné družstvo župních závodů nižšího oddělení ve Vítkovicích a~docílilo 86\,\%; 3~závodníci obdrželi diplom. Nejlepším z~nich byl Antonín Krejčí.

27.~dubna podrobili se závodům okrskovým 3~dorostenci, z~nichž Bedřich Konečný dobyl 3.~místa.

22.~června byli na veřejném cvičení v~Krásném Poli 4~muži, 2~žáci a~1~dorostenec v~kroji.

29.~června zúčastnila se naše jednota veřejného cvičení naší pobočky Vřesiny a~jednoty svinovské.

Župní školy se zúčastnil Vráťa Blažej a~Zdeněk Janíček; kursu rytmiky Milada Kozubová.

\phantomsection
\subsection{Činnost vzdělávací}

Před šikem měl vzdělavatel 29~proslovů. Přednášky ve členských schůzích měli:
\begin{itemize}
\item Br.\ František Mrázek: \enquote{Propagace sokolství}
\item Br.\ Jan Kameníček ze Svinova: a)~O~presidentu T.~G.\ Masarykovi, b)~O~Aloisu Jiráskovi
\item Br.\ Josef Bárta o~Dru Mir.\ Tyršovi
\end{itemize}

V~říjnu před širším obecenstvem v~sokolovně přednášel br.\ Fr.\ Šedivý \enquote{O~významu 28.~října}.

\textbf{Časopisy.} Jednota odebírala mimo povinného župního věstníku ještě časopis \emph{Sokol}, \emph{Věstník Č.O.S.}, \emph{Sokolské besedy}, \emph{Cvičitele}, \emph{Cvičitelku}, \emph{Jas} a~\emph{Vzkříšení}.

\phantomsection
\subsection{Divadla}

Za režie bra Inoše Osladila byly sehrány kusy:
\begin{enumerate}
\item 20.~března \emph{Probuzenci}, historické drama od Šuberta. Nová dekorace, kroje, účast veliká, plný zdar. Příjem 1\,337~Kč, vyd.\ 841~Kč.
\item 20.~dubna lidová operetka \emph{Z~českých mlýnů} s~úspěchem morálním, k~němuž přispěla hlavně oblíbená herečka a~zpěvačka Zdenka Křížková. Příjem 906~Kč, vydání 804~Kč.
\item 5.~října sehrána veselohra \emph{Teta Fany, postrach rodiny}.
\item 25.~prosince s~velkým zdarem sehrána byla fraška \emph{Tulácké dobrodružství}.
\end{enumerate}

Mimo těchto her byla pořádána 21.~prosince dětská besídka, k~níž nacvičil učitel Grohman kus \emph{Dvě Mařičky} a~vzdělavatel 3~zpěvy. Po besídce byly děti, jež cvičí v~Sokole, poděleny cukrovím, ořechy a~jablky.

Technické práce na jevišti vykonával režisér za pomocí některých bratrů. Divadelní knihovna čítá asi 265~knížek.

Pro úplnost se uvádí, že v~lednu byl pořádán obligátní ples, pěkně navštívený, který vynesl čistého 956~Kč a~v~září \enquote{Vinobraní} 300~Kč.

V~tomto roku mimo valné hromady bylo konáno 12~schůzí výborových a~4~členské, které byly oživeny čísly zpěvními a~hudebními (housle a~pianino). Účast 40\,\%.

\phantomsection
\subsection{Jiné události}

Blahopřejné projevy zaslala jednota:
\begin{enumerate}
\item K~osmdesátým narozeninám p.\ presidenta T.~G.\ Masaryka,
\item K~padesátinám slezského rodáka ministra K.\ Engliše,
\item K~šedesátinám zasloužilého župního starosty br.\ H.\ Součka (narozen 13.4.1870 v~Jihlavě).
\end{enumerate}

Za zaslaná blahopřání došlo poděkování od Kabinetní kanceláře p.\ presidenta i~od jmenovaných jubilantů.

V~červnu byly otevřeny nové sokolovny v~Bílovci a~Kateřinkách.

V~prosinci vzdali se pro nemoc svých funkcí: a)~bratr Dr.\ Josef Scheiner, starosta ČOS a~Slovanského Sokolstva a~b)~bratr Dr.\ Jindra Vaníček, náčelník ČOS. Celé Sokolstvo želí jejich odchodů a~vřele se sklání před jejich neocenitelnou celoživotní prací, kterou věnovali rozmachu Sokolstva.

\phantomsection
\subsection{Úmrtí}

12.~března t.r.\ zemřel mistr Alois Jirásek, tvůrce historických románů, buditel národa a~upřímný přítel sokolství, jehož byl oddaným členem v~několika jednotách. Slavného jeho pohřbu v~Praze zúčastnilo se 1\,885~sokolů v~kroji se 32~prapory. --- Važme si jeho velikého odkazu, čtěme jeho vzácné knihy, abychom poznali svou minulost a~čerpali z~nich lásku a~oddanost ku svému národu.

Při sokolských slavnostech o~Hronově, jeho rodišti, řekl Jirásek: \enquote{Četl jsem také o~Bratrstvu, o~válečné jednotě statečných mužů, kteří se proslavili svou silou a~udatenstvím. Ale ti bojovali v~cizině a~za cizí krev prolévali. Dnes však má národ náš jiné bratrstvo, statečné a~jaré, své Sokolstvo, které slouží jen jemu a~jméno jeho proslavilo i~za hranicemi. Duch tohoto bratrstva: zdraví těla i~ducha, radost z~práce i~z~zápasu pro právo, pro vše dobré a~lidské nechť zůstává neporušen, nechť osvěžuje a~sílí všechny!}

V~Praze zemřel 17.~března význačný sokolský pracovník br.\ Eduard Navrátil, jenž vydával časopisy \emph{Vzdělání lidu}, \emph{Vzkříšení} a~obrázkový týdeník \emph{Jas}. Byl župním vzdělavatelem a~členem výboru Č.O.S.

Dne 22.~května A.P.\ zahynul tragickou smrtí člen naší jednoty br.\ Jaroslav Augustin, byv při svém zaměstnání na železniční trati zasažen jedoucím vlakem. Jeho mladého života želí se zarmoucenými rodiči i~všichni členové porubské jednoty, kteří se jeho pohřbu na zdejším hřbitově v~hojném počtu zúčastnili.

\phantomsection
\subsection{Odchod členů do Svinova}

V~prosinci A.P.\ přestěhovali se z~Poruby do Svinova členové jednoty Josef Bárta, Josef Hurník a~jeho choť Miloslava.

Br.\ učitel Josef Hurník vykonal za dobu desítiletého členství v~naší jednotě mnoho záslužné práce jako jednatel, knihovník a~zvláště jako vzdělavatel. Jeho přednášky a~proslovy byly obsahem i~formou velmi cenné a~účinné. Také žákovské divadelní hry, například \emph{Jaro}, které nacvičil a~vypravil se svou chotí a~předvedl na sokolském jevišti, byly na vysoké úrovni a~dosáhly velkého úspěchu.

Jeho choť s.\ Miloslava Hurníková, rozená Bártová, sloužila Sokolu od útlého věku. Již jako malé děvčátko hrávala při sokolských představeních na klavír, později zastávala důležité místo jako cvičitelka, náčelní, jednatelka i~režisérka divadel. Velmi se uplatnila také jako herečka významných divadelních rolí.

\begin{novinovy-vystrizek}
\textbf{Rok 1930.}

\textbf{Poruba u~Svinova.} Ve čtvrtek 27.~listopadu konána členská schůze, na níž referoval br.\ starosta o~významných událostech ze sokolského i~národního života. Br.\ pokladník poukazuje na velké nedoplatky některých členů a~důrazně žádá, aby dluhující své příspěvky odvedli do konce roku. Za výběrčí zvoleni bratři Mika a~Moral. Sděleno, že v~prosinci uspořádána bude nadílka pro cvičící žactvo, o~svátcích sehráno bude divadlo a~oslaví se Silvestr. Ke konci schůze rozloučil se srdečnými slovy br.\ Ant.\ Figalla se starostou br.\ Josefem Bártou, odcházejícím do Svinova, děkuje mu za jeho 36letou činnost v~porubské jednotě a~přeje mu v~novém domově vše nejlepší. Starosta děkuje za slova uznání a~slibuje dále zůstati ve styku s~porubským Sokolem a~pro něj dle možnosti pracovati. V~neděli 30.~listopadu konal se v~sokolovně na rozloučenou se starostou veselý večer, jehož program obstaral p.\ Jaroslav Kocián, člen Mor.-slezského národního divadla, k~veliké spokojenosti všech přítomných. Za obec Porubu rozloučil se s~br.\ Bártou přítomný starosta obce p.\ Ludvík Dědoch, zdůrazňuje jeho úspěšnou činnost jako řídícího učitele ve zdejší obci a~přeje mu klidného odpočinku v~sousední obci. Bratr Bárta poděkoval jemu i~všem občanům za přízeň a~přál dalšímu rozvoji Poruby všeho zdaru.
\end{novinovy-vystrizek}

\clearpage

%%%%%%%%%%%%%%%%%%%%%%%%%%%%%%%%%%%%%%%%%%%%%%%%%%%%%%%%%%%%%%%%%%%%%%%%%%%%%%%
% ROK 1931
%%%%%%%%%%%%%%%%%%%%%%%%%%%%%%%%%%%%%%%%%%%%%%%%%%%%%%%%%%%%%%%%%%%%%%%%%%%%%%%

\phantomsection
\section{Rok 1931}

\phantomsection
\subsection{Odchod br.\ Josefa Bárty}

Tento rok začíná odchodem jednoho z~nejstarších členů --- zakladatelů jednoty. Br.\ Josef Bárta, ředitel zdejší obecné školy na odpočinku odstěhoval se z~rodinných důvodů do Svinova, načež vzdal se téměř veškeré činnosti, zůstaviv si pouze úlohu 2.~náměstka v~jednotě.

Až jako vzdálený pozorovatel a~teprve od r.\ 1929 zblízka byl jsem svědkem neúnavné práce sokolské tohoto význačného činovníka. Ať to bylo veřejné cvičení, akademie, divadlo, cvičební hodina, schůze, zájezd, zábava, nic se neodehrálo bez Josefa Bárty. Když již tělo sláblo, přece ještě obětavě hrál a~doprovázel na piano, skládal příležitostné básničky a~psal články do novin. Není funkce v~jednotě, jíž by neprošel. Vychoval též všecky děti zdárně po sokolsku.

Budiž mu těchto několik řádků věnováno z~úcty a~vděčnosti ku 40letému členství, nikoliv jako nekrolog, neboť mu přeji ještě mnoho let zaslouženého odpočinku.

\phantomsection
\subsection{Volby vedoucích}

Na valné hromadě dne 11.~ledna 1931 za účasti 10~sester a~36~bratří tj.\ 40\,\% v~sokolovně o~3.~hod.\ odpoledne zvoleni:

\begin{description}
\item[Starostou:] František Kozub.
\item[I.~náměstkem:] Eduard Jirák.
\item[II.~náměstkem:] Josef Bárta.
\item[Jednatelem:] Miroslav Kalus.
\item[Náčelníkem:] Isidor Mika.
\item[Náčelnicí:] Milada Kozubová.
\item[Pokladníkem:] Antonín Figalla.
\item[Výběrčím:] Bedřich Konečný.
\item[Matrikářem:] Otokar Schindler.
\item[Režisérem a~zapisovatelem:] Inoš Osladil.
\item[Vzdělavatelem:] Josef Konečný.
\item[Hospodářem:] Rudolf Buron.
\item[Kronikářem:] Josef Bárta.
\item[Archivářem:] František Pavlíček.
\item[Účetní dozorci:] Čeněk Konečný a~Miloslav Kozub.
\item[Zástupci do župy:] Josef Bárta a~Adolf Valder.
\item[Členy výboru:] Lida Bestová, Ludvík Šimko, Antonín Krejčí, Bohuš Bárta.
\item[Smírčí soud:] Frant.\ Kozub, Jirák Eda, Bárta Josef, Kozubová Milada, Mika Isidor.
\end{description}

Členský příspěvek stanoven na rok 1931: cvičící člen 12~Kč, necvičící 24~Kč.

\textbf{Br.\ Mirko Kalus.} Novinkou je, že byl zvolen za jednatele Mirko Kalus, jehož paní byla nájemkyní hostince v~sokolovně. Týž jest poručíkem legií a~byl pro svou milou povahu velmi vážen, po roce však odešel z~hospodářských důvodů do Lejšie.

\phantomsection
\subsection{Tělocvičná činnost}

Tělocvičná činnost za vedení houževnatého náčelníka Isidora Miky vyvrcholila veřejným cvičením dne 14.~června. V~průvodu kráčelo v~kroji 22~mužů, 5~dorostenců, 26~žáků, 7~žen, 12~dorostenek a~15~žákyň.

Cvičeno nářadí 17~mužů a~5~dorostenců, koprionová prostná 18~mužů a~17~žáků, 12~žen prostná, 12~dorostenek a~15~žákyň, pak 17~mužů, 4~dorostenců a~6~žáků provedlo skupinu na námět \enquote{práce, svornost, statečnost a~vítězství}, jež se velmi líbila. Příjem veřejného cvičení byl 3\,451,80~Kč, vydání 2\,504,65~Kč.

Mimo to pořádala jednota 2~lehkoatletické závody a~zúčastnila se v~okrsku a~župě závodů, v~nichž byl Miroslav Besta v~okrsku prvním, Mika Frant.\ třetím, v~župě Besta 12.~místo, Mika 20.~místo.

Na okrskovém veřejném cvičení 21.~června v~Třebovicích cvičilo 5~br.\ a~10~žáků. Ženy vedla náč.\ Milada Kozubová, pak 4~měs.\ Vlasta Hrubá, žákyně Em.\ Janíčková. V~Třebovicích cvičilo 5~žen, 10~dorostenek a~10~žákyň.

Tělocvičná činnost byla během roku zvýšena pod dojmem blížícího IX.~sletu.

\phantomsection
\subsection{Vzdělávací podniky}

Vzdělávací činnost v~roce 1931 byla sice poněkud slabší než v~minulém roce, avšak vzdělavatel Josef Konečný pokud mu nevadila služba železničního úředníka, vykonal čestně svou povinnost.

Výsledek: Proslovů 24, přednášky 4, divadla 4, vycházky se žactvem 1, den zábavy 2, mikulášská zábava 1; Tyršův večer a~slavnostní členská schůze.

Časopisy: Sokolské věstníky ČOS 8, Časopis Sokol 1, Besedy pro dorost 12, Vzkříšení pro žactvo 20, Župní věstníky 90. Mimo to 1~cvičitel, 1~cvičitelka, 1~vzdělavatel, 1~proslovy.

\phantomsection
\subsection{Hospodářství jednoty}

Hospodářství jednoty končí sice za r.\ 1931 příznivou zprávou pokladníkovu, leč velmi tíživě se převádí starý dluh členských nedoplatků 2\,500~Kč. Příčina tkví asi v~tom, že někteří členové odejdou z~obce a~dál se o~sokolské povinnosti nestarají, jiní platí velmi liknavě a~konečně vkrádá se do mnohých rodin již pokles výdělku nebo vůbec plná nezaměstnanost jako následek úžasné racionalisace a~úpadku vývozu na světový trh.

Zatím sice tato choroba doby nezasahuje přímo ani snad 1/10~členstva, přesto se vzmáhá a~konce nelze dohlédnouti. Proto také veškeré podniky vykazují čím dále tím menší výtěžky, jak to bylo patrno zvláště při divadle \emph{Husopaska}, jež vyžádalo si tolik práce na jevišti za vedení obětavého režiséra Inoše Osladila, avšak skončilo schodkem. Opakování této hry pak v~Kyjovicích pokaženo bylo silným deštěm.

Ukaz hospodářsky horších dob se jeví též ve snížení nájmu hostinského v~sokolovně z~10\,000~Kč na 8\,000~Kč, k~němuž družstvo pro udržování sokolovny přistoupilo. Z~téhož důvodu omezuje se družstvo na placení dluhů, provedení nejnutnějších oprav na střeše a~uvnitř a~upouští se od postavení nových záchodů, o~něž usiloval zvláště Eda Jirák, na dobu pozdější.

\phantomsection
\subsection{Pohyb členstva}

V~r.\ 1931 bylo 78~členů a~23~členek = 101. Během roku přibylo 10~mužů a~3~ženy = 13, ubylo 1~muž a~0~žen = 1. 31.~prosince 1931 je stav 87~mužů a~26~žen = \textbf{113~členů}.

\phantomsection
\subsection{Pobočka Vřesina}

Jednota má dosud pobočku Vřesinu, jež sice má dostatek členstva, totiž 43~mužů, 8~žen, z~nichž cvičí nejméně 14~muži, avšak malý výběr činovníků, proto se neuchází o~osamostatnění.

Dlužno uvésti, že v~tomto roce pobočka získává příděl lesa a~skály za úhrnnou cenu 5\,328~Kč, který zaplatila půjčkou ze záložny v~Porubě tj.\ 5\,000~Kč. Pobočku vedou učitelé Josef Martiník, starosta, Jaroslav Kučík, jednatel. Tohoto roku vedla si pobočka čile a~členstva jí přibývá.

\phantomsection
\subsection{Drobné zajímavosti}

\begin{enumerate}
\item Jednota dostává státní příspěvek ministerstva zdravotnictví a~tělesné výchovy 3\,960~Kč na opravu sokolovny.
\item \textbf{Nový úkoník.} Ustanoven k~čištění místností cvičebních Antonín Bouchal, vozač za odměnu 60~Kč měsíčně a~uzavřen pro příchod do cvičení malý sál.
\item \textbf{Prapor, vlajka.} Zahájeno na návrh Josefa Konečného jednání o~pořízení praporu a~dorostenské vlajky.
\item \textbf{Sokolské hřiště.} Uvažováno na návrh Karla Klosa o~zřízení sokolského kluziště na rybníku Vilčkově.
\item \textbf{DTJ.} Jednáno bezvýsledně s~dělnickou tělocvičnou jednotou, aby podniky obou spolků se nekonaly současně.
\end{enumerate}

\bigskip

\noindent\textit{Tím uzavírám svůj zápis a~kdyby snad v~něčem nebyl správný nebo úplný, nechť laskavě v~příštím zápise uvede se do patřičného stavu. Psal jsem s~upřímným srdcem a~s~tímže přeji jednotě nejlepší rozvoj.}

\medskip

\noindent V~Porubě 19.~února 1934\\
\textbf{Otakar Schindler}\\
t.~č.\ 1.~náměstek a~řídící učitel.

\clearpage

%%%%%%%%%%%%%%%%%%%%%%%%%%%%%%%%%%%%%%%%%%%%%%%%%%%%%%%%%%%%%%%%%%%%%%%%%%%%%%%
% ROK 1932
%%%%%%%%%%%%%%%%%%%%%%%%%%%%%%%%%%%%%%%%%%%%%%%%%%%%%%%%%%%%%%%%%%%%%%%%%%%%%%%

\phantomsection
\section{Rok 1932 (sletový)}

Valnou hromadou byli zvoleni:
\begin{description}
\item[Starostou:] František Kozub, rolník v~Porubě.
\item[Nám.\ starosty:] František Klos st., mlynář v~Porubě.
\item[Nám.\ starosty:] Jiřina Grünbaumová, choť správce tov.\ v~Porubě.
\item[Náčelníkem:] Miroslav Besta, zámečník v~Porubě.
\item[Náčelnicí:] Milada Kozubová, dcera Fr.\ Kozuba.
\item[Vzdělavatelem:] Josef Konečný, oficiál st.\ drah v~Porubě.
\item[Jednatelem:] Frant.\ Mrázek, o.~u.~Zimro Luda.
\item[Pokladníkem:] Ant.\ Figalla; o.~u.~Josef Buron.
\item[Matrikářem:] Otakar Schindler.
\item[Režisérem:] Inoš Osladil; o.~u.~Ant.\ Figalla.
\item[Archivářem:] Konečný Josef.
\item[Ostatní výb.:] Bárta Bohuš, Grünbaum Leoš, Bestová Lida.
\item[Zapisovatelem:] Buron Josef, Talver Vladimír.
\item[Hospodářem:] Buron Rudolf.
\item[Náhradníci:] Klos Antonín, Mika František vel.
\item[Smírčí soud:] Kozub František, Bárta Josef, Konečný Josef, Antonín Figalla.
\item[Vyslanci do župy:] Konečný Josef a~Zimro Luda.
\end{description}

\phantomsection
\subsection{Různé záměry}

V~roce 1932 pomýšleno v~jedné polovině také o~zřízení bruslaliště a~kluziště. Pro nedostatek vody a~velký náklad upuštěno od provedení.

Br.\ Grünbaum Leoš znovu předkládá otázku pořízení sokolského praporu. O~zakoupení toho už v~jednotě naší hovořeno dlouholetě. První se toho ujal br.\ Maras v~roce 1919 a~1920 a~druhým nositelem této myšlenky byl br.\ Konečný, leč oba nedosáhli úplného porozumění pro své záměry.

Budova sokolovny v~roce 1932 trpěla nedostatky, záchody špatné na jižní straně přímo křičely po radikální úpravě. Stalo se teprve v~roce 1933 nápravou. Okna i~dveře zevnitř byla nově natřena. Dvě lípy posadil br.\ Figalla ze školy, darovala je lesní správa slavnou Wilczkova. Od strany souseda Augusty postaven br.\ Janem Florem drátěný plot nákladem asi 320~Kč.

Též jednáno v~roce 1932 se sousedem Kružíkem o~zakoupení jeho pozemku na rozšíření cvičiště. Mimo toho by jednota získala přímý vjezd s~okresní silnice do zahrady. Požadavek sousedův byl však přemrštěný a~na takové podmínky nemohlo býti přistoupeno.

V~seznamu členstva v~roce 1932 jeví se schytek členů, který vznikl vyškrtnutím členů neplatících; na konci činila ztráta 74~mužů, 25~žen.

\bigskip

\noindent\textit{Psal jsem své první kroniku jednoty, ale psal, jak jsem pozoroval skutečný vývoj jednoty. Nechválil jsem jednotu, ale všechny, kdo kolem ní jednají.}

\medskip

\noindent V~Porubě 15.~března 1934.\\
\textbf{Antonín Figalla}\\
kronikář

\clearpage

%%%%%%%%%%%%%%%%%%%%%%%%%%%%%%%%%%%%%%%%%%%%%%%%%%%%%%%%%%%%%%%%%%%%%%%%%%%%%%%
% ROK 1933
%%%%%%%%%%%%%%%%%%%%%%%%%%%%%%%%%%%%%%%%%%%%%%%%%%%%%%%%%%%%%%%%%%%%%%%%%%%%%%%

\phantomsection
\section{Rok 1933}

Správní rok započal valnou hromadou 15.~ledna 1933. Účast na valné hromadě byla 37~osob. Z~toho 26~bratrů (35\,\%), 10~sester (40\,\%), 1~host; průměr 36,3\,\%.

Nově zvolený výbor:
\begin{description}
\item[Starosta:] František Kozub, rolník.
\item[I.~místostarosta:] František Klos, mlynář.
\item[II.~místostarosta:] Božena Grünbaumová, manželka revisora.
\item[Náčelník:] Antonín Krejčí, železničář.
\item[Náčelnice:] Milada Kozubová, dcera starosty.
\item[Vzdělavatel:] Miloslav Dvořáček, technický úředník.
\item[Členové výboru:] Figalla Antonín (učitel), Bárta Bohuslav (rolník), Schindler Otakar (řídící školy), Konečný Josef (oficiál st.\ drah), Buroň Josef (dělník), Grünbaum Leoš (správce), Mika Isidor (dělník), Janíček Zdeněk (obch.\ zaměstnanec).
\item[Náhradníci:] Hajda Karel (úředník), Valder Vladimír (dělník), Mika Valoš, Dvořáčková Idéla (učitelka).
\item[Smírčí soud:] Kozub Frant., Figalla Ant., Bárta Josef a~Bárta Bohuš.
\item[Zástupci do župy:] jednatel a~náčelník.
\item[Revisoři účtů:] Klos Jaroslav a~Moral Frant.
\item[Kronikář:] Bárta Josef.
\end{description}

Správní výbor ustanoven takto: jednatel Zd.\ Janíček, zpravodaj Ot.\ Schindler, zapisovatel Jos.\ Konečný, zdravotník Ant.\ Figalla, pokladník Jos.\ Buroň, knihovník a~archivář Rud.\ Buroň, matrikář Ot.\ Schindler, hospodář Rud.\ Buroň.

\textbf{Počet členů} jednoty na počátku roku 1933: mužů 74, žen 25, celkem 99. Úbytek v~roce 1932: 13~mužů a~1~žena, celkem tedy 14~členů, což je jistě veliký počet.

Obec Poruba čítá asi 1\,950~obyvatelů. Je tedy pouze 5\,\% obyvatelů členy sokola. To je neobyčejně málo!

Na valné hromadě odhlasován členský příspěvek: 13~Kč pro cvičící členstvo, 25~Kč pro necvičící.

K~naší jednotě patří dosud pobočka ve Vřesině, kde je ku konci roku 1932 celkem 51~členů (42~muži, 9~žen).

\phantomsection
\subsection{Cvičební činnost}

Cvičební činnost v~roce 1933 jeví znatelný vzrůst téměř u~všech složek, vyjma dorostu, kterého nebylo. Cvičitelský sbor konal řádné schůze. Celkem jich bylo v~roce 1933 --- 11. Členů cvičitelského sboru bylo 7. Čtyři z~nich udělali v~r.\ 1933 pomahatelské zkoušky.

Cvičební činnost v~roce 1933 byla o~80\,\% lepší nežli v~roce minulém.

5.~března byla veřejná cvičební hodina. Cvičilo 35~žáků a~17~mužů.

9.~července pořádáno veřejné cvičení. V~průvodu šlo 63~krojovaných bratrů. Z~naší jednoty cvičilo prostná 37~bratrů, 15~žen, 40~žáků.

Veřejné cvičení dopadlo velmi dobře, nejen po stránce cvičební, ale i~morálně a~propagačně. Na cvičišti byla vyvěšena sokolská hesla. Jako hosté zacvičilo vítkovické družstvo nářadí. Po veřejném cvičení sehrán zápas házené proti Sokolu Přívoz, jako ukázková hra.

24.9.\ byly závody o~přebor jednoty v~oddílu středním a~nižším.

26.11.\ pořádána akademie dorostu a~žactva.

23.7.\ jsme se účastnili okrskového veřejného cvičení v~Děhylově takto: 25~bratrů, 11~žen, 22~žáků, 12~žaček.

29.10.\ na náhlém srazu sokolstva v~Hlučíně bylo z~naší jednoty 65~bratrů (7~v~kroji) a~1~dorostenec, žen 8. Celkem tedy 74~osoby, což jest 57\,\% všeho členstva.

V~předvečer tohoto srazu 28.10.\ zapálena na polích statkáře Wilczka mezi Svinovem a~Porubou hranice za účasti hasičského sboru z~Poruby a~velkého počtu lidí (asi 200). U~této hranice promluvil vzdělavatel Dvořáček. Tímto způsobem oslaven den našeho osvobození.

\phantomsection
\subsection{Vzdělávací činnost}

Vzdělávací činnost byla vedena v~prvé řadě v~duchu výchovném, pak v~duchu propagačním. Je třeba v~naší jednotě v~prvé řadě vychovávati z~členů --- sokoly.

Statisticky byla vzdělávací činnost následující: sokolských proslovů bylo 13~pro členstvo, 19~pro žactvo, celkem 32. Přednášek 4~pro členstvo. 6~divadelních představení, z~toho 2~kusy sehráli herci domácí, 4~kusy hosté.

Pro žactvo hráno 5~loutkových her. Loutkové divadélko bylo dáno do pořádku, pořízena nová skříň, ve které uloženy všechny rekvizity s~loutkami.

\textbf{Vzdělávací škola.} V~listopadu pořádána řádná vzdělávací škola pro nově přistouplé členy. Škola potvrzena na 4~večery takto:
\begin{enumerate}
\item \enquote{O~sokolské myšlence}
\item \enquote{Sokolstvo a~jeho poměr k~politice}
\item \enquote{Sokolstvo a~jeho poměr k~náboženství}
\item \enquote{Sokolská organisace} a~\enquote{Tělocvik a~jeho význam}
\end{enumerate}

Školu prodělalo 15~mužů a~11~žen, celkem 26~členů nových. 12~členů zúčastnilo se jako hosté. Též navštívili školu: 1~dorostenec a~3~dorostenky. Všechny večery přednášel vzdělavatel Dvořáček. Tato ideová škola byla první vzdělávací školou v~naší jednotě.

\phantomsection
\subsection{Hry}

V~jednotě pěstuje se rovněž házená, odbíjená a~stolní tennis. Aby tyto hry se mohly zdárně rozvíjeti, zřízena t.~zv.\ \enquote{skupina her}, pro kterou byly vypracovány stanovy, schválené výborem 25.11.1933. Tento řád vypracovali bratři Klos Ota, Janíček Zd.\ a~Dvořáček.

Házená se počala v~porubské jednotě pěstovat asi v~roce 1927, odbíjená od r.\ 1932, stolní tenis od r.\ 1933.

Pro házenou upraveno vhodné hřiště, které má jednota pronajato od br.\ Bohuše Bárty.

V~tomto roce poprvé zřízeno na rybníku statkáře Wilczka sokolské kluziště. Na kluzišti zavedeno světlo, takže se bruslilo i~večer. Byl to vděčný podnik pro osvěžení nejen členstva, ale i~obecenstva nesokolského.

\phantomsection
\subsection{Další události}

V~tomto roce rovněž byly vystavěny Družstvem pro udržování sokolského domu v~Porubě nové záchody. Tím byla konečně vyřešena velká hygienická závada, která nikterak nepřispívala k~dobrému jménu sokolovny.

V~tomto roce odhlasovány valnou hromadou jednoty nové stanovy jednoty, jež byly schváleny zemským úřadem.

Jednota se připravuje k~důstojné oslavě 40letého trvání jednoty. V~roce 1934 to bude 40~let, kdy byla založena. Pomýšleno, na návrh br.\ Leoše Grünbauma, aby byl v~tomto roce na paměť 40letí odhalen prapor jednoty. Tato myšlenka se přetřásá v~jednotě již od r.\ 1925. Akce svěřena br.\ L.\ Grünbaumovi, který měl poříditi rozpočet a~návrhy některých umělců.

\textbf{Stolní tennis.} Pro stolní tennis zakoupen stůl, který zhotovila fa Blažej.

\phantomsection
\subsection{Fotografie z~roku 1933}

\begin{figure}[H]
\centering
\adjustimage{width=\textwidth, height=\textheight, keepaspectratio}{images/stara_kronika_transcript_fotky/3.png}
\caption{Tělocvična roku 1933. V~pozadí vlevo je pianino jednoty. Před jevištěm hrazda, připravená ke cvičení, v~popředí je vidět kruhy.}
\end{figure}

\begin{figure}[H]
\centering
\adjustimage{width=\textwidth, height=\textheight, keepaspectratio}{images/stara_kronika_transcript_fotky/4.png}
\caption{Hřiště pro házenou u~sokolovny. Vlevo je viděti stojan pro kruhy, u~něho sloup s~lampou pro osvětlení hřiště. Uprostřed jsou sloupky se sítí na odbíjenou.}
\end{figure}

\begin{figure}[H]
\centering
\adjustimage{width=\textwidth, height=\textheight, keepaspectratio}{images/stara_kronika_transcript_fotky/5.png}
\caption{Skupina bratrů pracujících na hřišti pro odbíjenou. V~pozadí zahrada, vpravo sokolovna.}
\end{figure}

\begin{figure}[H]
\centering
\adjustimage{width=\textwidth, height=\textheight, keepaspectratio}{images/stara_kronika_transcript_fotky/6.png}
\caption{Sokolské kluziště na rybníku statkáře Wilczka.}
\end{figure}

\begin{figure}[H]
\centering
\adjustimage{width=\textwidth, height=\textheight, keepaspectratio}{images/stara_kronika_transcript_fotky/7.png}
\caption{Závěrečná skupina veřejného cvičení. Muži a~ženy v~krojích vpochodovali na cvičiště za zpěvu sokolské pochodové Tyršovy hymny \enquote{V~nový život}. Na můstku stojící náčelník Krejčí Ant.\ vztyčil za zvuku národních hymen státní vlajku.}
\end{figure}

\phantomsection
\subsection{Stav tělocvičny a~nářadí}

Tělocvična naše, sokolský náš chrám je již ve velmi špatném stavu. Stěny u~podlahy moknou, malba je velmi již poškozena hřebíky, nářadím aj. Podlaha, věčně zaprášená, je obyčejná, prkenná.

Nářadí: 1~hrazda, 1~kruhy, 2~bradla, 1~kůň, 2~činky, pružný můstek (obyčejný můstek), šplh (lano), 2~žíněnky, několik kuželů, malých žídel, 2~míče, 2~žerdě rapiry. Všechno v~nedbalém stavu.

Na nářadí a~náčiní se věnuje velmi málo, téměř nic. V~rozpočtu se ještě nezvyklo pamatovat na pořízení nového nářadí, nebo aspoň na opravu a~podržování starého.

\bigskip

\noindent\textit{Píši poprvé do kroniky jednoty. Možná, že jsem věc neudělal úplně správně, ale první prací se školíme nejvíce. Úmyslně jsem věnoval několik řádků naší tělocvičně a~hřišti. Zvláště o~hrách nebylo dosud v~kronice zmínky. A~ty začínají býti důležitou složkou naší tělesné i~mravní výchovy.}

\medskip

\noindent\textit{Naše jednota vykazuje sice značnou činnost, ale na svou dlouholetou tradici (sokolovna je již od r.\ 1901) je pozadu co do počtu členstva i~co do ducha v~životě jednoty. Mezi tělocvičnou t.j.\ cvičícím členstvem a~výborem jednoty je velká mezera. Tělocvičně a~jejímu nářadí a~náčiní se nevěnuje téměř žádná pozornost. Divadlo nemusí vychovávat, ale vydělávat. Proto také se věnuje větší pozornost restauraci. A~proto se peníze věnují v~prvé řadě hospodě na přestavbu a~úpravu místností. A~tělocvična čeká. Chátrá, rezaví, hnije.}

\medskip

\noindent\textit{Něco zde není v~pořádku. Chybí zde skutečný sokolský duch, uvědomění, že jsme spolkem tělocvičným a~výchovným a~žádným jiným. Věřím, že jednou bude lépe, že se to obrátí k~lepšímu, na cestu Tyršovu a~Fügnerovu.}

\medskip

\noindent\textbf{Co Čech to Sokol, co Sokol, to lidový šlechtic.}

\medskip

\noindent V~Porubě v~roce 1935.\\
\textbf{Miloslav Dvořáček}

\bigskip

\begin{center}
\fbox{\begin{minipage}{0.8\textwidth}
\centering
\textbf{Stvrzenka}\\
\medskip
Za fotografie pro kroniku zaplatil Kč 40-50 h., obdržel Kč \dots h.\\
Kč čtyřicet korun padesát hal. 50/100\\
za vybráno na valné hromadě od členů\\
\medskip
Dne 13./2. 1938\\
\bigskip
Pokladník: [Podpis] \hfill Podpis: [Podpis]\\
\raggedleft \footnotesize 60300-31.
\end{minipage}}
\end{center}

\clearpage

%%%%%%%%%%%%%%%%%%%%%%%%%%%%%%%%%%%%%%%%%%%%%%%%%%%%%%%%%%%%%%%%%%%%%%%%%%%%%%%
% ROK 1934 - 40leté jubileum
%%%%%%%%%%%%%%%%%%%%%%%%%%%%%%%%%%%%%%%%%%%%%%%%%%%%%%%%%%%%%%%%%%%%%%%%%%%%%%%

\phantomsection
\section{Rok 1934 --- 40leté jubileum trvání}

Tento rok vyznačoval se opět největším ruchem tělocvičným, vzdělávacím a~kladl na vedení tím větší požadavky, protože jednota slavila čtyřicet let svého trvání.

Již v~několika výborových schůzích a~na lednové valné hromadě vypracovalo členstvo program oslavy.

\phantomsection
\subsection{Vedoucí činovníci}

Vedení jednoty ujali se mimo některé nepatrné změny titíž činovníci jako loni. Tedy valnou hromadou 14.~ledna byli zvoleni:
\begin{description}
\item[Starostou:] František Kozub.
\item[Jednatelem:] Zdeněk Janíček.
\item[Náčelníkem:] Mika Isidor.
\item[Náčelnicí:] Milada Kozubová.
\item[Vzdělavatelem:] Miloš Dvořáček.
\end{description}

Volba byla šťastná z~toho důvodu, že duší jednoty náčelnictvo a~jednatel byli lidé pilní a~svědomití. Bylo třeba nejen činnosti tělocvičné, ale také pořádati podniky výnosné, aby se umořovaly staré dluhy a~nové naléhavé opravy provésti.

\phantomsection
\subsection{Podniky}

Začalo se tancem 6.~ledna = ples a~rok byl ukončen opět tancem = Silvestr.

Z~tělocvičných podniků jmenuji zdařilé veřejné cvičení 3.~června, jež zahájeno mohutným průvodem 116~krojovaných, z~toho 67~našich, ostatní přespolní. V~této době, kdy je všeobecný nedostatek krojů, ještě je to úspěch. Též finančně úspěch veřejného cvičení a~přilehlých podniků přes 1\,500~Kč čistého příjmu je pěkný!

Též pobočka Vřesina pomocí porubských členů měla 8.~srpna veřejné cvičení.

Dalším podnikem byla 2.~prosince tělocvičná akademie opět dobře provedena.

Jednota obeslala i~jiné tělocvičné podniky jako ve Svinově, v~Opavě župní slet, závody v~Klimkovicích a~Polance. Nejlepšími cvičenci jsou bři Valoš a~Isidor Mika, sestry Milada a~Drahuše Kozubovy.

Nemohu také opomenouti rušného turnaje hry \enquote{podbíjené}, z~něhož vítězně vyšlo z~družstev Opava dne 29.~července a~12.~srpna turnaj \enquote{házené}, z~5~družstev zvítězili mistři Svinova. Duší této krásné herní činnosti jest br.\ Miloš Dvořáček.

Podniky vzdělávací byly hojnější proti loňsku, např.\ proslovů 63, t.j.\ o~100\,\% více, divadla vedl br.\ Dvořáček celkem 5, z~toho sehráli cizí jen 1. Jejich úroveň byla však dobrá i~veřejností pěkně navštívena. Br.\ vzdělavatel měl úspěch v~těchto podnicích již proto, že sám je cvičícím členem a~neustále ve styku s~členstvem.

\phantomsection
\subsection{Sokolská vlajka}

Praporová akce, která minulý rok zásluhou br.\ Leoše Grünbauma dostoupila vrcholu tím, že upsáno téměř 1\,000~Kč příspěvků, nevedla sice k~pořízení praporu pro svůj vysoký náklad přes 4\,000~Kč, avšak výbor dne 25.~ledna odhlasoval místo praporu vlajku. Vlajka ponese stuhu s~nápisem \enquote{1894~\dag~1934 Sokol Poruba u~Svinova}.

První praporečník ustanoven br.\ Buroň Rudolf, strojník č.\ 125 a~jeho zástupce br.\ Jindřich Novák. O~nošení vlajky obdržela jednota přísné instrukce, podle nichž zejména není dovoleno nositi na vlajce stuhy a~v~průvodu musí jíti vlajka s~čestným doprovodem první v~čele.

Cena vlajky je okrouhle 1\,000~Kč, jest uschována v~sokolovně. Zhotovila ji odb.\ stk.\ pro zem.\ povolání ve Vítkovicích za režijní náklady --- 341~Kč.

\phantomsection
\subsection{Sokolské kluziště}

Ve schůzi dne 8.~listopadu jednáno o~zřízení sokolského kluziště na rybníku velkostatkáře Wilczka u~skladiště fy Blažej. Zařízeno elektrické osvětlení fi.\ Blažej a~pokud byl dobrý led, bylo kluziště hojně mládeží i~staršími navštěvováno. Obtížné jiné udržování kluziště a~vybírání poplatků, které dobře obstarali bři Brandstättři. Finančně však kluziště nepřineslo jednotě žádného užitku.

\phantomsection
\subsection{Doslov}

Jubilejní rok 1934, po všech stránkách tak rušný, minul a~zbyly jen krásné vzpomínky a~fotografie.

Ještě by bylo záhodno zmíniti se o~vzrůstající branné výchově vyvolané výhružným chováním Německa a~Maďarska, o~krisi hospodářské, jež způsobila dlouholetou (od r.\ 1930) nezaměstnanost, takže jednota asi 10\,\% členstva z~řad mužů chová a~jimž snaží se pomoci. I~jiné akce, např.\ prodej srdíček mají pomoci nezaměstnaným.

Pozoruhodné je také, že Družstvo si půjčuje 25\,000~Kč od Čes.\ akc.\ pivovaru v~Mor.\ Ostravě na úhradu přestavby bytu provedené v~minulém roce.

\bigskip

\noindent\textit{Končím zápis a~prosím za shovívavost, jestli jsem nebyl dosti úplný a~věrný, měl jsem totiž tolik jiných úkolů, jež mne velmi zdržovaly v~psané této kroniky.}

\medskip

\noindent\textit{Zdar jednotě a~všem mým následovníkům.}

\medskip

\noindent Dokončeno 2.~února 1938.\\
\textbf{Otakar Schindler}

\begin{novinovy-vystrizek}
\textbf{Sokol Poruba u~Svinova.} V~neděli 28.~září sehrán byl na sokolském hřišti turnaj házené, jehož se zúčastnily jednoty Polanka, Svinov, Smolkov, Klimkovice, Michálkovice a~Poruba. Zápasu přihlíželo se zájmem četné obecenstvo. Vítězství si odnesly jednoty Polanka a~Svinov. Na počest hostů a~na rozloučenou s~odvedenými bratry byl večer v~sokolovně pořádán kabaret. V~jednoaktovce \enquote{Ich melde} dobře se uplatnili bratři: Ervín Motl, Zd.\ Janíček a~Vráťa Blažej a~sestry: Milada Kozubová a~Mařenka Bestová. Komický výstup \enquote{Pan lajtnant a~jeho prefatýna} vyvolal veselost. Krásné zpěvy sl.\ Ed.\ Scheichové a~br.\ G.\ Grohmanna přispěly k~úspěchu večera. Po vhodném proslovu br.\ Janíčka byly vítězům odevzdány odměny: Polance vlajka a~Svinovu diplom. Starosta br.\ Bárta blahopřál jim za jednotu a~srdečnými slovy se rozloučil s~odvedenými bratry. --- V~neděli 5.~října večer bude v~sokolovně vinobraní.

\hfill J.~B.
\end{novinovy-vystrizek}

\clearpage

%%%%%%%%%%%%%%%%%%%%%%%%%%%%%%%%%%%%%%%%%%%%%%%%%%%%%%%%%%%%%%%%%%%%%%%%%%%%%%%
% POŘAD SLAVNOSTNÍ VALNÉ HROMADY - 40. VÝROČÍ
%%%%%%%%%%%%%%%%%%%%%%%%%%%%%%%%%%%%%%%%%%%%%%%%%%%%%%%%%%%%%%%%%%%%%%%%%%%%%%%

\phantomsection
\section{Pořad slavnostní valné hromady}

\begin{center}
\textbf{konané na oslavu 40.letého trvání těl.~jed.\ Sokol Poruba u~Svinova,}\\
\textbf{27.~května 1934 v~sokolovně.}
\end{center}

\bigskip

\noindent Od 9.~hod.\ zápis do listiny přítomných.

\bigskip

\noindent Od 10.~hod.:
\begin{description}
\item[Zahájení a~uvítání přítomných] \dotfill Frant.\ Kozub
\item[Čtení došlých přípisu] \dotfill Zdena Janíček
\item[Čtení zápisu ustavující schůze] \dotfill Ant.\ Figalla
\item[Stručný přehled historie jednoty] \dotfill Ant.\ Figalla
\item[Vzpomínky ze života jednoty] \dotfill Adolf Bárta
\item[Předání plaket zakládajícím a~zasloužilým členům] \dotfill Ot.\ Schindler
\item[Projevy hostů] ~
\item[Odeslání telegramu Č.O.S.\ a~Sl.~Ž.M.S.] ~
\item[Státní hymny] ~
\item[Ukončení] \dotfill Frant.\ Kozub
\end{description}

\bigskip

\noindent P.S.\ Od 14.~hod.\ velký zahradní koncert.

\bigskip

\begin{flushright}
\textbf{František Kozub v.r.}\\
t.~č.\ starosta\\[1em]
\textbf{Zdena Janíček v.r.}\\
t.~č.\ jednatel
\end{flushright}

